\section*{Úvod}

V posledních letech došlo k výrazným změnám v oblasti vývoje webu. 
Z jednoduchých statických stránek jsme se přesunuli ke komplexním, interaktivním, jednostránkovým aplikacím (SPA). 
Webové aplikace jsou vyvíjeny s pomocí moderních technologií nebo frameworků, které vývojářům nabízí nástroje pro rychlejší a efektivnější vývoj. 
Výběr vhodných technologií je velice důležitý, protože může mít vliv na celkovou kvalitu a úspěch projektu. 
Nesprávný výběr technologie může vést ke složitým řešením, ztrátě času či finančních prostředků, nebo k problémům, které v budoucnu stíží práci vývojovým týmům.

Tato bakalářská práce se zaměřuje na analýzu a porovnání technologií určených k vývoji frontendu moderních webových aplikací. 
Cílem práce je poskytnout čtenáři ucelený pohled na vybrané SPA technologie a na základě implementace demonstrační aplikace porovnat jednotlivé frameworky.

V úvodní části práce představíme základní pojmy týkající se moderního vývoje jednostránkových aplikací. 
V rámci teoretické části uvedeme motivaci pro výběr SPA technologií, zaměříme se na analýzu vybraných frameworků a také provedeme porovnání analyzovaných nástrojů. 
V praktické části práce navrhneme webovou aplikaci, která umožní odhalit klíčové vlastnosti a rozdíly mezi vybranými technologiemi. 
Dále provedeme implementaci aplikace v rámci vybraných technologií a na závěr provedeme srovnání jednotlivých implementací.

U čtenáře předpokládáme základní znalosti v oblasti webových technologií, dále také praktické zkušenosti s jazyky HTML, CSS a JavaScript. 
Tato práce čtenáři nabídne jak teoretický přehled, tak i praktické zkušenosti s moderními technologiemi frontendu, 
což by mělo čtenářům usnadnit výběr vhodné technologie pro jejich budoucí projekty a zároveň se lépe orientovat ve světě moderního vývoje webu.

% V~Úvodu především rozvedeme cíl práce (ten najdeme v~zadání tématu práce), 
% můžeme poněkud méně formálně o~tématu povykládat (ale nepřehánějte to, žádných 10 stran úvodu prosím).
% Můžeme psát o~své motivaci, tedy proč jsme si téma zvolili, proč je považujeme za zajímavé či důležité. Můžeme také jemně uvést čtenáře do problematiky.

% Je zvykem zde psát o~tom, z~jakých částí se práce skládá. 
% Není nutné jít po kapitolách, můžeme například napsat, že práce má teoretickou a~praktickou část, 
% přičemž v~teoretické části je čtenář nejdřív seznámen s~problematikou xxx, jsou zde vysvětleny základní pojmy a~v~několika kapitolách nejběžnější metody používané pro yyy. 
% V~praktické části je popsána aplikace zzz sloužící k~qqqq, najdeme zde manuál k~jejímu používání s~postupem instalace a~zprovoznění a~také popis její vnitřní struktury a~okomentované ukázky kódu. 
% NEBO: Praktickou částí je srovnání metod sloužících k~rrrrr, přičemž po analýze metod daného typu se zdůvodněním výběru a~metodikou pro jejich porovnání jsou v~jednotlivých kapitolách popsány vybrané metody a~v~poslední kapitole najdeme jejich vzájemné srovnání. 
% NEBO: V~první kapitole rozebíráme typické požadavky na sociální sítě/informační systémy/webové aplikace/… pro účel stanovený v~zadání, 
% v~následující kapitole nastiňujeme momentální stav co se týče existujících produktů pro daný účel a~hodnotíme, do jaké míry splňují stanovené požadavky. 
% Další kapitoly obsahují návrh vlastní xxxxx s~popisem jak samotné xxxx, jejího zprovoznění, rozhraní, atd., 
% tak i~postup vytvoření (byl použit programovací jazyk zzzz), a~opět zhodnocení míry splnění stanovených požadavků.

% Tato část úvodu má čtenáře připravit na vlastní obsah práce, tak si dejte záležet, ať čtenáře neotrávíte předem :-)

% V~Úvodu dále můžeme připsat informaci o~tom, že obrázky bez uvedeného zdroje byly vytvořeny v~nástroji xxxxx, 
% případně také že u~čtenáře předpokládáme alespoň základní znalosti v~oblasti 
% \dots{} (programování, počítačových sítí, informačních systémů, sociálních sítí, tvorby webových stránek, atd. podle tématu), 
% abyste nemuseli vysvětlovat ty nejzákladnější pojmy.
