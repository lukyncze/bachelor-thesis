\section*{Závěr}

V bakalářské práci jsme se zaměřili na analýzu a porovnání současných možností frontendového vývoje. 
Navrhli a implementovali jsme demonstrační aplikace, díky kterým jsme identifikovali možnosti a výhody použití vybraných moderních frontendových technologií. 
V neposlední řadě jsme provedli závěrečné srovnání jednotlivých implementací.

Samotná analýza frameworků Angular, React, Svelte a Vue odhalila rozdíly ve způsobech, jakými jednotlivé technologie přistupují k vývoji frontendu. 
Každá technologie disponuje jedinečnými vlastnostmi, které mohou odlišným způsobem ovlivnit vývoj webových aplikací. 

V praktické části jsme provedli implementaci aplikací pomocí tří vybraných technologií -- Angular, React a Svelte, aplikace jsme zveřejnili na web pomocí platformy Netlify. 
Na základě implementace jsme provedli srovnání, ve kterém jsme zjistili, že každá technologie má své přednosti a nedostatky. 
Výsledky srovnání ukázaly, že výběr vhodné technologie závisí na konkrétních požadavcích a cílech projektu.

Zatímco Angular je vhodný spíše k vývoji velkých aplikací, React vyniká ve flexibilitě a široké podpoře komunity. 
V nespolední řadě Svelte překvapil svým minimalismem, jednoduchostí, ale také vysokou efektivitou.

Závěrem lze konstatovat, že nejuniverzálnější framework pro vývoj frontendu neexistuje. 
Práce splnila svůj cíl, kterým bylo poskytnout čtenáři ucelený pohled na frontendové technologie, jejich možnosti a srovnání.
Rovněž byla splněna motivace práce, která spočívala v usnadnění výběru vhodného nástroje pro vývoj frontendu čtenáři.

V práci jsme se řešili problémy při vykreslení více než jednoho rozevíracího seznamu v rámci jedné stránky. 
Problematické bylo také nasazení aplikací na server, kde bylo nutné provést úpravy v konfiguraci, aby aplikace fungovaly dle očekávání.

Budoucí rozšíření práce by mohlo spočívat v přidání testovacích scénářů, na které během práce nebyl kladen důraz, například přidáním unit a integračních testů.
Dalším rozšířením by také mohlo být přidání backendového systému, který by umožnil porovnání možností integrace s backendovými technologiemi.

% Do Závěru píšeme souhrn poznatků zjištěných v~práci, hodnotíme výsledek práce (někde by tu měla být i~větička \uv{Cíl práce byl splněn}, nějak vhodně rozvedená). 
% Také tu můžeme psát o~případných problémech, se kterými jsme se při psaní práce setkali, 
% možnostech využití, námětech na pokračování do budoucna (tj. co by se dalo zlepšit, přidat, rozšířit, atd.).