\subsection{Vue}

% Fullstack Vue.js book
% https://vuejs.org/
% https://www.w3schools.com/vue/
% https://developer.mozilla.org/en-US/docs/Learn/Tools_and_testing/Client-side_JavaScript_frameworks/Vue_getting_started
% https://www.tutorialspoint.com/vuejs/vuejs_overview.htm
% https://www.itnetwork.cz/javascript/vuejs/uvod-do-vuejs-a-prvni-aplikace
% https://worldline.github.io/vuejs-training/
% https://www.rascasone.com/cs/blog/co-je-framework-vuejs

Vue dostalo svůj název díky anglickému slovu view. Jedná se o deklarativní JavaScriptový open-source framework. 
Je určen efektivní tvorbě jak jednoduchých, tak i komplexních uživatelských rozhraní na webu. 
Framework je v současné době jedním z nejpopulárnějších frameworků pro tvorbu webových aplikací.\cite{vuemacrae,vue}

Evan You, tvůrce Vue, se inspiroval určitými částmi frameworku AngularJS, který však měl velmi strmou křivku učení. 
Vue tedy mělo být lehké, přizpůsobivé a snadné k naučení. Bylo vytvořeno roku 2013, uvolněno do světa až o rok později. 
Od té doby byly vydány pouze 3 majoritní verze, avšak ty přinesly mnoho změn.\cite{vueflexiple,vuemedium}

Vue nabízí svobodnou volbu při tvorbě komponent ve formě dvou hlavních API -- Options a Composition API. 
Options API můžeme přirovnat k objektovému přístupu, v porovnání s Composition API, jež využívá funkcionální přístup. 
Podle \cite{vue} Composition API přináší větší flexibilitu a umožňuje efektivnější návrhové vzory pro organizaci a znovupoužitelnost kódu. 
Analýza bude probíhat pomocí Composition API.

Framework klade důraz na progresivitu, což znamená, že roste s vývojářem a přizpůsobuje se jeho potřebám. 
Díky tomu si Vue oblíbily společnosti jako Xiaomi, Adobe, Gitlab, Trivago, BMW.\cite{vuetriodev,vue}

\subsubsection{Single-File Components}

Základní funkcí Vue jsou tzv. Single-File Components (SFC). Jedná se o hlavní stavební blok frameworku, který reprezenzuje část webové stránky. 
Komponenta se skládá z šablony, dat komponenty, funkcí a kaskádových stylů. Hlavní výhodu tohoto přístupu představuje znovupoužitelnost. 
JavaScriptové funkce musí být zapsány mezi párové značky script s atributem setup, šablona do template tagů a styly do style bloku.\cite{vuemacrae,vue}

\subsubsection{Reaktivita}

V komponentě můžeme uchovávat informace pomocí reaktivních stavů. Oficiální dokumentace doporučuje používat funkci ref, která vyžaduje počáteční hodnotu. 
K hodnotě stavu pak v rámci skriptu přistupujeme pomocí klíčového slova value. V šabloně nám stačí pouze název stavu. 
Modifikaci stavu lze provést pomocí přiřazení nové hodnoty. Při změně jakéhokoli stavu komponenty pak Vue automaticky aktualizuje DOM s novými daty.\cite{vue}

\subsubsection{Předávání vlastností}

Komponenty spolu komunikují pomocí předávání vlastností. V parent komponentě je třeba předat požadovanou hodnotu do proměnné child komponenty. 
V child komponentě pak definujeme props vlastnost, kterou vytvoříme pomocí funkce defineProps. 
DefineProps funkce vyžaduje objekt s názvem a datovým typem předávané vlastnosti.

Pro předání vlastnosti z child to parent komponenty se využívá tzv. emitování eventů. 
V potomku vytvoříme vlastnost emit, v níž nadeklarujeme pole emitovaných hodnot pomocí funkce defineEmits. Dále je třeba definovat jednotlivé emity. 
První argumentem je název emitu, další argumenty jsou již předávané hodnoty. Rodičovská komponenta musí naslouchat na emitované eventy. 
Toho lze docílit pomocí @response direktivy, která typicky v callback funkci přeukládá argumenty na lokalní stavy.\cite{vuemacrae,vue}

\subsubsection{Direktivy a eventy}

Framework disponuje mnoha direktivami, které umožňují přidávat do šablony různé funkce. Logiku vykreslování umožňují direktivy v-if, v-else-if a v-else. 
Pro iteraci přes pole hodnot slouží v-for. Mezi další užitečné direktivy patří v-bind a v-model. 
Díky v-bind je možné přidat jakémukoli elementu dynamickou hodnotu atributu, direktivu můžeme zkrátit i pomocí dvojtečky. 
Direktiva v-model zase zajistí obousměrné propojení pro formulářové prvky.

Vue také umožnuje naslouchat na DOM eventy pomocí direktivy v-on. Ta pak vyžaduje libovolný DOM event a callback funkci, jež se vykoná při daném DOM eventu. 
Můžeme také zvolit ekvivalentní zápis s @. Další možnost představuje využití modifkátorů eventů, které se postarají například o vypnutí výchozího chování prvku.\cite{vuemacrae,vue}

\subsubsection{Životní cyklus}

Každá instance Vue komponenty má daný životní cyklus. Ten můžeme rozdělit na 3 části -- inicializační část, část, při které se mění data a část, kdy komponenta zaniká. 
Pro zajištění kontroly životního cyklu slouží tzv. hooks, které se vyznačují tím, že vždy přes svým názvem mají předponu on.

Při inicializaci komponenty můžeme využít hook akce beforeMount či mount. BeforeMount se volá ještě před tím, než se komponenta přidá na stránku. 
Mount až poté, kdy je vytvořen element komponenty -- komponenta však ještě nemusý být v DOM. 
Před chystanou změnou dat v DOM se volá beforeUpdate, po vykonání změny je zavoláno updated. Před samotným zánikem komponenty Vue volá beforeUnmount. 
Po dokončení zničení se pak volá unmounted.\cite{vuemacrae,vue}
% super obrázek: https://vuejs.org/guide/essentials/lifecycle.html#lifecycle-diagram

\subsubsection{State management}
% Vuex - https://vuex.vuejs.org/
\subsubsection{Routování}
% Vue Router - https://router.vuejs.org/
\subsubsection{Ekosystém}