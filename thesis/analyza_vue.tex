\subsection{Vue}

% Fullstack Vue.js book
% https://vuejs.org/
% https://www.w3schools.com/vue/
% https://developer.mozilla.org/en-US/docs/Learn/Tools_and_testing/Client-side_JavaScript_frameworks/Vue_getting_started
% https://www.tutorialspoint.com/vuejs/vuejs_overview.htm
% https://www.itnetwork.cz/javascript/vuejs/uvod-do-vuejs-a-prvni-aplikace
% https://worldline.github.io/vuejs-training/
% https://www.rascasone.com/cs/blog/co-je-framework-vuejs

Vue dostalo svůj název díky anglickému slovu view. Jedná se o deklarativní JavaScriptový open-source framework. 
Je určen efektivní tvorbě jak jednoduchých, tak i komplexních uživatelských rozhraní na webu. 
Framework je v současné době jedním z nejpopulárnějších frameworků pro tvorbu webových aplikací.\cite{vuemacrae,vue}

Evan You, tvůrce Vue, se inspiroval určitými částmi frameworku AngularJS, který však měl velmi strmou křivku učení. 
Vue tedy mělo být lehké, přizpůsobivé a snadné k naučení. Bylo vytvořeno roku 2013, uvolněno do světa až o rok později. 
Od té doby byly vydány pouze 3 majoritní verze, avšak ty přinesly mnoho změn.\cite{vueflexiple,vuemedium}

Vue nabízí svobodnou volbu při tvorbě komponent ve formě dvou hlavních API -- Options a Composition API. 
Options API můžeme přirovnat k objektovému přístupu, v porovnání s Composition API, jež využívá funkcionální přístup. 
Podle \cite{vue} Composition API přináší větší flexibilitu a umožňuje efektivnější návrhové vzory pro organizaci a znovupoužitelnost kódu.

Framework klade důraz na progresivitu, což znamená, že roste s vývojářem a přizpůsobuje se jeho potřebám. 
Díky tomu si Vue oblíbily společnosti jako Xiaomi, Adobe, Gitlab, Trivago, BMW.\cite{vuetriodev,vue}

\subsubsection{Komponenty}

Single-File Component (SFC)

\subsubsection{Reaktivita}
\subsubsection{Předávání vlastností}
\subsubsection{Eventy???}
\subsubsection{Životní cyklus}
\subsubsection{State management}
% Vuex - https://vuex.vuejs.org/
\subsubsection{Routování}
% Vue Router - https://router.vuejs.org/
\subsubsection{Ekosystém}