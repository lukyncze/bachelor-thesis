\documentclass{vdocdiplcz}

\usepackage[czech]{babel}

%\usepackage[cp1250]{inputenc}  % vyberte podle toho, jaké kódování používáte - jestli ``Windows'' nebo UTF-8
%
\usepackage[utf8]{inputenc}
\usepackage[T1]{fontenc}
%\usepackage{comment}   % komentare pres vice radku \begin{comment} .... \end{comment}
%
\usepackage{graphicx}

\usepackage{alltt}  % pro zapis programoveho kodu: jestlize nepouzivate, klidne odstrante
\usepackage{pdfpages}
\usepackage{tablefootnote}

% nastaveni barev a dalsich vlastnosti pro interaktivni odkazy:
\usepackage{color}
\definecolor{seda}{rgb}{.5,.5,.5}
\definecolor{modra}{rgb}{.02,.02,.4}
\definecolor{zelena}{rgb}{.02,.4,.01}
\definecolor{fialova}{rgb}{.5,.05,.5}
\definecolor{zluta}{rgb}{.5,.5,.05}
\definecolor{cervena}{rgb}{.7,.2,.2}
\definecolor{oranzova}{rgb}{.8,.4,.05}
\usepackage[bookmarkstype={toc},colorlinks,bookmarksnumbered,%
unicode,linkcolor={modra},%
citecolor={zelena},urlcolor={modra}]{hyperref}

% nastaveni zvyrazneni pozadi
% pouziti:
% \begin{zvyraznenyodstavec}
%   text, ktery chceme zvyraznit
% \end{zvyraznenyodstavec}
\usepackage{mdframed}
\newmdenv[backgroundcolor=yellow]{zvyraznenyodstavec}

%\usepackage{picins,wrapfig}   % obtekani obrazku - picins pro obrazky bez popisku, wrapfig pro obrazky s popiskem
%\usepackage{multicol}    % vicesloupcova sazba, pokud ji potrebujeme
%
%\usepackage{subfigure}  % sada zvlast oznacenych objektu v jednom figure
%\usepackage{appendix}  % prilohy, obvykle neni nutne

%\usepackage{textcomp,gensymb,bbding,amsfonts,mathrsfs}  									%  balicky pro symboly
%\usepackage{wasysym,latexsym,gensymb,pifont,phonetic,mathcomp,fancybox}	% balicky se symboly



%******************************************************************************
% pokud pracujeme na konkretni kapitole a chceme prekladat jenom ji, pouzijeme:
%******************************************************************************
%\includeonly{kapitola01_nazev}      % nebo vice kapitol podle vyberu:
%\includeonly{uvodni_kapitola,kapitola01_nazev}
% \includeonly{testovani_frameworku} % pouze 1 kapitola


\autor{Lukáš Sukeník}
\vedouci{doc. RNDr. Lucie Ciencialová, Ph.D.}
\nazev{Porovnání SPA frontend frameworků}
\nazevanglicky{Comparison of SPA frontend frameworks}
\rok{2024}   % pozor, musi byt rok obhajoby, nikoliv rok, kdy byla prace vytisknuta
\typprace{Bakalářská práce}    % Bakalářská práce, Diplomová práce, Rigorózní práce, Disertační práce

\studijniprogram{Moderní informatika}
\studijnispec{Informační a komunikační technologie}

\abstrakt{Tato bakalářská práce pojednává o~technologiích určených k~vývoji frontendu moderních webových aplikací. 
V~práci se zabýváme analýzou a~porovnáním vybraných frontendových nástrojů, které jsou určeny k~vývoji jednostránkových aplikací. 
Praktickou částí je návrh a~implementace webových aplikací, na základě kterých srovnáme implementace ve vybraných frameworcích. 
Práce tak čtenáři nabízí nejen teoretický přehled, ale především praktické zkušenosti s~moderními technologiemi frontendu. 
Hlavní přínos spočívá v~poskytnutí uceleného pohledu na vybrané frameworky, což čtenářům usnadní výběr vhodného nástroje pro budoucí projekty. 
Práce rovněž přispívá k~hlubšímu porozumění aktuálním trendům ve vývoji frontendu.}

% \abstrakt{Text abstraktu v češtině. Rozsah by měl být 50 až 100 slov. Abstrakt není cíl práce, zde stručně popište, co čtenář má na následujících stránkách očekávat. 
% Typické formulace: \uv{V práci se zabýváme...}, \uv{Tato bakalářská práce pojednává o...}, \uv{součástí je}, \uv{je provedena analýza}, \uv{praktickou částí práce je aplikace xxx} \dots{} 
% Prostě napište stručný souhrn či charakteristiku obsahu práce.}

\abstraktanglicky{This bachelor thesis deals with technologies for frontend development of modern web applications.
In this thesis we analyze and compare selected frontend tools that are designed to develop single page applications.
The practical part is the design and implementation of web applications, based on which we will compare implementations in selected frameworks.
Thus, the work offers the reader not only a~theoretical overview, but also practical experience with modern frontend technologies.
The main contribution is to provide a~comprehensive view of the selected frameworks, which will make it easier for the reader to choose a~suitable tool for future projects. 
The thesis also contributes to a~deeper understanding of current trends in frontend development.}

% \abstraktanglicky{Anglická verze abstraktu by měla odpovídat české verzi, třebaže nemusí být úplně doslova. Když nutně potřebujete automatický překlad, použijte raději\\ 
% \iadresa{https://www.deepl.com/cs/translator},\\ je lepší než Google Translator. Není nutno překládat doslova.}

\klicovaslova{Vývoj webu, webová aplikace, frontend, framework, Angular, React, Svelte, Vue.}

\klicovaslovaanglicky{Web development, web application, frontend, framework, Angular, React, Svelte, Vue.}


% pokud máte odstavce s abstrakty příliš dlouhé a nevejde se vše na jednu stranu, pak odkomentujte následující makro:
%\longabstract


% pouze pro potřeby demonstračního dokumentu, jinak dovnitř tohoto makra doplňte vlastní podklad zadání práce a případně upravte rozměr:
\renewcommand{\kopiepodkladu}{
  \includepdf[pages={1-2}]{zadani_BP_lukas_sukenik.pdf}
}

% původní obsah vlastního zadání práce...
% \renewcommand{\kopiepodkladu}{
%   \begin{picture}(0,0)
%   \put(10,-50){\rotatebox{-35}{\scalebox{3}{\color{seda}Kopie podkladu zadání práce}}}
%   \put(10,-130){\rotatebox{-35}{\scalebox{3}{\color{seda}z IS, podepsaná}}}
%   \end{picture}}

\cestneprohlaseni{%
  Prohlašuji, že jsem tuto práci vypracoval samostatně.
  Veškerou literaturu a~další zdroje, z~nichž jsem při zpracování čerpal, v~práci řádně cituji a~jsou uvedeny v~seznamu použité literatury.}

% Doplňte jméno vedoucího, případně můžete poděkování přeformulovat.
\podekovani{%
  Rád bych poděkoval za odborné vedení, rady a~cenné poznatky k~danému tématu vedoucímu práce doc. RNDr. Lucii Ciencialové, Ph.D. 
  Také bych rád poděkoval mé rodině a~přátelům za podporu a~pomoc během mého studia.}



\begin{document}


% neměnit (vygeneruje titulní strany):
\maketitlepages

% neměnit (vygeneruje obsah):
\tableofcontents\clearpage

% neměnit (nastaví parametry odstavce a řádkování):
\nastavodstavec   %nastavi parametry odstavce
\radkovani   % navrat na hlavni nastaveni radkovani urcene prikazem \radkovanistandard{...}

% neměnit (odtud bude začínat číslování stránek, začátek stránek s kapitolami):
\mainmatter

%%%%%%%%%%%%%%%%%%%%%%
% soubory s kapitolami, kazda kapitola v samostatnem souboru, pak lze pouzivat mechanismus \includeonly{...}

\section*{Úvod}

V posledních letech došlo k výrazným změnám v oblasti vývoje webu. 
Z jednoduchých statických stránek jsme se přesunuli ke komplexním, interaktivním, jednostránkovým aplikacím (Single Page Application -- SPA). 
Webové aplikace jsou vyvíjeny s pomocí moderních technologií nebo frameworků, které vývojářům nabízí nástroje pro rychlejší a efektivnější vývoj. 
Výběr vhodných technologií je velice důležitý, protože může mít vliv na celkovou kvalitu a úspěch projektu. 
Nesprávný výběr technologie může vést ke složitým řešením, ztrátě času či finančních prostředků, nebo k problémům, které v budoucnu stíží práci vývojovým týmům.

Tato bakalářská práce se zaměřuje na analýzu a porovnání technologií určených k vývoji frontendu moderních webových aplikací. 
Cílem práce je poskytnout čtenáři ucelený pohled na vybrané SPA technologie a na základě implementace demonstrační aplikace porovnat jednotlivé frameworky.

V úvodní části práce představíme základní pojmy týkající se moderního vývoje jednostránkových aplikací. 
V rámci teoretické části uvedeme motivaci pro výběr SPA technologií, zaměříme se na analýzu vybraných frameworků a také provedeme porovnání analyzovaných nástrojů. 
V praktické části práce navrhneme webovou aplikaci, která umožní odhalit klíčové vlastnosti a rozdíly mezi vybranými technologiemi. 
Dále provedeme implementaci aplikace v rámci vybraných technologií a na závěr provedeme srovnání jednotlivých implementací.

Práce čtenáři nabídne jak teoretický přehled, tak i praktické zkušenosti s moderními technologiemi frontendu, 
což by mělo čtenářům usnadnit výběr vhodné technologie pro jejich budoucí projekty a zároveň se lépe orientovat ve světě moderního vývoje webu.

% V~Úvodu především rozvedeme cíl práce (ten najdeme v~zadání tématu práce), 
% můžeme poněkud méně formálně o~tématu povykládat (ale nepřehánějte to, žádných 10 stran úvodu prosím).
% Můžeme psát o~své motivaci, tedy proč jsme si téma zvolili, proč je považujeme za zajímavé či důležité. Můžeme také jemně uvést čtenáře do problematiky.

% Je zvykem zde psát o~tom, z~jakých částí se práce skládá. 
% Není nutné jít po kapitolách, můžeme například napsat, že práce má teoretickou a~praktickou část, 
% přičemž v~teoretické části je čtenář nejdřív seznámen s~problematikou xxx, jsou zde vysvětleny základní pojmy a~v~několika kapitolách nejběžnější metody používané pro yyy. 
% V~praktické části je popsána aplikace zzz sloužící k~qqqq, najdeme zde manuál k~jejímu používání s~postupem instalace a~zprovoznění a~také popis její vnitřní struktury a~okomentované ukázky kódu. 
% NEBO: Praktickou částí je srovnání metod sloužících k~rrrrr, přičemž po analýze metod daného typu se zdůvodněním výběru a~metodikou pro jejich porovnání jsou v~jednotlivých kapitolách popsány vybrané metody a~v~poslední kapitole najdeme jejich vzájemné srovnání. 
% NEBO: V~první kapitole rozebíráme typické požadavky na sociální sítě/informační systémy/webové aplikace/… pro účel stanovený v~zadání, 
% v~následující kapitole nastiňujeme momentální stav co se týče existujících produktů pro daný účel a~hodnotíme, do jaké míry splňují stanovené požadavky. 
% Další kapitoly obsahují návrh vlastní xxxxx s~popisem jak samotné xxxx, jejího zprovoznění, rozhraní, atd., 
% tak i~postup vytvoření (byl použit programovací jazyk zzzz), a~opět zhodnocení míry splnění stanovených požadavků.

% Tato část úvodu má čtenáře připravit na vlastní obsah práce, tak si dejte záležet, ať čtenáře neotrávíte předem :-)

% V~Úvodu dále můžeme připsat informaci o~tom, že obrázky bez uvedeného zdroje byly vytvořeny v~nástroji xxxxx, 
% případně také že u~čtenáře předpokládáme alespoň základní znalosti v~oblasti 
% \dots{} (programování, počítačových sítí, informačních systémů, sociálních sítí, tvorby webových stránek, atd. podle tématu), 
% abyste nemuseli vysvětlovat ty nejzákladnější pojmy.

\section{Webové aplikace}

Webová aplikace je typ počítačového software, který je uživateli sdílen prostřednictvím internetu. 
Uživatel se k~aplikaci dostane pomocí webového prohlížeče a~nemusí si ji instalovat na svůj počítač. 
Aplikace jsou jakýmsi prostředníkem mezi uživatelem a~serverem, kde se nachází data a~logika aplikace.\cite{codeacademywebapp}

\begin{flushleft}
  \textbf{Frontend a Backend}
\end{flushleft}

Uživatelské rozhraní, se kterým uživatel přímo interaguje, nazýváme frontend, neboli klient.
Úkolem klienta je zobrazovat vizuální stránku uživateli a~zpracovávat jeho vstupy. 
Mezi klíčové frontendové technologie patří HTML, CSS a~JavaScript.

Backend, či také server, je část aplikace starající se o zpracování dat a~logiku aplikace. 
Backend se skládá z databází a~algoritmů, které komunikují s klientem prostřednictvím API a~zpracovávají jeho požadavky. 
Serverová část bývá naprogramována v PHP, Pythonu, Ruby, JavaScriptu nebo jiném programovacím jazyce.\cite{stateofartframeworks}

\begin{flushleft}
  \textbf{Single Page Application}
\end{flushleft}

Pod pojmem Single Page Application (SPA) rozumíme webové aplikace, které se skládají z jediné stránky a~také jednotlivých částí aplikace. 
Obsah stránek bývá dynamicky aktualizován pomocí JavaScriptu. Tento přístup umožňuje aktualizaci stránek bez nutnosti obnovování celé stránky. 
V praxi to znamená, že při jakékoli akci uživatele se aktualizuje pouze obsah stránky, nikoli celá stránka. 
Mezi výhody tohoto přístupu patří rychlejší odezva aplikace, méně dotazů na server a~také lepší uživatelský zážitek.\cite{jadhavspa}

\begin{flushleft}
  \textbf{Framework}
\end{flushleft}

Framework je software, který poskytuje architekturu a~nástroje pro rychlejší a~jednodušší vývoj daných aplikací. 
Vývojář při využití frameworku nemusí řešit základní problémy, které jsou již vyřešeny v rámci frameworku. 
Použití frameworku snižuje technický dluh, zlepšuje rozšiřitelnost a~flexibilitu aplikace. 
Zlepšuje také přenositelnost, spolehlivost a~škálovatelnost aplikace.\cite{schmidtframeworks}
\section{Analýza frameworků}

Text druhé kapitoly.

\begin{citemize}
	\item odůvodnit výběr frameworků:
	\begin{citemize}
		\item \iadresa{https://survey.stackoverflow.co/2023/}
	\end{citemize}

	\item analýza frameworků:
	\begin{citemize}
		\item co to je
		\item stručná historie
		\item kdo jej vytvořil
		\item kdo jej používá
		\item jak se používá
		\item popularita ve společnosti
	\end{citemize}

	\item srování frameworků dle:
	\begin{cenumerate}
		\item Component-Based architektura (HTML, CSS, JS/TS)
		\item Šablony
		\item Předávání vlastností (props)
		\item Správa stavů
		\item Life-cycle
		\item State management
		\item Reaktivita ???
		\item DOM
		\item Routování
		\item Server-Side Rendering
		\item Kompilace zdrojových souborů
		\item Ekosystém
	\end{cenumerate}
\end{citemize}


\subsection{React}

Pod pojmem React rozumíme open-source JavaScript framework, který vyvinula a~dále vyvíjí společnost Meta (dříve Facebook). 
Podle \cite{reactbanks} jde spíše o knihovnu funkcí, než-li o komplexní nástroj pro tvorbu webových aplikací (framework). 
Tato technologie se používá pro vývoj interaktivních uživatelských rozhraní a~webových aplikací.\cite{reacthubspot}

První kořeny Reactu sahají až do roku 2010, kdy tehdejší společnost Facebook přidala novou technologii XHP do PHP. 
Jde o možnost znovu použít určitý blok kódu, stejného principu posléze využívá i React. Následně Jordan Walke vytvořil FaxJS, jenž byl prvním prototypem Reactu.
O rok později byl přejmenován na React a~začal jej využívat Facebook. 
V roce 2013 byl na konferenci JS ConfUS představen široké veřejnosti a~stal se open-source.

Od roku 2014 vývojáři představují nespočet vylepšení samotné knihovny, stejně jako spoustu rozšíření pro zlepšení vývojových procesů. 
Kolem roku 2015 postupně React nabývá na popularitě i celkové stabilitě. Následně je představen také React Native, což je framework pro vývoj nativních aplikací.
V dnešní době je React využíván společnostmi každého rozsahu po celém světě. 
Z těch největších jde například o Metu, Uber, Twitter a~Airbnb.\cite{reactbanks,reactrisingstack}

\subsubsection{Komponenty}

Hlavním stavebním kamenem Reactu jsou komponenty, jež představují nezávislé, vnořitelné a~opakovaně použitelné bloky kódu. 
Komponentu v Reactu tvoří JavaScript funkce a~HTML šablona. Validně seskládané komponenty poté tvoří webovou aplikaci.
V Reactu se můžeme setkat s funkčními a~třídními komponentami. Vytváření třídních komponent oficiální dokumentace nedoporučuje.
Pro komunikaci mezi komponentami se používá předávání vlastností (props), přes které je možné předávat hodnoty jakýchkoli datových typů.
Výstup komponent tvoří elementy ve formě JSX. Tyto elementy obsahují informace o vzhledu a~funkcionalitě dané komponenty.\cite{reactbanks,react}

\subsubsection{JSX}

Název JSX kombinuje zkratku jazyka JavaScript -- JS a~počáteční písmeno ze zkratky XML. 
Konkrétně jde o syntaktické rozšíření, které vývojářům umožňuje tvořit React elementy pomocí hypertextového značkovacího jazyku přímo v JavaScriptu. 
V rámci JSX pak je možné dynamicky vykreslovat obsah na základě logiky definované pomocí JavaScriptových hodnot.
Při kompilaci se JSX překládá do JavaScriptu pomocí nástroje Babel.\cite{reactbanks,react}

\subsubsection{Správa stavů}

Stav lze definovat jako privátní vnitřní vlastnost či proměnnou dané komponenty, jež představuje základní mechanismus pro uchovávání a~aktualizaci dat. 
Pro aktualizaci komponenty je tedy nutné stav změnit. React pak na tuto skutečnost zareaguje a~vyvolá tzv. re-render neboli překreslení komponenty s novými daty.

Za účelem ukládání stavu se využívá hook (funkce) useState. Ten poskytuje stavovou proměnnou, přes kterou se dostaneme k aktuálnímu stavu. 
Dále useState poskytuje state setter funkci, díky které můžeme stav aktualizovat. Jediný argument useState definuje počátační hodnotu daného stavu.\cite{reactitnetwork,react}

\subsubsection{Hooky}

Specifickou funkcionalitou pro React jsou tzv. hooky, které byly do Reactu přidány až ve verzi 16.8.0.\cite{reactgithub} 
Hook je definován jako funkce, která obohacuje komponenty pomocí předdefinovaných funkcionalit. Jedním z nejpoužívanejších hooků je useState. 
Vývojáři mohou používat již zabudované hooky, nebo si vytvářet své vlastní s pomocí předdefinovaných hooků. 
Mezi zabudované hooky patří např. useEffect, useMemo, useCallback, useRef, useContext.\cite{react}

\subsubsection{Životní cyklus}

Životní cyklus komponenty je sekvence událostí, jež nastanou mezi vytvořením a zničením komponenty. 
Ve třídních komponentách existovaly speciální metody, tzv. lifecycle metody, starající se o provedení určité části kódu při daném okamžiku v životě komponenty. 
Nyní React disponuje pár hooky, které umožňují provádět side-effects podobně jako lifecycle metody.

O momentu, kdy je komponenta přidána na obrazovku, mluvíme jako o namontování (mount) komponenty. Při změně stavu či obdržení nových parametrů hovoříme o aktualizaci (update) komponenty. 
A v neposlední řadě okamžik, kdy je komponenta odstraněna z obrazovky, nazýváme odmontování (unmount) komponenty.\cite{reactlifecycle, react}

\subsubsection{Routování}
\subsubsection{Ekosystém}


% https://www.simplilearn.com/tutorials/reactjs-tutorial/what-is-reactjs

\subsection{Svelte}

\subsection{Vue}

\subsection{Porovnání}

\begin{citemize}
	\item co zjistím na první pohled, platforma,
	\item jako bych si četl reklamu ...,
	\item prvotní srovnání.
\end{citemize}

\section{Testování frameworků}

\begin{citemize}
	\item proč a co je obsahem kapitoly?
\end{citemize}

\subsection{Analýza a návrh testových úloh}

\begin{citemize}
	\item co a proč porovnávám,
	\item v návrhu - jak, jaké testové úlohy?
	\item (dokumentace - možná nahoře, syntax, výkonnostní testy, velikosti bundlů, účel aplikace, rychlost, srozumitelnost, ...)
\end{citemize}

\subsection{Demonstrační aplikace}

V této kapitole srovnáme implementaci stejných funkcionalit ve třech vybraných frameworcích.

\subsubsection{Angular}

\begin{flushleft}
  \textbf{Správa stavů, předávání vlastností}
\end{flushleft}

Pro implementaci jednoduchého counteru nejprve vytvoříme counter komponentu. Můžeme začít se strukturou HTML značek pro hlavní komponentu.

\begin{prog}
// Soubor counter.component.html

<div class="bg-gray-200 p-6 rounded-md shadow-md">
  <p class="text-xl font-semibold mb-4">Current count: \{\{ count \}\}</p>

  <div class="flex gap-4">
    <counter-button
      [className]="'bg-blue-500 text-white hover:bg-blue-600'"
      (buttonClicked)="increment()"
    >
      Increment
    </counter-button>

    <!-- další tlačítka... -->
  </div>
</div>
\end{prog}

Jelikož potřebujeme opakovaně použít logiku jednotlivých tlačítek, vytvoříme komponentu counter-button. 
Ta může přijímat například CSS styly nebo přes output (\emph{EventEmitter}) posílat informaci o~kliknutí na tlačítko směrem nahoru ve stromě komponent.

\begin{prog}
// Soubor counter-button.component.ts

import \{CommonModule\} from '@angular/common';
import \{Component, EventEmitter, Input, Output\} from '@angular/core';

// Nastavení komponenty.
@Component(\{
  selector: 'counter-button',
  standalone: true,
  templateUrl: './counter-button.component.html',
  imports: [CommonModule],
\})
export class CounterButtonComponent \{
  // Vstupní vlastnost komponenty.
  @Input() public className = '';

  // Výstupní vlastnost komponenty.
  @Output() public buttonClicked = new EventEmitter<void>();
\}
\end{prog}

Funkci \emph{emit()} \emph{EventEmitteru} zavoláme na tlačítku v~counter-buttonu právě tehdy, když uživatel klikne na tlačítko -- použijeme listener ve formě \emph{(click)}. 
K~propsání textu či jiných elementů nebo komponent mezi párovými tagy <counter-button> pak poslouží párový či nepárový element <ng-content />.

\begin{prog}
// Soubor counter-button.component.html

<button
  class="px-4 py-2 rounded-md focus:outline-none"
  [ngClass]="className"
  (click)="buttonClicked.emit()"
>
  <!-- ng-content slouží k vykreslení obsahu, který vložíme
    mezi párové tagy (selectory) dané komponenty. -->
  <ng-content></ng-content>
</button>
\end{prog}

Následně v~counter komponentě importujeme třídu CounterButtonComponent a~do všech elementů counter-button předáme jejich vstupy a~výstupy. 
Námi definovanovanému outputu \emph{buttonClicked} předáme v~šabloně metodu, která se vykoná po emitu (kliknutí na tlačítko ve vnořené komponentě) a~metodu zavoláme pomocí kulatých závorek. 
V~rámci counter komponenty pak definujeme stav jako vlastnost \emph{count} na třídě. Vlastnost pak můžeme modifikovat skrze metody třídy, které voláme v~outputu \emph{buttonClicked}.

\begin{prog}
// Soubor counter.component.ts

import \{CommonModule\} from '@angular/common';
import \{Component\} from '@angular/core';
import \{CounterButtonComponent\} from './button/counter-button.component';

@Component(\{
  selector: 'counter',
  standalone: true,
  templateUrl: './counter.component.html',
  imports: [CommonModule, CounterButtonComponent],
\})
export class CounterComponent \{
  protected count = 0;

  protected increment(): void \{
    this.count++;
  \}

  protected decrement(): void \{
    this.count--;
  \}

  protected reset(): void \{
    this.count = 0;
  \}
\}
\end{prog}

\begin{flushleft}
  \textbf{Interakce v uživatelském prostředí}
\end{flushleft}

Při vytváření jakékoli UI komponenty můžeme začít šablonou, nebo definovat funkční stránku. My začneme s~tvorbou šablony. V~případě vlastního dropdown samotným tlačítkem a~seznamem možností. 
Otevření možností zajístíme tak, že na tlačítko přidáme click listener. Funkčnost pak zajistíme díky modifikaci stavu \emph{isOpen}, který se provede při volání metody \emph{toggleDropdown}. 
V~rámci této metody je třeba zavolat i~\emph{event.stopPropagation()}. Předejdeme tak potenciální chybě ve formě tzv. event bubblingu -- spuštění událostí na prvcích odlišných od cílového.

\begin{prog}
// Část souboru dropdown.component.html

<div class="rounded-md shadow-sm">
  <!-- Pro poslouchání na události v DOMu můžeme 
    použít syntaxi: (NÁZEV_UDÁLOSTI)="OBSLUŽNÁ_METODA". -->
  <button
    type="button"
    class="" <!-- Statické styly... -->
    [ngClass]="buttonStyles + ' ' + sizeStyles"
    (click)="toggleDropdown(\$event)"
  >
    \{\{ selectedOption ? selectedOption.label : placeholder \}\}
    <!-- Pro podmíněné vykreslovaní můžeme využít bloky @if, @else if, @else. -->
    @if (isOpen) \{
      <arrow-up-icon />
    \} @else \{
      <arrow-down-icon />
    \}
  </button>
</div>
\end{prog}

Podmíněně zobrazíme seznam možností, které získáme v~jednom z~inputů. K~vykreslení všech možností použijeme blok \emph{@for}. 
Pro vybraní konkrétní možnosti využijeme \emph{(click)} a~do obslužné metody předáme aktuální prvek v~poli -- option. 
Metoda \emph{handleOptionClick} pak zajistí uložení aktuálně vybrané možnosti, zavření dropdownu a~vyemitování vybrané možnosti do rodičovské komponenty.

\begin{prog}
// Část souboru dropdown.component.html

@if (isOpen) {
  <div
    class="" <!-- Statické styly... -->
    [ngClass]="divStyles"
  >
    <div class="py-1" role="menu"> <!-- WAI-ARIA atributy... -->
      <!-- Pro vykreslení listu (pole hodnot) můžeme využít blok @for. -->
      @for (option of options; track option.value) {
        <button
          class="block w-full text-left px-4 py-2 text-sm hover:text-gray-900"
          [ngClass]="optionStyles"
          role="menuitem"
          (click)="handleOptionClick(option)"
        >
          {{ option.label }}
        </button>
      }
    </div>
  </div>
}
\end{prog}

V~případě, že máme dropdown otevřen a~chceme jej po kliknutí mimo tentýž dropdown bezpečně zavřít, nehledě na počet vykreslených dropdown komponent na stránce, budeme postupovat následovně. 
Pro každou komponentu vytvoříme unikátní vlastnost ve formě ID. To pak dynamicky umístíme na kořenový element dropdownu.

\begin{prog}
// Část souboru dropdown.component.ts

protected dropdownId = `id-\${crypto.randomUUID()}`;

// Část souboru dropdown.component.html

<div class="relative inline-block text-left" [id]="dropdownId">
\end{prog}

V~komponentě pak budeme naslouchat na události v~DOM pomocí dekorátoru \emph{@HostListener}. 
Dekorátor přijímá DOM událost, na který má poslouchat -- \emph{document:pointerdown}, případně další argumenty nebo také formu vypublikované události. 
Pod dekorátorem pak definujeme obslužnou metodu, která se volá při emitu specifikované události. V~rámci metody pak zajistíme uzavření aktuálně otevřeného dropdownu.

\begin{prog}
// Část souboru dropdown.component.ts

@HostListener('document:pointerdown', ['\$event.target'])
onClickOutsideDropdown(target: HTMLElement): void \{
  if (this.isOpen && !target.closest(`#\$\{this.dropdownId\}`)) \{
    this.isOpen = false;
  \}
\}
\end{prog}

Dropdown pak může mít různé inputy, které povedou k~lepší znovupoužitelnosti. Hodnotu inputu (konkrétně např.~\emph{defaultValue}) v~komponentě získáme v~metodě životního cyklu \emph{OnInit}. 
Styly ve formě JS hodnot do šablony přidáme pomocí \emph{ngClass}. Když těchto hodnot potřebujeme na elementu více, zřetězíme předávané hodnoty pomocí JavaScriptu. 
Další možnost spočívá ve sloučení požadovaných stylů na úrovni třídy.

\begin{flushleft}
  \textbf{Reaktivita, asynchronní operace}
\end{flushleft}

Pro ukázku reaktivity a~asynchronních operací můžeme vytvořit komponentu, která bude překládat zadaný text do vybraného jazyka. 
Začneme tedy vytvořením rodičovské komponenty, která při změně vlastností (zadaného textu uživatelem a~výstupního jazyka) zavolá API, které vrátí přeložený text. 
V~rámci této komponenty vytvoříme vnořené komponenty, které budou sloužit k~zadání vstupního textu, výběru jazyka a~zobrazení výsledku. 

LanguageDropdownComponent umožní uživateli vybrat jazyk, do kterého chce přeložit text. 
Přes \emph{EventEmitter} aktualizujeme výstupní jazyk v~rodičovské komponentě. V~rámci obslužné metody \emph{handleLanguageChange} pak také modifikujeme hodnotu vlastnosti \emph{inputValuesChanges\$}.
Tato vlastnost je \emph{Subject}, speciální typ observable, z~knihovny RxJS. Později dovolí na základě změny hodnoty poslat dotaz na server ve správný moment. 
Podobným způsobem poté můžeme registrovat událost změny vstupního textu -- naslouchat na změnu vstupního textu.
% TODO: Přidat schéma

\begin{prog}
// Část souboru translator.component.ts

protected handleLanguageChange(outputLanguage: Option): void \{
  this.outputLanguage = outputLanguage.value;
  // Synchronní aktualizace hodnoty observable (v tomto případě Subjectu).
  // Slouží pro následné operace při změně hodnoty observable.
  this.inputValuesChanges\$.next(outputLanguage.value);
\}
\end{prog}

Zadání vstupního textu pak může řešit komponenta TranslationInputComponent, která obdobným způsobem aktualizuje hodnotu vstupního textu v~rodičovské komponentě. 
Aktuální hodnotu formulářového prvku nastavíme pomocí \emph{[ngModel]}. Pro naslouchání na změnu hodnoty formulářového prvku zase využijeme \emph{(ngModelChange)}.

\begin{prog}
// Část souboru translation-input.component.html

<textarea
  autosizeTextArea
  class="block w-full min-h-0 p-3 pr-12 pb-8 resize-none !outline-none"
  placeholder="Type to translate ..."
  [ngModel]="inputText"
  (ngModelChange)="handleInputChange(\$event)"
>
</textarea>
\end{prog}

V~případě, že potřebujeme aktualizovat výšku textového pole na základě jeho obsahu, můžeme využít vlastní direktivu AutosizeTextAreaDirective. 
V~konstruktoru direktivy získáme element, na který přidáme tuto direktivu. Dále budeme potřebovat třídu \emph{Renderer2}, která umožňuje manipulovat s~DOM. 
V~direktivě budeme naslouchat na změnu hodnoty textového pole pomocí dekorátoru \emph{@HostListener} a~události input. Následně v~rámci obslužné metody zajistíme aktualizaci výšky.

Změny hodnoty vlastnosti \emph{inputValuesChanges\$} začneme odebírat pomocí \emph{subscribe}. 
Abychom předešli dotazování serveru ihned po změně hodnoty vlastnosti \emph{inputValuesChanges\$}, použijeme operátor \emph{debounceTime}. 
Ten povolí poslat dotaz na server až po uplynutí určité doby od poslední změny, kterou můžeme nastavit. 
Subscribe zavolá veřejnou metodu služby (\emph{getTranslation}), která vrací přeložený text. 
Nakonec, aby dotazování serveru fungovalo, je třeba metodu \emph{setupInputChangeSubscription} zavolat v~konstruktoru nebo hooku \emph{OnInit}.

\begin{prog}
// Část souboru translator.component.ts

private setupInputChangeSubscription(): void \{
  // Naslouchá změnám vstupního textu a výstupního jazyka.
  // Operátor debounceTime zajistí, že se změna vstupního textu 
    nebo výstupního jazyka vyhodnotí až po uplynutí 300 ms.
  // Dále operátor distinctUntilChanged zajišťuje, 
    že se změna vyhodnotí pouze v případě, kdy je odlišná od předchozí hodnoty.
  // Operátor takeUntil() zajišťuje, 
    že se subscription zruší při zničení komponenty.
  // Pokud se změní vstupní text nebo výstupní jazyk, 
    v rámci methody subscribe se spustí překlad.
  this.inputValuesChanges\$
    .pipe(debounceTime(300), distinctUntilChanged(), takeUntil(this.destroy\$))
    .subscribe(() => this.triggerTranslation());
\}
\end{prog}

V~rámci služby TranslationService použijeme třídu \emph{HttpClient} z~Angular modulů, která umožňuje odesílat HTTP požadavky na server.
Službu \emph{HttpClient} získáme v~konstruktoru, kde ji pomocí klíčového slova \emph{private} přiřadíme do vlastností třídy. 
Pokračujeme implementací metody \emph{getTranslation}, v~níž zavoláme metodu \emph{post} na HTTP klientovi s~patřičným nastavením. 
Tímto způsobem budeme dotazovat Microsoft Translator Text API \cite{translatortextapi}, díky kterému v~odpovědi obdržíme přeložený text. 
Pokud úspěšná odpověď ze serveru obsahuje složitější strukturu, ze které potřebujeme získat jen nějakou část, pak s~konverzí odpovědi pomůže RxJS operátor \emph{map()}. 
Metoda \emph{getTranslation} vrací observable, v~translator komponentě proto hodnoty odebíráme pomocí metody \emph{subscribe}. 
Subscription bychom také vždy měli zrušit, abychom předešli možným únikům paměti či chybám.

\begin{prog}
// Část souboru translation.service.ts

return this.httpClient
  .post<TranslationResponseData>(url, body, options)
  .pipe(map(data => this.convertToOutputText(data)));

// Část souboru translator.component.ts

// Slouží ke zrušení subscriptions při zničení komponenty.
private destroy\$: Subject<void> = new Subject();
// Slouží k naslouchání na změny vstupního textu a výstupního jazyka.
private inputValuesChanges\$ = new Subject<string>();

public ngOnDestroy(): void \{
  // Slouží k manuálnímu unsubscribe všech observables při zničení komponenty.
  this.destroy\$.next();
  this.destroy\$.complete();
\}

this.translationService
  .getTranslation(this.inputText, this.outputLanguage)
  .pipe(
    // Zajišťuje, že se subscription zruší při zničení komponenty.
    takeUntil(this.destroy\$),
    // Zachytí chybu v observable.
    catchError(error => this.handleError(error)),
  )
  // V metodě subscribe dostaneme transformovanou odpověď 
    (v rámci next callbacku) nebo chybu (v rámci error callbacku).
  // Po poslední úspěšné aktualizaci observable se volá callback funkce complete.
  .subscribe(\{
    next: response => (this.outputText = response),
    error: error => (this.error = error),
    complete: () => (this.loading = false),
  \});
\end{prog}

V~momentě, kdy obdržíme odpověď ze serveru, zobrazíme přeložený text uživateli. 
K~tomu poslouží TranslationOutputComponent, které na vstupu předáme výstupní text spolu s~dalšími vstupními vlastnostmi. 
V~rámci šablony pak podmíněně vykreslíme přeložený text, chybu nebo načítání. 

Při zarovnání vstupního a~výstupního pole v~UI si musíme dát pozor na to, že šířku je potřeba nastavit již v~prvním potomku div elementu, na kterém nastavíme flexbox. 
Důvod spočívá v~tom, že Angular v~DOM vytváří element pro každou komponentu.

\begin{prog}
// Část souboru translation.service.ts

<div class="flex text-xl">
  <translation-input 
    <!-- Vstupní a výstupní vlastnosti... -->
    class="relative w-1/2"
    <!-- Šířka musí být nastavena zde. -->
  />

  <translation-output 
    <!-- Vstupní a výstupní vlastnosti... -->
    class="relative w-1/2"
    <!-- Šířka musí být nastavena zde. -->
  />
</div>
\end{prog}

\begin{flushleft}
  \textbf{Tvorba formulářů, validace}
\end{flushleft}

Angular je flexibilní z~pohledu možností tvorby formulářů. My použijeme reaktivní formuláře, jelikož jsou flexibilnější a~umožní nám jednodušší reakce na změny prvků.
Vytvoříme komponentu zaměřenou na jednoduché investiční kalkulace. 
Bude obsahovat dvě vnořené komponenty: formulář pro zadání vstupních dat a~komponentu výsledku kalkulace, která se zobrazí po potvrzení formuláře.

Začneme s~tvorbou reaktivního formuláře. Typ \emph{InvestForm} popisuje strukturu souvisejících formulářových prvků formuláře.

\begin{prog}
// Část souboru invest-form.component.ts

type InvestForm = FormGroup<\{
  oneOffInvestment: FormControl<number | null>;
  investmentLength: FormControl<number | null>;
  averageSavingsInterest: FormControl<number | null>;
  averageSP500Interest: FormControl<number | null>;
\}>;
\end{prog}

Protože prvků budeme mít více, deklarujeme formulářovou skupinu jako vlastnost třídy, ve které následně definujeme samotné formulářové prvky. 
Vlastnost \emph{investForm} pak umožní přístup k~hodnotám formuláře a~jeho validaci. Zde narazíme na problém s~nenastavením počáteční hodnoty vlastnosti přímo nebo v~konstruktoru. 
Můžeme ho vyřešit za pomoci vykřičníku -- řekneme tak TypeScriptu, že obsah proměnné je nenulový. Další možností je vypnout pravidlo \emph{strictPropertyInitialization} v~souboru \emph{tsconfig.json}.

\begin{prog}
// Část souboru invest-form.component.ts

protected investForm!: InvestForm;
\end{prog}

Hodnotu vlastnosti \emph{investForm} nastavíme pomocí metody \emph{initializeInvestForm} v~rámci \emph{OnInit} hooku. 
Tento postup zvolíme, protože chceme nastavovat počáteční hodnoty formuláře na základě vstupní vlastnosti \emph{defaultValues}.
Důvodem je, že hodnoty vstupních vlastností jsou v~komponentě dostupné nejdříve v~rámci hooku \emph{OnInit}.

Metoda \emph{initializeInvestForm} vrátí instanci třídy \emph{FormGroup}, kterou vytvoříme pomocí třídy \emph{FormBuilder} ze základního balíčku \emph{@angular/forms}. 
Argumentem pro metodu \emph{group} pak je objekt, který popisuje strukturu formuláře. 
Vlastnosti objektu budou klíče formulářových prvků a~jejich hodnoty pole, kde první prvek bude počáteční hodnota a~druhý prvek pole validátorů.

\begin{prog}
// Část souboru invest-form.component.ts

private initializeInvestForm(): InvestForm \{
  // Vytvoření formuláře s výchozími hodnotami 
    (případně vlastnostmi) a validátory.
  // Jednotlivé prvky FormGroup bývají označovány jako FormControl.
  return this.fb.group(\{
    oneOffInvestment: [
      this.defaultValues.oneOffInvestment,
      [Validators.required, Validators.min(20), Validators.max(99_999_999)],
    ],
    // Další formulářové prvky... 
  \});
\}
\end{prog}

V~šabloně následně propojíme formulářovou skupinu s~formulářem. K~tomu poslouží direktiva \emph{[formGroup]} a~její hodnotu nastavíme na vlastnost \emph{investForm}. 
V~rámci formuláře pak vytvoříme formulářové prvky, které propojíme direktivou \emph{formControlName}. Hodnota pak musí odpovídat klíči prvku ve formulářové skupině. 
Pro zajištění efektivní obsluhy chyb formuláře můžeme využít getter metody, které vrátí konkrétní formulářový prvek.

\begin{prog}
// Část souboru invest-form.component.html

<form [formGroup]="investForm" (ngSubmit)="onSubmit()">
  <div class="md:flex md:gap-4">
    <div class="mb-4 md:w-1/2">
      <input-label id="oneOffInvestment">
        One-off investment (20-99.999.999€)
      </input-label>

      <!-- Direktiva formControlName slouží k propojení inputu 
        s odpovídajícím FormControl v FormGroup. -->
      <input
        id="oneOffInvestment"
        type="number"
        formControlName="oneOffInvestment"
        class="" <!-- Statické styly... -->
      />

      @if (oneOffInvestmentControl.errors?.['required']) \{
        <p class="text-red-500 text-xs italic mt-1">
          Please enter a valid amount of one-off investment (positive number).
        </p>
      \}
      <!-- Další chybové hlášky... -->
    </div>

    <!-- Další formulářové prvky... -->
  </div>
</form>
\end{prog}

Dále vytvoříme tlačítko s~typem submit, přes které uživatel formulář potvrdí. Na form značku přidáme \emph{(ngSubmit)}, který vyemituje událost při potvrzení formuláře. 
Obslužná metoda pak prostřednictvím výstupové vlastnosti publikuje aktuální hodnotu reaktivního formuláře do rodičovské komponenty.

V~rámci rodičovské komponenty tedy vykreslíme samotný formulář a~při jakémkoli potvrzení formuláře získáme aktuální hodnoty z~formuláře díky outputu. 
Hodnoty formuláře pak dostaneme v~obslužné metodě \emph{handleFormChanged}. Pomocí služby FutureValuesCalculatorService tyto hodnoty transformujeme do požadovaného formátu. 
Výsledek uložíme do vlastnosti \emph{futureValues}.

Když jsou hodnoty vypočteny, vykreslíme je na stránce prostřednictvím komponent future-values-info a~future-value-info. 
První z~komponent slouží k~rozložení výsledků do požadovaného formátu a~vytvoření komponent pro jednotlivé výsledky. 
Komponenta future-value-info pak přijímá vstupní vlastnost, kterou v~šabloně před vykreslením v~DOM přetransformujeme díky rouře (\emph{LocalizedNumberPipe}).

\begin{prog}
// Část souboru future-value-info.component.html

<p class="text-5xl font-bold">{{ futureValue | localizedNumber }}</p>
\end{prog}

Stejného výsledku bychom mohli dosáhnout i~přes metodu na třídě. Tento přístup Angular nedoporučuje, jelikož metody se v~rámci šablony spouští opakovaně a~mohou způsobit problémy s~výkonem. 
Oproti tomu roura umožní lepší znovupoužitelnost a~přehlednost.

\begin{prog}
// Soubor localized-number.pipe.ts

import {Pipe, PipeTransform} from '@angular/core';

// Roura, která převede číslo na formátovaný string s měnou (€).
@Pipe(\{name: 'localizedNumber', standalone: true\})
export class LocalizedNumberPipe implements PipeTransform \{
  public transform(value: number): string \{
    return `\$\{value.toLocaleString('de-DE')\}€`;
  \}
\}
\end{prog}

\begin{flushleft}
  \textbf{Modularita, použití knihoven}
\end{flushleft}

V~této sekci vytvoříme webovou hru, kde cílem uživatele bude uhádnout název státu na základě poskytnutých nápověd. Práci si ulehčíme pomocí externích knihoven a~služeb.
Ve hře se postupně bude odkrývat 8 nápověd, které by měly pomoct s~uhádnutím daného státu. 
Klíčovým prvkem je textové pole, přes které uživatel zadává názvy hádaných zemí a~tlačítko pro potvrzení. 
Součástí je také seznam již zadaných hádaných zemí a~modální okna sloužící k~vyhodnocení hry.

Začneme s~implementací rodičovské komponenty, která bude získávat data o~všech zemích světa z~REST Countries API \cite{restcountriesapi}. 
Další zodpovědností této komponenty bude vykreslování odpovídajících stavů při získávání dat -- stav načítání, úspěšné získání dat a~chyba při získávání dat. 
Vytvoříme službu CountryService, díky které získáme data o~zemích. Pro tyto účely vytvoříme metodu \emph{getAllCountries}, která vrátí observable pole všech zemí. 
Výsledek registrace služby a~přímé zavolání metody \emph{getAllCountries} uložíme do vlastnosti třídy.

\begin{prog}
// Soubor country-guesser-wrapper.component.ts

protected countries\$: Observable<Countries> 
  = inject(CountryService).getAllCountries();
\end{prog}

V~šabloně posléze potřebujeme odebírat hodnotu z~observable. Práci v~šabloně výrazně ulehčí knihovna \emph{ngx-load-with} \cite{ngxloadwith}. 
Tato knihovna poskytuje integrovanou podporu načítání a~zpracování chyb. To programátorovi umožní využívat předdefinované šablony pro dané stavy bez nutnosti další implementace. 
Navíc se programátor nemusí starat o~zrušení odběru observable.

\begin{prog}
// Část souboru country-guesser-wrapper.component.html

<ng-container
  *ngxLoadWith="countries\$ as countries; 
  loadingTemplate: loading; errorTemplate: error"
>
  <country-guesser [countries]="countries" />
</ng-container>

<!-- \#loading je reference na načítací šablonu. -->
<ng-template \#loading>
  <!-- Vlastní načítací šablona... -->
</ng-template>

<!-- \#error je reference na chybovou šablonu. -->
<!-- let-error umožňuje přístup k chybě. -->
<ng-template \#error let-error>
  <!-- Vlastní chybová šablona... -->
</ng-template>
\end{prog}

V~rámci komponenty country-guesser budeme implementovat jednotlivé herní prvky, komponenta také bude vyhodnocovat průběh hry. 
Definujeme tedy vlastnosti třídy, které budou reprezentovat stav a~průběh hry. V~hooku \emph{OnInit} získáme náhodnou zemi (zemi pro uhádnutí). 
Uvnitř zavoláme veřejnou metodu \emph{usePolyfill} na službě CountryFlagPolyfillService, která zajistí zobrazení ikon vlajek v~prohlížečích, které nepodporují zobrazení vlajek.
Do komponenty také přidáme obslužné metody \emph{handleEvaluateGuessAndUpdateState} a~\emph{handleSetInitialState}, ve kterých implementujeme logiku hry. 
V~šabloně následně vykreslíme UI komponenty hry a~podmíněně modální okna při výhře či prohře.

V~metodě \emph{usePolyfill} služby CountryFlagPolyfillService zavoláme funkci \emph{polyfillCountryFlagEmojis} z~knihovny \emph{country-flag-emoji-polyfill} \cite{countryflagemojipolyfill}. 
Pokud prohlížeč uživatele nepodporuje zobrazení ikon vlajek, ale podporuje emojis a~webové fonty, skrze funkci \emph{polyfillCountryFlagEmojis} knihovna přidá webový font do HTML hlavičky. 
Font Twemoji Country Flags pak umožní zobrazení vlajek. Aby se programaticky přidaný font použil, nesmíme zapomenout nastavit font-family pravidlo v~rámci CSS stylů.

\begin{prog}
// Část souboru styles.css

@layer base \{
  html \{
    font-family: 'Twemoji Country Flags', 'ALTERNATIVNÍ_FONTY...';
  \}
\}
\end{prog}

HintBoxesComponent postupně vykreslí nápovědy. Při jakékoli změně vstupních vlastností vytvoříme pole nápověd pomocí vlastnosti randomCountry. 
V~šabloně iterujeme přes pole nápověd a~vykreslíme jednotlivé nápovědy. Vlastnost hintEnabled nastavíme pomocí indexu a~vstupní vlastnosti hintsEnabledCount. 
Samotný hint-box pak dynamicky vykreslí název a~SVG ikonu nápovědy, textovou nápovědu, případně obrázek vlajky státu.

Pokračujeme implementací komponenty country-guess-input, která uživateli umožní zadat svůj tip. 
Začneme šablonou, kde vytvoříme formulářový prvek pro zadání názvu země a~potvrzovací tlačítko. 
Dále podmenu textového pole, které zobrazí nejpodobnější země na základě zadaného textu -- filtrované země a~chybové hlášky. 
Můžeme také rovnou přidat obslužné metody pro akce a~události nad formulářem, které následně postupně doimplementujeme.

V~souboru \emph{country-guess-input.component.ts}, tedy v~rámci třídy CountryGuessInputComponent, při změně vstupních vlastností (v~hooku \emph{OnChanges}) aktualizujeme vlastnost \emph{countriesWithoutAlreadyGuessed} a~\emph{filteredCountries}. 
V~případě první vlastnosti jde o~pole všech zemí bez těch, které uživatel již hádal. Druhá vlastnost poté představuje pole počátečních 8 prvků vlastnosti \emph{countriesWithoutAlreadyGuessed}. 
Metoda \emph{handleGuessButtonClick} zavolá obslužnou metodu rodičovské komponenty, která vyhodnotí tip a~aktualizuje stav hry. 
Aktualizujeme také hodnoty aktuálního tipu, filtrovaných zemí a~uzavřeme podmenu, k~čemuž slouží metoda \emph{handleChangeSelectedGuess} volaná i~napřímo z~šablony. 
Tělo metody \emph{handleInputChange} převede uživatelům tip do správného formátu a~pomocí převedené hodnoty aktualizuje aktuální tip spolu s~filtrovanými zeměmi.
Metoda \emph{handleKeyDown} se postará o~interakce s~podmenu pomocí klávesnice. Skrze šipky nahoru a~dolů povolíme uživateli vybrat hádanou zemi. 
Enter umožní změnu aktuálního tipu názvu země na právě tu, kterou uživatel označil v~podmenu. Escape poslouží k~uzavření podmenu.

Uvnitř pomocné metody \emph{updateGuessAndFilteredCountries} následně modifikujeme vlastnost \emph{currentGuess}. 
Dle metody \emph{getFilteredCountries} získáme aktuálně filtrované země na základě uživatelova tipu. 
Rovněž přenastavíme vlastnost \emph{isValidGuess}, která určuje, zda je uživatelův tip validní (taková země existuje). 
V~neposlední řadě aktualizujeme vlastnost \emph{selectedGuessIndex}, jež určuje, která země je vybraná v~podmenu. 
K~tomu slouží metoda \emph{clampSelectedGuessIndex}, která index udrží v~požadovaném rozmezí (0 až počet filtrovaných zemí).
V~metodě \emph{getFilteredCountries} získáme filtrované země na základě vlastnosti \emph{currentGuess}. 
Metoda \emph{changeSelectedGuessIndex} aktualizuje vlastnost \emph{selectedGuessIndex} o~hodnotu předanou v~argumentu. 
K~převodu tipu uživatele slouží pomocná metoda \emph{convertToFormattedGuess}. Metoda zajistí, aby tip začínal velkým písmenem a~zbytek řetezce byl složen z~malých písmen.

Implementujeme komponentu guessed-countries-list, jenž zobrazí seznam již hádaných zemí. 
Mezi vstupními vlastnostmi bude pole všech zemí (\emph{countries}), pole hádaných zemí uživatelem (\emph{guessedCountries}) a~také země, kterou uživatel musí uhodnout (\emph{randomCountry}).
Pomocí vstupních vlastností a~služby EnrichGuessedCountriesService získáme pole hádaných zemí s~jejich vlajkou a~vzdáleností od \emph{randomCountry} (\emph{distanceFromRandomCountry}).
Služba EnrichGuessedCountriesService ke každé hádané zemi přidá vlajku a~vypočte \emph{distanceFromRandomCountry}. 
Pro vypočtení vzdálenosti použijeme funkci \emph{getDistanceBetweenTwoPoints} a~vlastnost \emph{latlng}, kterou získáme ze serveru při získávání všech zemí. 
Funkci \emph{getDistanceBetweenTwoPoints} importujeme z~knihovny \emph{calculate-distance-between-coordinates} \cite{distancebetweencoordinates}. 
Hodnotu vlastnosti \emph{enrichedGuessedCountries} aktualizujeme v~rámci hooku \emph{OnChanges}. Seznam hádaných zemí vykreslíme v~šabloně.

Pokračujeme implementací modálních oken, které se zobrazí při výhře či prohře. 
Vlastnosti \emph{isWinModalOpen} a~\emph{isLoseModalOpen}, určující, zda se mají okna zobrazit, bychom měli aktualizovat v metodě \emph{handleEvaluateGuessAndUpdateState} v~CountryGuesserComponent. 
Oběma modálním oknům předáme vlastnost \emph{randomCountry} a~output \emph{handleClose} v~podobě obslužné události, která se vyvolá při zavření modálního okna. 
Do výherního modálu rovněž poskytneme vstupní vlastnost \emph{totalGuessesNeeded}, jíž využijeme v~obsahu okna. 
Obě modální okna budou velice podobné, a~proto vytvoříme komponentu base-modal, která bude sloužit jako šablona pro obě okna. 
BaseModalComponent bude přijímat titulek, obsah modálního okna a~\emph{handleClose} jako výstupní vlastnost. 
Šablona base-modal pak vykreslí základní strukturu modálního okna, s~dynamicky nastaveným titulkem, obsahem a~obslužnou metodou volanou při zavření modálního okna.

\begin{flushleft}
  \textbf{Routování a layout aplikace}
\end{flushleft}

Demonstrační aplikace bude složena z~hlavičky, patičky a~samotného obsahu, v~němž se vykreslí jednotlivé komponenty. 
Mezi jednotlivými stránkami se uživatel bude moci přepínat pomocí navigačního menu.

K~routování mezi jednotlivými stránkami využijeme modul Router přímo od Angularu. Nejprve vytvoříme cesty (\emph{routes}) pro jednotlivé stránky v~souboru \emph{app.routes.ts}. 
Proměnnou \emph{routes} vytvoříme dle předpisu \emph{Routes} a~následně exportujeme. Cesty pak poskytneme routeru v~rámci \emph{app.config.ts}.

\begin{prog}
// Část souboru app.routes.ts

import \{Routes\} from '@angular/router';

export const routes: Routes = [
  \{
    title: 'Home',
    path: '',
    component: LandingComponent,
    pathMatch: 'full',
  \},
  \{
    title: 'Counter',
    path: 'counter',
    component: CounterComponent,
  \},
  // Další cesty...
  \{path: '**', component: PageNotFoundComponent\},
];

// Část souboru app.config.ts

export const appConfig: ApplicationConfig = \{
  // V tomto nastavení poskytujeme služby a poskytovatele pro celou aplikaci.
  providers: [
    provideRouter(routes),
    // Další poskytované služby...
  ],
\};
\end{prog}

Pokračujeme vytvořením požadované struktury stránek v~AppComponent. Šablona bude obsahovat hlavičku, patičku a~obsah, který vykreslíme pomocí elementu \emph{router-outlet}. 

\begin{prog}
// Soubor app.component.html

<div class="min-h-screen flex flex-col">
  <app-header />

  <main class="flex-grow p-8">
    <!-- Router-outlet vykresluje šablonu (komponentu) pro aktuální URL adresu. -->
    <router-outlet></router-outlet>
  </main>

  <app-footer />
</div>
\end{prog}

V~rámci komponenty hlavičky pak vytvoříme navigační menu, které bude obsahovat odkazy na jednotlivé stránky. 
Můžeme se inspirovat například architekturou a~vzhledem navigačního menu Flowbite. 
Pokračujeme vypsáním všech cest aplikace, k~čemuž využijeme direktivy \emph{routerLink}, \emph{routerLinkActive}, \emph{routerLinkActiveOptions} a~referenci \emph{\#link}. 
Do \emph{routerLink} předáme patřičnou cestu a~\emph{routerLinkActive} umožní naslouchat na aktuální URL. Direktiva \emph{routerLinkActiveOptions} pak přepíše výchozí nastavení \emph{routerLinkActive}.
Díky referenci \emph{\#link} získáme informaci o~tom, zda je odkaz aktivní. To využijeme při podmíněném nastavení správných CSS tříd na aktivních a~neaktivních odkazech.

\begin{prog}
// Část souboru header.component.html

@for (route of appRoutes; track route.title) \{
  <li>
    <a
      [routerLink]="route.path"
      routerLinkActive
      [routerLinkActiveOptions]="routerLinkActiveOptions"
      #link="routerLinkActive"
      class="block py-2 pr-4 pl-3 lg:p-0"
      [ngClass]="\{
        'STATICKÉ STYLY PRO AKTIVNÍ ODKAZ...': link.isActive,
        'STATICKÉ STYLY PRO NEAKTIVNÍ ODKAZ...': !link.isActive
      \}"
      ariaCurrentWhenActive="page"
    >
      \{\{ route.title \}\}
    </a>
  </li>
\}
\end{prog}

Přepínání barevného režimu, otevírání a~zavírání mobilní navigace implementujeme pomocí obslužných metod a~vlastností třídy. 
Informaci o~tom, zda má uživatel zapnutý tmavý režim ukládáme do LocalStorage v~prohlížeči. 
Při kliknutí na tlačítko pro přepnutí režimu zavoláme metodu \emph{toggleDarkMode}, která změní hodnotu vlastnosti a~uloží ji do LocalStorage.

\begin{prog}
// Část souboru header.component.ts

protected toggleDarkMode(): void \{
  this.isDarkMode = !this.isDarkMode;
  this.updateDarkMode();
\}

private updateDarkMode(): void \{
  if (this.isDarkMode) \{
    document.documentElement.setAttribute('data-mode', 'dark');
    localStorage.setItem('data-mode', 'dark');
  \} else \{
    document.documentElement.removeAttribute('data-mode');
    localStorage.removeItem('data-mode');
  \}
\}
\end{prog}
\subsubsection{React}

\begin{flushleft}
  \textbf{Instalace projektu}
\end{flushleft}

\begin{flushleft}
  \textbf{Správa stavů, předávání vlastností}
\end{flushleft}

Při implementaci jednoduchého čítače začneme tím, že vytvoříme Counter komponentu. Ta bude mít stav count a setter setCount pro tento stav.

Dále vytvoříme komponentu Button kvůli principu DRY a celkově znovupoužitelnosti kódu. 
Typ ButtonProps obsahuje vlastnosti, které můžeme tlačítku předat -- className, onClick a children. 
Díky tomu, že typ rozšiřuje ButtonHTMLAttributes<HTMLButtonElement>, můžeme předat do komponenty i další běžné atributy HTML tlačítek (např. type, value, disabled).

V rámci argumentu Button komponenty použijeme ES6 destructuring assignment pro získání vlastností. 
Z objektu vlastností získáme className a children, ostatní vlastnosti ponecháme zabalené v proměnné props pomocí spread operátoru. 
Nyní můžeme vytvořit JSX pro samotné tlačítko. Vlastnost className přidáme do tříd tlačítka. 
Pomocí children můžeme do tlačítka vložit libovolný obsah, který bude mezi párovými značkami <Button>. 
Všechny ostatní vlastnosti pomocí spread operátoru předáme přímo tlačítku.

V Counter komponentě v rámci JSX vrátíme hodnotu stavu count a vykreslíme Button komponenty, jimž předáme potřebné vlastnosti. 
Pro aktualizaci stavu využijeme vlastnost onClick, které předáme anonymní funkci (arrow function) a v ní zavoláme setCount.

\begin{flushleft}
  \textbf{Interakce v uživatelském prostředí}
\end{flushleft}

Pro vytvoření jakékoliv UI komponenty můžeme začít tvořit jak JSX, definici komponenty, nebo znovupoužitelný hook. 
V tomto případě, při vývoji komponenty rozevíracího seznamu, začneme naprogramováním vlastního hooku, který se odděleně postará o veškerou logiku seznamu.

Hook useDropdown bude mít 2 parametry -- výchozí hodnotu vybrané možnosti (defaultValue) a obslužnou funkci ke změně vybrané v možnosti v rodičovské komponentě (onChange). 
V rámci hooku nadefinujeme stavy selectedOption, isOpen a vygenerujeme unikátní identifikátor. 
Dále vytvoříme funkci handleOptionClick, která zajistí změnu vybrané možnosti, zavření seznamu a vypublikuje změnu hodnoty do rodičovské komponenty. 
Z hooku vracíme potřebné stavy a funkce ve formě objektu nebo pole -- pole musíme označit jako const.

Pokračujeme tvorbou JSX komponenty Dropdown, kde vložíme tlačítko a seznam možností. Otevření možností zajistíme přidáním onClick (což je vlastně MouseEventHandler). 
V anonymní funkci pak změníme stav pomocí isOpen na opačnou hodnotu. Abychom předešli event bubblingu, v rámci anynomní obslužné funkce zavoláme event.stopPropagation().

Seznam možností zobrazíme podmíněně na základě stavu isOpen. Pro vykreslení možností seznamu (options) použijeme JavaScriptovou funkci map uvnitř JavaScriptové hodnoty v JSX. 
V Reactu je důležité vždy při použití funkce map nastavit unikátní klíč (key) pro každou položku v seznamu. Tento klíč slouží k identifikaci jednotlivých prvků a optimalizaci procesu renderování. 
Pro vybrání konkrétní možnosti použijeme onClick, kterému předáme anonymní funkci. V anonymní funkci zavoláme funkci handleOptionClick hooku useDropdown s aktuální položkou ze seznamu.

Abychom uzavřeli jakýkoli aktuálně otevřený rozbalovací seznam na stránce, po kliknutí mimo tento seznam, předáme kořenovému elementu dříve vytvořený unikátní identifikátor. 
Do useDropdown přidáme useEffect a díky němu budeme naslouchat na události pointerdown v DOM. Obslužná funkce pak zajistí zavření aktuálně otevřeného dropdownu.

Dropdown samozřejmě může mít i jiné vstupy, které povedou k lepší znovupoužitelnosti. 
Dynamické CSS třídy ve formě JavaScriptu na element přidáme pomocí šablonových literálů (template literals) a JavaScriptové hodnoty.

\begin{flushleft}
  \textbf{Reaktivita, asynchronní operace}
\end{flushleft}

Následující komponenta bude demonstrovat využití reaktivity a asynchronních operací. Vytvoříme komponentu, která přeloží zadaný text do cílového jazyka. 
Začneme vytvořením komponenty Translator. Komponenta při změně stavů (zadaného textu uživatelem a výstupního jazyka) zavolá API, které vrátí přeložený text.
V rámci komponenty vytvoříme vnořené komponenty pro zadání vstupního textu, výběr jazyka a zobrazení výsledku.

Komponenta LanguageDropdown uživateli umožní vybrat jazyk, do kterého chce text přeložit. 
Díky vlastnosti onChange (callback funkci) aktualizujeme výstupní jazyk v rodičovské komponentě. 

Pokračujeme implementací komponenty TranslationInput, která bude sloužit k zadání vstupního textu přes textové pole. Aktuální hodnotu formulářového prvku nastavíme pomocí atributu value.
Po změně hodnoty textového pole, kterou získáme v události přes atribut onChange, aktualizujeme hodnotu vstupního textu v Translator komponentě. 
Abychom reaktivně aktualizovali výšku pole na základě obsahu, použijeme vlastní hook. 
Hook bude potřebovat referenci elementu, a tak vytvoříme ref, který přidáme na element textového pole.

Hook useAutosizeTextArea bude příjímat referenci na element. Dále také hodnotu textového pole, aby po jakékoli změně této hodnoty přepočítala výška pole. 
V rámci hooku vytvoříme useEffect, který se znovu zavolá při každé změně textAreaRef, nebo hodnoty textu. Následně v rámci těla hooku aktualizujeme výšku textového pole.

V Translator komponentě potřebujeme ukládat vstupní hodnotu a výstupní jazyk z vnořených komponent. Dále při každé změně těchto hodnot zavoláme API, k čemuž využijeme useEffect.
V rámci hooku definujeme asynchronní funkci handleTranslation, která pomocí fetch API odešle korektní HTTP POST požadavek na server. 
Pokud bychom definovali funkci mimo useEffect, museli bychom ji přidat do pole závislostí hooku.
Při úspěšné odpovědi aktualizujeme stav s přeloženým textem, v opačném případě nastavíme chybový stav.

Aby dotazování fungovalo, vytvoříme referenci delayTimerRef. V rámci těla useEffect hooku nejprve zrušíme předchozí časovač. 
Funkci handleTranslation zavoláme v callbacku funkce setTimeout, která umožní předejít dotazování serveru ihned po změně nějaké vstupní hodnoty. 
Výsledek funkce setTimeout uložíme do delayTimerRef.current. Nesmíme také zapomenout na zrušení časovače při zničení komponenty.

V okamžiku, kdy obdržíme odpověď ze serveru, zobrazíme přeložený text uživateli pomocí komponenty TranslationOutput. 
Předáme jí samotný výstupní text a další vstupní vlastnosti, na základě kterých pak podmíněně vykreslíme přeložený text, chybu nebo načítání.

\begin{flushleft}
  \textbf{Tvorba formulářů, validace}
\end{flushleft}

React sám o sobě poskytuje jen základní API pro správu formulářů. Disponuje však mnoha knihovnami, které tuto funkcionalitu rozšiřují. 
Mezi takové knihovny patří např. Formik, Redux Form nebo React Hook Form. V této sekci se zaměříme na tvorbu formulářů pomocí React Hook Form. 
Vytvoříme komponentu pro jednoduchou investiční kalkulaci. 
V rámci této komponenty naprogramujeme formulář pro zadání vstupních dat a komponentu výsledku kalkulace, která se zobrazí po potvrzení formuláře.

Začneme s reaktivním fomulářem, který bude přijímat počáteční hodnoty (defaultValues) a callback funkci handleFormSubmit pro předání hodnot formuláře do rodičovské komponenty. 
Strukturu formuláře popíšeme v typu InvestFormData. Pomocí hooku useForm z knihovny React Hook Form vytvoříme instanci formuláře, které předáme defaultValues a nastavíme reaktivní validaci. 
Následně z hooku dostameme funkce register, handleSubmit a formState, které poslouží ke správě formuláře.

Následně do JSX přidáme form s onSubmit atributem, kterému předáme funkci handleSubmit z React Hook Form. Do handleSubmit pak vložíme vstup handleFormSubmit v rámci nějž získáme aktuální hodnoty formuláře. 
Ve formuláři vytvoříme formulářové prvky, které propojíme s reaktivním formulářem pomocí funkce register. První argument představuje název formulářového prvku, druhý argument je validační objekt. 
V rámci range inputu potřebujeme HTML atributy min a max, díky kterým omezíme rozsah vstupních hodnot. Abychom mohli měli přístup k aktuální hodnotě range inputu, využijeme vlastnost value a onChange. 
Chyby formuláře získáme z formState a vykreslíme je pod formulářovými prvky. V neposlední řadě přidáme tlačítko s typem submit, které zajistí odeslání formuláře a zavolání callback funkce handleFormSubmit.

V rodičovské komponentě získáme aktuální hodnoty formuláře díky obslužné funkci handleFormSubmit. Pomocí funkce futureValuesCalculator získáme hodnoty, které následně vykreslíme v komponentě FutureValuesInfo. 
Tato komponenta obsahuje dvě vnořené komponenty FutureValueInfo, pro zobrazení jednotlivých výsledků. Hodnotu v JSX transformujeme pomocí JavaScriptové funkce.

\begin{flushleft}
  \textbf{Modularita, použití knihoven}
\end{flushleft}

V následující sekci vytvoříme webovou hru, ve které cílem uživatele je uhádnout název státu na základě poskytnutých nápovědí. Práci si ulehčíme pomocí externích knihoven a služeb.
Postupně se bude odkrývat 8 nápovědí, které by měly pomoci s uhádnutím daného státu. Klíčovým prvkem je textové pole, přes které uživatel zadává názvy hádaných zemí a tlačítko pro potvrzení. 
Součástí také bude seznam zemí, které uživatel hádal a modální okna sloužící k vyhodnocení hry.

Začneme s implementací rodičovské komponenty, která získá země z veřejného API. Naprogramujeme hook useAllCountries, který bude vracet data (countries), chybu a stav načítání. 
V rámci useEffect zavoláme asynchronní funkci fetchCountriesData. Uvnitř funkce fetchCountriesData zavoláme getAllCountries. 
Jde o převzatý requestHandler s využítím knihovny axios, jenž umožní otypování příchozí odpovědi. Po ošetření chyb aktualizujeme patřičné stavy, které následně z hooku vrátíme. 
Implementaci načítacích a chybových stavů nebo rušení asynchronních dotazů nám do značné míry může ulehčit knihovna react-query.

Následně v rámci rodičovské komponenty podmíněně vykreslíme dané komponenty. V případě chyby komponentu ErrorAlert. Pokud ze serveru úspěšně dostaneme země, tak vykreslíme komponentu CountryGuesser. 
Pokud nevykreslíme ani jednu z předchozích komponent, zobrazíme LoadingSkeleton.

Komponenta CountryGuesser bude vyhodnocovat průběh hry a zobrazovat jednotlivé herní prvky. Začneme definicí stavů a inicializujeme náhodnou zemi (randomCountry), kterou bude uživatel hádat. 
Dále použijeme hook useCountryFlagPolyfill, který při namontování komponenty zajistí podporu zobrazení ikon vlajek v prohlížečích, které to přímo nepodporují. 
Prohlížeč pak však musí podporovat emojis a webové fonty. Pokračujeme implementací obslužných metod handleEvaluateGuessAndUpdateState a handleSetInitialState, které budou sloužit k aktualizaci stavu hry. 
V rámci JSX pak vykreslíme jednotlivé herní prvky a modální okna při výhře či prohře.

Hook useCountryFlagPolyfill zavolá funkci polyfillCountryFlagEmojis, která do HTML hlavičky přidá webový font Twemoji Country Flags. Aby se font využil, přidáme jej do CSS stylů.

Úkolem komponenty HintBoxes bude postupné vykreslení nápověd. Na základě vstupu randomCountry vytvoříme pole nápověd. V JSX pak iterujeme přes pole nápověd a vykreslíme jednotlivé nápovědy. 
Jednotlivé komponenty HintBox pak dynamicky vykreslí název a SVG ikonu nápovědy, textovou nápovědu, případně obrázek vlajky státu.

Klíčová komponenta CountryGuessInput pak uživateli umožní zadat svůj tip. 
Začneme s JSX, kde vytvoříme formulářový prvek pro zadání názvu země, potvrzovací tlačítko a podmenu textového pole, které zobrazí nejpodobnější země na základě zadaného textu (filtrované země). 
Přidáme obslužné metody pro akce a události nad formulářem, které následně doimplementujeme.

Ve funkční komponentě CountryGuessInputComponent, na základě vstupu countries, získáme pole všech zemí bez těch, které uživatel již hádal (countriesWithoutAlreadyGuessed). 
Poté definujeme a inicializujeme ostatní stavy komponenty. 
Při kliknutí na tlačíko se zavolá funkce handleGuessButtonClick, která volá obslužnou funkci handleEvaluateGuessAndUpdateState v rodičovské komponentě a také funkci handleChangeSelectedGuess. 
Funkce handleChangeSelectedGuess aktualizuje aktuální tip, filtrované země a uzavře podmenu. Funkce handleInputChange převede tip uživatele do daného formátu, poté aktualizuje aktuální tip a filtrované země. 
Ovládání formulářového prvku pomocí klávesnice umožní funkce handleKeyDown.

Pomocná funkce updateGuessAndFilteredCountries získá aktuálně filtrované země na základě uživatelova tipu. Následně aktualizuje stavy currentGuess, isValidGuess a filteredCountries. 
Funkce clampSelectedGuessIndex zajistí, aby index uživatelem vybrané země byl v požadovaném rozmezí (0 až počet filtrovaných zemí). 
Pro aktualizaci vlastnosti selectedGuessIndex slouží funkce changeSelectedGuessIndex, která index aktualizuje o hodnotu předanou v argumentu.
V neposlední řadě funkce convertToFormattedGuess převede tip uživatele tak, aby začínal velkým písmenem a zbytek řetezce byl složen z malých písmen.

Ke zobrazení všech již hádaných zemí uživatelem vytvoříme komponentu GuessedCountriesList. Ze vstupních vlastností countries, guessedCountries a randomCountry získáme proměnnou enrichedGuessedCountries. 
Jde o uživatelem hádané země s vlajkou a vzdáleností od randomCountry. K převodu využijeme JavaScriptové funkce z jiného souboru. 
K vypočtení vzdálenosti použijeme knihovnu calculate-distance-between-coordinates, která obsahuje funkci getDistanceBetweenTwoPoints. Jednotlivé prvky pole enrichedGuessedCountries pak vykreslíme v rámci JSX.

Nakonec vytvoříme modální okna, která se zobrazí při výhře či prohře. Stavy isWinModalOpen a isLoseModalOpen aktualizujeme v rámci funkce handleEvaluateGuessAndUpdateState v CountryGuesser. 
Na základě těchto stavů pak podmíněně vykreslíme daná modální okna. Oběma modálům předáme randomCountry a obslužnou funkci handleClose. Výhernímu modálu také počet potřebných pokusů. 
V jednotlivých komponentách (WinModal, LoseModal) vykreslíme komponentu BaseModal, která bude sloužit jako šablona pro obě okna. Do této komponenty vždy předáme titulek, obsah a obslužnou metodu handleClose. 
BaseModal následně v JSX vykreslí základní strukturu modálního okna, s dynamickými možnostmi pro titulek, obsah a obslužnou metodu handleClose.

\begin{flushleft}
  \textbf{Layout aplikace, routování}
\end{flushleft}

Layout aplikace bude rozdělen do tří částí: hlavičky, patičky a samotného obsahu, v nemž se vykreslí jednotlivé komponenty. Uživatel bude mít možnost přepínání mezi jednotlivými stránkami pomocí navigačního menu.

Pro routování využijeme knihovnu react-router-dom. Začneme vytvořením souboru s cestami (routes). Následně v kořeni aplikace vytvoříme router pomocí předem definovaných cest aplikace. 
Router vytvoříme díky dvěma pomocným funkcím k tomu určených: createBrowserRouter a createRoutesFromElements. Pokračujeme přiřazením routeru do kořenové komponenty aplikace, konkrétně do poskytovatele routeru.

Hlavní komponenta AppLayout pak v JSX vykreslí hlavičku, patičku a dynamický obsah dle aktuální URL adresy, jenž vykreslí komponenta Outlet.

V hlavičce aplikace se budou nacházet odkazy na jednotlivé stránky. My se inspirujeme architekturou a vzhledem navigačního menu Flowbite. 
V rámci komponenty Header vypíšeme všechny cesty aplikace pomocí komponenty NavLink z knihovny react-router-dom. NavLink umožňuje v rámci atributu className přistoupit k vlastnosti isActive, která indikuje, zda je cesta odkazu aktivní. 
Vlastnosti isActive tedy využijeme pro podmíněné přidání CSS stylů. Pro korektní nastavení aria-current atributu, v závislosti na aktuální URL, použijeme hook useLocation, který vrací aktuální URL.

Mobilní navigaci a barevný režim implementujeme díky stavům isMobileNavOpen a isDarkMode. Informaci o tom, zda má uživatel zapnutý tmavý režim budeme ukládat do LocalStorage v prohlížeči. 
K tomu použijeme useEffect hook, který při změně stavu isDarkMode přidá patřičný data-mode a provede aktualizaci LocalStorage.
\subsubsection{Svelte}

\begin{flushleft}
  \textbf{Instalace projektu}
\end{flushleft}

\begin{flushleft}
  \textbf{Správa stavů, předávání vlastností}
\end{flushleft}

Prvním krokem k vytvoření jednoduchého čítače bude definice komponenty Counter s reaktivním stavem count.

Dále vytvoříme komponentu Button z důvodu dodržování principu DRY a efektivnějšímu znovupoužití kódu v budoucnu.
Komponenta bude přijímat vlastnosti className a onClick. ClassName rozšíří CSS třídy tlačítka a onClick bude obsahovat obslužnou funkci, která se zavolá při kliknutí na tlačítko.
Nyní do šablony přidáme tlačítko a předáme mu vlastnosti className a onClick.
Svelte umožňuje zachytit všechny nedefinované vlastnosti do proměnné \$\$restProps. Proměnnou \$\$restProps tedy pomocí spread operátoru předáme tlačítku a tím jej obohatíme o další vlastnosti. 
Obsah tlačítka, který definujeme mezi párovými značkami Button, vykreslíme pomocí komponenty slot.

V Counter komponentě pak v rámci šablony vykreslíme stav count a Button komponenty, kterým předáme příslušné vlastnosti. 
Pro aktualizaci stavu count použijeme obslužné funkce, v nichž přímo modifikujeme count.

\begin{flushleft}
  \textbf{Interakce v uživatelském prostředí}
\end{flushleft}

V této sekci implementujeme rozbalovací seznam s možnostmi (dropdown). Tvorbu UI komponenty můžeme začít jak vytvořením HTML struktury, tak definicí funkční stránky komponenty. 
My začneme tvorbou šablony, v níž vytvoříme tlačítko a seznam možností. Otevření seznamu možností zajistíme přidáním on:click události na tlačítko. V obslužné funkci pak změníme stav isOpen. 
Seznam možností zobrazíme podmíněně na základě isOpen. Pro vykreslení možností seznamu (dle vstupu options) použijeme blok \#each. 
Pro vybrání konkrétní možnosti použijeme on:click událost, při které v anonymní funkci zavoláme funkci handleOptionClick s aktuální položkou ze seznamu.

Dropdown komponenta bude přijímat vlastnosti options a onChange, případně další vlastnosti pro znovupoužitelnost. Pro každou komponentu také vytvoříme jednoznačný identifikátor, který využijeme při uzavírání seznamu.
Obslužná funkce handleOptionClick zajistí změnu vybrané možnosti, zavření seznamu a provede změnu hodnoty v rodičovské komponentě. 

K uzavření jakéhokoli otevřeného seznamu, při kliknutí mimo tento seznam, předáme kořenovému elementu komponenty dříve vytvořený unikátní identifikátor.
V rámci akce (Svelte action) clickOutsideDropdown budeme naslouchat na události pointerdown v DOM. Obslužná funkce pak zajistí spuštění callbacku v Dropdown komponentě. 
Samotné akci tedy předáme obslužnou funkci handleClickOutsideDropdown, která zavře aktuálně otevřený dropdown.

Třídy CSS v JavaScriptové formě přidáme k elementu pomocí šablonových literálů a JavaScriptové hodnoty.

\begin{flushleft}
  \textbf{Reaktivita, asynchronní operace}
\end{flushleft}

V rámci následující sekce se zaměříme na reaktivitu a asynchronní operace. Naprogramujeme komponentu, která přeloží zadaný text do cílového jazyka. 
Začneme vytvořením komponenty Translator. Komponenta reaktivně (při změně zadaného textu či výstupního jazyka) zavolá API, které vrátí přeložený text. 
V Translator komponentě využijeme vnořené komponenty, které budou sloužit k zadání vstupního textu, výběru jazyka a zobrazení výsledku.

Skrze komponentu LanguageDropdown umožníme uživateli vybrat jazyk, do kterého bude chtít text přeložit. Výstupní jazyk v rodičovské komponentě změníme přes vlastnost onChange.

Pokračujeme implementací komponenty TranslationInput, která umožní zadat vstupní text (inputText) přes textové pole. Aktuální hodnotu formulářového prvku nastavíme pomocí bind:value. 
V rámci rodičovské komponenty použijeme bind, díky čemuž pak reaktivně aktualizujeme inputText v Translator komponentě. K reaktivní změně výšky textového pole použijeme akci autoresizeTextArea. 
Akce přijme element, na kterém se má provést změna výšky. Elementu přidáme listener na událost input. V obslužné funkci následně modifikujeme výšku pole.

V rámci rodičovské komponenty zavoláme API v momentě, kdy dojde ke změně vstupního textu nebo výstupního jazyka. K tomu použijeme reaktivní prohlášení (reactive statement). 
V těle prohlášení zrušíme předchozí časovač a pomocí funkce setTimeout zavoláme funkci handleTranslation. Tímto způsobem předejdeme dotazování serveru ihned po změně nějaké vstupní hodnoty. 
Při zničení komponenty zrušíme časovač a asynchronní požadavky.

Účelem asynchronní funkce handleTranslation pak je odeslání korektního HTTP POST požadavku na server pomocí fetch API. 
Při úspěšné odpovědi aktualizujeme stav s přeloženým textem, v opačném případě nastavíme chybový stav.

Při obdržení odpovědí ze serveru vykreslíme přeložený text uživateli pomocí komponenty TranslationOutput. 
Komponentě předáme výstupní text spolu s dalšími vlastnostmi, na základě kterých v šabloně podmíněně vykreslíme přeložený text, chybu nebo načítání.

\begin{flushleft}
  \textbf{Tvorba formulářů, validace}
\end{flushleft}

Svelte, stejně jako React, nepodporuje pokročilou správu formulářů. Můžeme však využít knihovny třetích stran, jako např. svelte-forms-lib nebo Superforms, které nám umožní lépe spravovat a validovat formuláře. 
V rámci této sekce vytvoříme komponentu pro jednoduchou investiční kalkulaci s využitím knihovny svelte-forms-lib. 
Komponenta InvestForm bude obsahovat formulář pro zadání vstupních hodnot a komponentu FutureValuesInfo pro zobrazení výsledků kalkulace.

Začneme implementací reaktivního formuláře, který bude přijímat počáteční hodnoty (defaultValues) a investFormData sloužící k předání hodnot formuláře do rodičovské komponenty. 
Strukturu formuláře popíšeme v typu InvestFormData. Pomocí funkce createForm z knihovny svelte-forms-lib vytvoříme instanci formuláře, které předáme defaultValues do vlastnosti initialValues. 
V rámci nastavení formuláře také definujeme validační schéma pomocí knihovny yup a onSubmit obslužnou funkci. Knihovnu yup volíme, jelikož je jedinou kompatibilní možností s knihovnou svelte-forms-lib ve verzi 2.0.1. 
% vlastnost validationSchema musí být typu ObjectSchema<any>, který je definován jen a pouze v knihovně yup:
% import type {ObjectSchema} from 'yup';
% validationSchema?: ObjectSchema<any>;
Aby výstupní data (investFormData) odpovídala InvestFormData, musíme transformovat hodnoty formuláře pomocí funkce cast na validačním schématu. V opačném případě budou hodnoty typu string. 
Z createForm následně získáme obslužné funkce handleChange a handleSubmit, dále stavy formuláře form, errors a isValid. Ke stavům formuláře přistupujeme pomocí \$, protože jde o stores observables.

Do šablony přidáme form s on:submit událostí, které předáme handleSubmit. Pokračujeme vytvořením formulářových prvků. 
Jedlotlivé prvky propojíme s reaktivním formulářem pomocí funkce bind:value a události on:change, do které přiřadíme handleChange. Chyby formuláře získáme z errors a vykreslíme je pod formulářovými prvky. 
V neposlední řadě přidáme tlačítko s typem submit, které se postará o odeslání formuláře a zavolání obslužné funkce onSubmit.

Pokračujeme tím, že v rodičovské komponentě pomocí bind získáme aktuální hodnoty formuláře (investFormData). Po změně hodnot formuláře hodnoty transformujeme pomocí funkce futureValuesCalculator. 
Výsledek (futureValues) pak zobrazíme v komponentě FutureValuesInfo. Jedlotlivé výsledky bude zobrazeny v rámci vnořených komponent FutureValueInfo. 
K modifikaci vstupní hodnoty v komponentě FutureValueInfo použijeme reactive statement.

\begin{flushleft}
  \textbf{Modularita, použití knihoven}
\end{flushleft}

Nyní vytvoříme webovou hru, ve které bude úkolem uživatele uhádnout název státu na základě poskytnutých nápovědí. Práci si zlehčíme využitím externích knihoven. 
V rámci hry se postupně odkryje 8 nápovědí, které uživateli pomohou uhádnout název daného státu. Mezi klíčovými prvky bude textové pole pro zadání názvu země a potvrzovací tlačítko. 
Dále ve hře bude seznam zemí, které uživatel hádal a modální okna pro vyhodnocení hry.

Začneme s programováním rodičovské komponenty, jejíž úkolem bude získat země z veřejného API. K tomu využijeme balíček @tanstack/svelte-query a axios. 
Vytvoříme funkci useAllCountries, která vrátí výsledek HTTP dotazu (CreateQueryResult) pomocí funkce createQuery. Argumentem funkce createQuery bude objekt s názvem dotazu (queryKey) a funkce, která vykoná dotaz (queryFn). 
Dotaz na serveru provede asynchronní funkce getAllCountries, v níž využijeme převzatou asynchronní funkci requestHandler a knihovnu axios. Po získání odpovědi ošetříme chyby a vrátíme výsledek.

V rámci rodičovské komponenty dostaneme výsledek dotazu a uložíme jej do proměnné countries. Následně pomocí vlastností isError a data podmíněně vykreslíme jednotlivé komponenty. 
V případě chyby zobrazíme komponentu ErrorAlert. Když úspěšně získáme pole zemí, vykreslíme komponentu CountryGuesser. LoadingSkeleton zobrazíme, pokud se nezobrazí žádná z předchozích komponent.

Komponenta CountryGuesser zobrazí jednotlivé herní prvky a bude vyhodnocovat průběh hry. Začneme definicí stavů a náhodně vybereme náhodnou zemi (randomCountry), kterou uživatel bude hádat. 
V hooku onMount zavoláme funkci polyfillCountryFlagEmojis. Při namontování komponenty tak zajistíme zobrazení ikon vlajek v prohlížečích, které to přímo nepodporují. Prohlížeč uživatele však musí podporovat emojis a webové fonty. 
Funkce polyfillCountryFlagEmojis přidá do hlavičky stránky webový font Twemoji Country Flags. Aby se font použil, přidáme jej do CSS stylů. 
Pokračujeme implementací obslužných funkcí handleEvaluateGuessAndUpdateState, handleSetInitialState, které budou sloužit k aktualizaci stavu hry. V rámci šablony zobrazíme jednotlivé herní prvky a modální okna při výhře či prohře.

V rámci komponenty HintBoxes postupně zobrazíme nápovědy. Podle vstupu randomCountry budeme reaktivně vytvářet pole nápověd, jelikož randomCountry se může změnit. 
V šabloně posléze vykreslíme napovědy pomocí HintBox komponent. HintBox dynamicky vykreslí název a SVG ikonu nápovědy, textovou nápovědu, případně obrázek vlajky státu.

Komponenta CountryGuessInput umožní uživateli zadání názvu země (uživatelova tipu). Začneme šablonou, kde vytvoříme formulářový prvek pro zadání tipu a potvrzovací tlačítko. 
Dále také podmenu textového pole, které zobrazí nejpodobnější země na základě zadaného textu (filtrované země). Přidáme obslužné funkce pro akce a události nad formulářem, které následně doimplementujeme.

V rámci skriptové části získáme, na základě vstupu countries, pole všech zemí bez těch, které uživatel již hádal (countriesWithoutAlreadyGuessed). Dále definujeme a inicializujeme ostatní stavy komponenty. 
Po kliknutí na potvrzovací tlačítko zavoláme funkci handleGuessButtonClick. V tělě funkce zavoláme obslužnou funkci evaluateGuessAndUpdateState, pomocí níž vyhodnotíme stav hry v rodičovské komponentě. 
Následně také funkci handleChangeSelectedGuess, která aktualizuje aktuální tip, filtrované země a uzavře podmenu. Funkce handleInputChange převede tip uživatele do daného formátu, aktualizuje aktuální tip a filtrované země. 
Ovládání textového pole pomocí klávesnice umožní funkce handleKeyDown.

V pomocné funkci updateGuessAndFilteredCountries nejprve získáme filtrované země podle uživatelova tipu. Následně aktualizujeme stavy currentGuess, isValidGuess a filteredCountries. 
Funkce clampSelectedGuessIndex zajistí, aby index vybrané země byl v požadovaném rozmezí (0 až počet filtrovaných zemí). 
K modifikaci stavu selectedGuessIndex použijeme funkci changeSelectedGuessIndex, která index aktualizuje o hodnotu předanou v argumentu. 
Tip uživatele v rámci funkce convertToFormattedGuess převedeme tak, aby začínal velkým písmenem a zbytek řetezce byl složen z malých písmen.

Pro zobrazení všech již hádaných zemí uživatelem vytvoříme komponentu GuessedCountriesList. Ze vstupních vlastností countries, guessedCountries a randomCountry získáme proměnnou enrichedGuessedCountries. 
Jde o uživatelem hádané země s vlajkou a vzdáleností od randomCountry. K převodu využijeme JavaScriptovou funkci z jiného souboru. 
Vzdálenost zemí vypočteme pomocí knihovny calculate-distance-between-coordinates, která exportuje funkci getDistanceBetweenTwoPoints. Proměnnou enrichedGuessedCountries následně vykreslíme v rámci šablony.

Nakonec vytvoříme modální okna, které vykreslíme při výhře nebo prohře. Stavy isWinModalOpen a isLoseModalOpen aktualizujeme v rámci funkce handleEvaluateGuessAndUpdateState v CountryGuesser. 
Na základě těchto stavů podmíněně zobrazíme daná modální okna. Oběma oknům předáme randomCountry a obslužnou funkci handleClose. Do výhernímu modálu také počet potřebných pokusů. 
V jednotlivých komponentách (WinModal, LoseModal) vykreslíme komponentu BaseModal, která bude sloužit jako šablona pro obě okna. Do této komponenty pak předáme titulek, obsah modálu a obslužnou metodu handleClose. 
V šabloně BaseModal vykreslíme základní strukturu modálního okna s dynamickými vlastnostmi.

\begin{flushleft}
  \textbf{Layout aplikace, routování}
\end{flushleft}

Aplikaci rozdělíme do tří částí: hlavičky, patičky a samotného obsahu, v němž vykreslíme jednotlivé komponenty. Uživatel se bude moci přepínat mezi jednotlivými stránkami přes navigační menu. 

Pro routování v aplikaci využijeme knihovnu svelte-spa-router. Nejprve vytvoříme seznam cest aplikace (appRoutes). 
Následně v hlavní komponentě transformujeme appRoutes do požadovaného formátu (typu RouteDefinition) a výsledek uložíme do proměnné routes. 
V šabloně zobrazíme hlavičku, patičku a Router, kterému předáme proměnnou routes. Router následně vykreslí šablonu na základě aktuální URL adresy.

Hlavička zobrazí odkazy na jednotlivé stránky. Architekturu a vzhled navigačního menu převezmeme např. od Flowbite. 
V rámci komponenty Header vypíšeme cesty aplikace pomocí HTML elementu a, na nějž přidáme atribut href. Dále přidáme také akce link a active, které poskytuje svelte-spa-router. 
Akci active předáme objekt, kde přes vlastnosti className a inactiveClassName nastavíme požadovanou CSS třídu podle toho, zda je odkaz aktivní nebo neaktivní. 
K nastavení aria-current použijeme location objekt ze svelte-spa-router, díky kterému získáme aktuální URL.

Stavy isMobileNavOpen a isDarkMode umožní ovládat zobrazení mobilní navigace a barevného režimu. 
Pro uložení informace o preferenci tmavého režimu uživatele použijeme LocalStorage v prohlížeči. Logiku pro přepnutí barevného režimu zavoláme v rámci hooku beforeUpdate. 
Tento hook se spustí po změně lokálního stavu, ale před aktualizací HTML.

\subsection{Testování aplikací a výsledky}

\begin{citemize}
	\item výsledky a průběh z 3.1
\end{citemize}


\section{Ukázková kapitola}

\subsection{Obrázky a tabulky}


Obrázky a~tabulky mají být uzavřeny v příslušném prostředí (pro obrázky je to \uv{figure}, pro tabulky \uv{table}, vždy s titulkem, viz níže). Toto prostředí zajistí správné zarovnání a umístění objektu. Není nutné, aby objekt byl přesně na místě, kde se o něm píše v textu, lze použít odkaz, třeba na obrázek \ref{fig:slecnasnotebookem} na straně \pageref{fig:slecnasnotebookem}.

Popisky sázíme pod obrázky a~nad tabulky. Popisek nebastlete ručně, ale využijte prostředky \LaTeX u -- důvodem je, aby bylo možné automaticky vygenerovat seznam obrázků a~seznam tabulek.

\begin{figure}[htb]
	\centering
		\includegraphics[width=.5\textwidth]{slecna-s-notebookem.png}
	\caption[Ukázka vložení titulku s~označením zdroje]{Ukázka vložení titulku s~označením zdroje\cite{mytyprog}}
	\label{fig:slecnasnotebookem}
\end{figure}


Všimněte si ve zdroji souboru (v~tomto případě soubor zakladni-info.tex), jak je zajištěno, aby označení zdroje sice bylo u~obrázku, ale neobjevilo se v~seznamu obrázků -- makro \ttsmall{\zpetnelomitko{}caption} má povinný parametr (to, co se objeví u~obrázku) i~nepovinný parametr (to, co se objeví v~seznamu obrázků).

V~tabulkách použijte raději řádkování trochu užší než 1,5, třeba 1.1 nebo 1.2.

Pokud jste přejali tabulku nebo to, co se v~ní nachází, také je nutné uvést zdroj, podobně jako u~obrázku.

Pokud obrázky a~tabulky nepoužíváte (nebo máte jednu či dvě), vzadu odstraňte seznam obrázků/tabulek.


\begin{table}[htb]
	\centering
	\caption{Ukázka tabulky}
	\medskip
	\radkovani[1.2]
		\begin{tabular}{@{}l||l|c|c@{}}\hline
			\textbf{Číslo} & \textbf{Jméno} & \textbf{Věk} & \textbf{Očkování}\\ \hline\hline
			1	&Žeryk&	4	&ano\\ \hline
			2	&Andy&	7	&ano\\ \hline
			3	&Ťapka&	2	&ne\\ \hline			
		\end{tabular}
	\label{tab:ukazkatabulky}
\end{table}


\subsubsection{Vkládání ukázkového kódu}

Pokud vkládáte ukázkový kód, můžete buď formátovat ručně, nebo použít zde definované prostředí \ttsmall{prog}, nebo použít vhodný balíček, můžu doporučit například algorithm2e -- informace, manuál a~samotný balíček najdete na \cite{algorithm2e}.

Doporučení pro ruční formátování:
\begin{citemize}
	\item V~daném úseku nastavte řádkování na 1 nebo jen mírně větší:\\	
	\ttsmall{\zpetnelomitko radkovani[1]}
	
	\item Použijte neproporcionální písmo a~menší font:\\	
	\ttsmall{\{\zpetnelomitko ttfamily\zpetnelomitko small}\\[-2pt]
	\ttsmall{...}\\[-2pt]
	\ttsmall{\}}
	\item Vraťte řádkování na výchozí hodnotu:\\	
	\ttsmall{\zpetnelomitko radkovani}
\end{citemize}
Ukázka využití prostředí \ttsmall{prog}:

\begin{prog}
if (pocet_bodu > 100) {
	print ("chyba při výpočtu bodů nebo zásah hackera");
	ukonci_program();
}
if (pocet_bodu > 50)
	udelit_zapocet(pocet_bodu);
else
	informuj_nahradni_terminy();
\end{prog}

Kolem tohoto prostředí vždy nechejte volný řádek.


\subsection{Pojmenované odstavce}

Finální úpravou je zajištění toho, aby na koncích řádků nebyly jednopísmenné předložky a~spojky. V~\LaTeX u~toho docílíme tak, že mezeru mezi jednopísmennou předložkou/spojkou a~následujícím slovem nahradíme vlnkou: \vlnka{}. Koncem řádku by taktéž nemělo být odděleno číslo od jednotky, tedy například 12~kg. Lze použít program \ttsmall{vlna} od Olšáka, informace a~program na \cite{vlnazdroj}.

Poznámky pod čarou používejte jen v~nejnutnějších případech -- prosím nenuťte čtenáře každou chvíli šilhat na konec stránky :-).

\paragraph{Pojmenovaný odstavec.} Takto můžete vyřešit situaci, kdy potřebujete sekci rozdělit, ale nechcete použít číslovaný nadpis (třeba proto, že tento text by byl jen krátký a~nemá smysl navážet ho do obsahu).

\section*{Závěr}

\begin{zvyraznenyodstavec}
V rámci této bakalářské práce jsme se zaměřili na analýzu a porovnání současných možností frontendového vývoje. 
Navrhli a implementovali jsme demonstrační aplikace, díky kterým jsme identifikovali možnosti a výhody použití vybraných moderních frontendových technologií. 
V neposlední řadě jsme provedli závěrečné srovnání jednotlivých implementací.

Samotná analýza frameworků Angular, React, Svelte a Vue odhalila rozdíly ve způsobech, jakými jednotlivé technologie přistupují k vývoji frontendu. 
Každá technologie disponuje jedinečnými vlastnostmi, které mohou odlišným způsobem ovlivnit vývoj webových aplikací. 

V praktické části jsme provedli implementaci aplikací pomocí tří vybraných technologií -- Angular, React a Svelte. 
Na základě implementace jsme provedli srovnání, ve kterém jsme zjistili, že každá technologie má své přednosti a nedostatky. 
Výsledky srovnání ukázaly, že výběr vhodné technologie závisí na konkrétních požadavcích a cílech projektu.

Zatímco Angular je vhodný spíše k vývoji velkých aplikací, React vyniká ve flexibilitě a široké podpoře komunity. 
V nespolední řadě Svelte překvapil svým minimalismem, jednoduchostí, ale také vysokou efektivitou.

Závěrem lze konstatovat, že nejuniverzálnější framework pro vývoj frontendu neexistuje. 
Práce splnila svůj cíl, kterým bylo poskytnout čtenáři ucelený pohled na frontendové technologie, jejich možnosti a srovnání.
Rovněž byla splněna motivace práce, která spočívala v usnadnění výběru vhodného nástroje pro vývoj frontendu čtenáři.

Budoucí rozšíření práce by mohlo spočívat v přidání testovacích scénářů, na které v rámci práce nebyl kladen důraz. 
Další rozšíření by také mohlo spočívat v přidání backendového systému, který by umožnil porovnání možností integrace s backendovými technologiemi.
\end{zvyraznenyodstavec}

% Do Závěru píšeme souhrn poznatků zjištěných v~práci, hodnotíme výsledek práce (někde by tu měla být i~větička \uv{Cíl práce byl splněn}, nějak vhodně rozvedená). 
% Také tu můžeme psát o~případných problémech, se kterými jsme se při psaní práce setkali, 
% možnostech využití, námětech na pokračování do budoucna (tj. co by se dalo zlepšit, přidat, rozšířit, atd.).


\renewcommand{\refname}{Seznam použité literatury}

\begingroup


\begin{thebibliography}{99}\radkovani[1.2]\raggedright
\label{chap:literatura}

\bibitem{reactbanks}
\textsc{BANKS}, Alex a \textsc{PORCELLO}, Eve. \emph{Learning React}. Online. 2nd Edition. O'Reilly Media, 2020. ISBN 9781492051725. Dostupné z: \iadresa{https://www.oreilly.com/library/view/learning-react-2nd/9781492051718/}. [cit. 2023-10-15].

\bibitem{reactrisingstack}
\textsc{HÁMORI}, Ferenc. \emph{The History of React.js on a Timeline}. Online. RisingStack. 2022. Dostupné z: \iadresa{https://blog.risingstack.com/the-history-of-react-js-on-a-timeline/}. [cit. 2023-10-15].

\bibitem{reacthubspot}
\textsc{HERBERT}, David. \emph{What is React.js? (Uses, Examples, \& More)}. Online. HUBSPOT. HubSpot Blog. 2022. Dostupné z: \iadresa{https://blog.hubspot.com/website/react-js}. [cit. 2023-10-15].

\bibitem{react}
\emph{React}. Online. 2023. Dostupné z: \iadresa{https://reactjs.org/}. [cit. 2023-10-15].

% nahradte svymi vlastnimi zdroji, podle abecedy


\bibitem{mytyprog}
11 Mýtů o programátorech. \emph{Green Fox Academy} [online]. 2018 [cit. 2022-07-22]. Dostupné z: \iadresa{https://www.greenfoxacademy.cz/post/11-mytu-o-programatorech}

\bibitem{autorskepravo}
\textsc{Bartoš}, Aleš. \emph{Autorské právo v otázkách a odpovědích}. Praha: Pierot, 2012. ISBN 978-80-7353-223-9.

\bibitem{csniso690}
\emph{Bibliografické citace}. Praha, 2011.

\bibitem{algorithm2e}
\textsc{Fiorio}, Christophe. Algorithm2e.sty -- package for algorithms. CTAN: Package algorithm2e [online]. 2017 [cit. 2022-09-15]. Dostupné na: \iadresa{https://www.ctan.org/pkg/algorithm2e}

\bibitem{generatorcitaci} 
Generátor citací [online]. \emph{Citace PRO} [cit. 2022-07-22]. Dostupné z: \iadresa{https://www.citacepro.com}

\bibitem{praktickatypo}  
\textsc{Kočička}, Pavel a Filip \textsc{Blažek}. \emph{Praktická typografie}. Dotisk druhého vydání. Brno: Computer Press, 2007. DTP. ISBN 80-722-6385-4.


\bibitem{overleaf}
\LaTeX, Evolved. \emph{Overleaf, online \LaTeX{} editor} [online]. Overleaf.com [cit. 2022-09-15]. Dostupné z: \iadresa{https://www.overleaf.com/}


\bibitem{akademicka}
\textsc{Meško}, Dušan, Dušan \textsc{Katuščák}, Ján \textsc{Findra} a kol. \emph{Akademická příručka}. 2. upravené a dopl. vyd. Martin: Osveta, 2005. ISBN 80-8063-200-6.

\bibitem{metodickypokyn} 
\emph{Metodický pokyn děkana č. 1/2021 ke zpracování, zveřejňování a ukládání závěrečných prací na Filozoficko-přírodovědecké fakultě v Opavě} [online]. Opava: FPF SU v Opavě, 2021 [cit. 2022-09-15]. Dostupné na: \iadresa{https://www.slu.cz/fpf/cz/sostatnizaverecnezkousky}

\bibitem{stanford}
Stanford Libraries: Find Dissertations and Theses [online]. \emph{Stanford.edu} [cit. 2018-03-19] Dostupné z: \iadresa{http://library.stanford.edu/guides/find-dissertations-and-theses}


\bibitem{vlnazdroj}
\textsc{Russl}. Tipy pro LaTeX: Program vlna pro Windows. WorkIT [online]. K-media [cit. 2022-09-15], 2014. Dostupné na:
\iadresa{https://www.work-it.cz/tipy-pro-latex-program-vlna-pro-windows/}

\bibitem{miktex}
\textsc{Schenk}, Christian. Welcome to the \MikTeX{} project page. \MikTeX{} [online]. [cit. 2022-09-15], \textcopyright 2022. Dostupné z: \iadresa{https://miktex.org/}

\bibitem{jakcist}
\textsc{Šanderová}, Jadwiga. \emph{Jak číst a psát odborný text}. Praha: Slon, 2007. ISBN 978-80-86429-40-3.



\end{thebibliography}




\endgroup

%
\listoffigures\clearpage
%
\listoftables\clearpage


\section*{Seznam zkratek}
\vspace{2em}

\noindent
\begin{tabular}{@{}ll@{}}
API	  &	Application Programming Interface\\
CLI   & Command Line Interface\\
CSS   & Cascading Style Sheets\\
DOM	  &	Document Object Model\\
DRY	  &	Don't repeat yourself\\
HTML  & HyperText Markup Language\\
HTTP  & Hypertext Transfer Protocol\\
JS    & JavaScript\\
JSX   & JavaScript XML\\
KISS	&	Keep it simple, stupid!\\
NPM   & Node Package Manager\\
OOP	  &	Objektově orientované programování\\
SOLID	&	Návrhový princip SOLID\\
SPA   & Single Page Application\\
TS    & TypeScript\\
UI    & User Interface\\
URL   & Uniform Resource Locator
\end{tabular}



%%%%%%%%%%%%%%%%%%%%%
% Pokud máte přílohy:

\clearpage
%
\addcontentsline{toc}{section}{{Přílohy }}
\clearpage
%
\appendix
%

%%%%%%%%%%%%%%%%%%%%%
%Seznam priloh:
  \rule{0pt}{1pt}\thispagestyle{empty}
  %\cleardoublepage\thispagestyle{empty}
    \vspace*{\stretch{1}}\par
    \noindent{\Huge\bfseries\scshape\appendixpagename}
    \vspace{7em}
    
% Do tohoto seznamu napište přílohy vložené přímo do této práce a také seznam elektronických příloh, které se vkládají přímo do archivu závěrečné práce v informačním systému zároveň se souborem závěrečné práce. Elektronickými přílohami mohou být například soubory zdrojového kódu aplikace či webových stránek, předpřipravený produkt (spustitelný soubor, kontejner apod.), vytvořená metodická příručka, tutoriál... (tento text odstraňte)
		
\begin{citemize}
	% \item Přílohy v souboru závěrečné práce:
	% \begin{citemize}
	% 	\item Příloha A\quad xxxx
	% 	\item 
	% \end{citemize}
	\item Elektronické přílohy:
	\begin{citemize}
		\item FPF\_BP\_2024\_60639\_Sukeník\_Lukáš.pdf
		\item webove\_aplikace.zip
	\end{citemize}
\end{citemize}
    \vspace{\stretch{4}}\rule{0pt}{1pt}

  \clearpage
  \pagenumbering{roman}

%\include{priloha01_prvnipriloha} % pro nadpis použijte "normální" \section{xxx}
%\include{priloha02_zdroje}


\end{document}
