\section{Testování frameworků}

\begin{citemize}
	\item proč a co je obsahem kapitoly?
\end{citemize}

\subsection{Analýza a návrh testových úloh}

\begin{citemize}
	\item co a proč porovnávám,
	\item v návrhu - jak, jaké testové úlohy?
	\item (dokumentace - možná nahoře, syntax, výkonnostní testy, velikosti bundlů, účel aplikace, rychlost, srozumitelnost, ...)
\end{citemize}

\subsection{Demonstrační aplikace}

V této kapitole srovnáme implementaci stejných funkcionalit ve třech vybraných frameworcích.

\subsubsection{Angular}

\begin{flushleft}
  \textbf{Správa stavů}
\end{flushleft}

\begin{flushleft}
  \textbf{UI interakce (Dropdown)}
\end{flushleft}

\begin{flushleft}
  \textbf{Předávání vlastností, získávání dat z API}
\end{flushleft}

\subsubsection{React}

\begin{flushleft}
  \textbf{Správa stavů}
\end{flushleft}

\begin{flushleft}
  \textbf{UI interakce (Dropdown)}
\end{flushleft}

\begin{flushleft}
  \textbf{Předávání vlastností, získávání dat z API}
\end{flushleft}

\subsubsection{Svelte}

\begin{flushleft}
  \textbf{Správa stavů}
\end{flushleft}

\begin{flushleft}
  \textbf{UI interakce (Dropdown)}
\end{flushleft}

\begin{flushleft}
  \textbf{Předávání vlastností, získávání dat z API}
\end{flushleft}

\subsection{Testování aplikací a výsledky}

\begin{citemize}
	\item výsledky a průběh z 3.1
\end{citemize}
