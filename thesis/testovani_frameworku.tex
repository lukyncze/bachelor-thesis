\section{Testování frameworků}

\begin{citemize}
	\item proč a co je obsahem kapitoly?
\end{citemize}

\subsection{Analýza a návrh testových úloh}

\begin{citemize}
	\item co a proč porovnávám,
	\item v návrhu - jak, jaké testové úlohy?
	\item (dokumentace - možná nahoře, syntax, výkonnostní testy, velikosti bundlů, účel aplikace, rychlost, srozumitelnost, ...)
\end{citemize}

\subsection{Demonstrační aplikace}

V této kapitole srovnáme implementaci stejných funkcionalit ve třech vybraných frameworcích.

\subsubsection{Angular}

\begin{flushleft}
  \textbf{Instalace projektu}
\end{flushleft}

\begin{citemize}
	\item Node.js + NPM
  \item npm init @angular@latest NAZEV\_APLIKACE
  \item \iadresa{https://www.npmjs.com/package/@angular/create}
  \item \iadresa{https://tailwindcss.com/docs/guides/angular}
\end{citemize}

\begin{flushleft}
  \textbf{Správa stavů}
\end{flushleft}

Pro implementaci jednoduchého counteru nejprve vytvoříme counter komponentu. Můžeme začít se strukturou HTML značek pro hlavní komponentu. 
Protože chceme opakovaně použít logiku jednotlivých tlačítek, vytvoříme komponentu counter-button. 
Ta může přijímat například nějaké CSS styly nebo přes EventEmitter (output) posílat informaci o kliknutí na tlačítko směrem nahoru ve stromě komponent. 
Funkci emit() našeho EventEmitteru zavoláme na tlačítku v counter-buttonu právě tehdy, když uživatel klikne na tlačítko -- použijeme listener ve formě (click). 
K propsání textu či jiných elementů nebo komponent mezi párovými tagy <counter-button></counter-button> nám pak poslouží párový či nepárový element <ng-content />.

Následně v counter komponentě musíme importovat třídu CounterButtonComponent a do všech elementů counter-button předat jejich vstupy a výstupy. 
Námi defikovanovanému outputu buttonClicked předáme v šabloně metodu, která se vykoná po emitu (kliknutí na tlačítko ve vnořené komponentě) a metodu zavoláme pomocí kulatých závorek. 
V rámci counter komponenty pak definujeme stav jako vlastnost count na třídě. Vlastnost pak můžeme modifikovat skrze metody třídy, které voláme v outputu buttonClicked.

\begin{citemize}
	\item šablony + logika komponenty
	\item správa stavů (reaktivita)
	\item body k vypíchnutí: boilerplate frameworku
\end{citemize}

\begin{flushleft}
  \textbf{Interakce v uživatelském prostředí}
\end{flushleft}

Při tvorbě jakékoli UI komponenty můžeme začít jak šablonou, tak i logikou. My začneme s tvorbou šablony. V případě vlastního dropdown samotným tlačítkem a seznamem možností. 
Otevření možností zajístíme tak, že na tlačítko přidáme click listerner. Funkčnost pak zajistíme díky modifikaci stavu isOpen, který se provede při volání metody toggleDropdown. 
V rámci této metody je třeba zavolat i event.stopPropagation(). Předejdeme tak potenciální chybě ve formě tzv. event bubblingu -- spuštění událostí na prvcích odlišných od cílového. 

Podmíněně pak můžeme vypsat list možností, které získáme v jednom z inputů. Pro vypsání všech možností použijeme blok @for. 
K vybraní konkrétní možnosti použijeme zase (click) a do metody pošleme konkrétní možnost pole -- option. 
Metoda handleOptionClick pak zajistí uložení aktuálně vybrané možnosti, zavření dropdownu a vyemitování vybrané možnosti do rodičovské komponenty.

V případě, že máme dropdown otevřen a chceme jej po kliknutí mimo tentýž dropdown bezpečně zavřít nehledě na počet vykreslených dropdown komponent na stránce budeme postupovat následovně. 
Pro každou komponentu vytvoříme unikátní vlastnost ve formě ID. To pak dynamicky umístíme na kořenový element dropdownu. 
V komponentě pak budeme naslouchat na DOM eventy pomocí dekorátoru @HostListener. 
Přijímá DOM event, na který má poslouchat -- document:pointerdown, případně další argumenty nebo také formu vypublikovaného eventu. 
Pod dekorátorem pak definujeme obslužnou metodu, která se volá při emitu specifikovaného eventu. V rámci metody pak zajistíme uzavření aktuálně otevřeného dropdownu.

Dropdown pak může mít různé inputy, které povedou k lepší znovupoužitelnosti. Hodnotu inputu (konkrétně např. defaultValue) v komponentě získáme až v lifecycle hooku OnInit. 
Kupříkladu v konstruktoru bychom dostali pouze undefined. Styly ve formě JavaScriptových hodnot do šablony přidáme pomocí ngClass. 
Když těchto hodnot chceme na elementu více, musíme je zřetězit pomocí JavaScriptu, nebo sloučit již dříve.

\begin{citemize}
	\item body k vypíchnutí: dynamické stylování, logika v template
	\item problémy: zavírání posledně otevřeného dropdownu před otevřením dalšího D.
	\item výhody frameworku: podle bodů nahoře..., tvorba typů ve Svelte
\end{citemize}

\begin{flushleft}
  \textbf{Předávání vlastností, získávání dat z API}
\end{flushleft}

Pro ukázku předávání vlastností a získávání dat z API můžeme vytvořit komponentu, která bude překládat zadaný text do vybraného jazyka. 
Začneme tedy vytvořením rodičovské komponenty, která nám při změně vlastností (zadaného textu uživatelem a výstupního jazyka) zavolá API, které nám vrátí přeložený text. 
V rámci této komponenty vytvoříme vnořené komponenty, které budou sloužit k zadání vstupního textu, výběru jazyka a zobrazení výsledku. 

LanguageDropdownComponent umožní uživateli vybrat jazyk, do kterého chce přeložit text. 
Přes EventEmitter aktualizujeme výstupní jazyk v rodičovské komponentě. V rámci obslužné metody handleLanguageChange pak také aktualizujeme hodnotu vlastnosti inputValuesChanges\$.
Tato vlastnost je Subject, speciální typ observable, z knihovny RxJS. Později nám na základě změny hodnoty dovolí poslat dotaz na server ve správný moment. 
Podobným způsobem poté můžeme naslouchat na změny vstupního textu.

Zadání vstupního textu pak může řešit komponenta TranslationInputComponent, která obdobným způsobem aktualizuje hodnotu vstupního textu v rodičovské komponentě. 
Aktuální hodnotu formulářového prvku nastavíme pomocí [ngModel]. Pro naslouchání na změnu hodnoty formulářového prvku zase využijeme (ngModelChange). 
V případě, že chceme aktualizovat výšku textového pole na základě jeho obsahu, můžeme využít vlastní direktivu AutosizeTextAreaDirective. 
V konstruktoru direktivy získáme element, na který přidáme tuto direktivu. Dále budeme potřebovat třídu Renderer2, která nám umožní manipulovat s DOM. 
V direktivě budeme naslouchat na změnu hodnoty textového pole pomocí dekorátoru @HostListener a události input. Následně v rámci obslužné metody zajistíme aktualizaci výšky.
 
Změny hodnoty vlastnosti inputValuesChanges\$ musíme začít odebírat pomocí subscribe. 
Abychom předešli dotazování serveru ihned po změně hodnoty vlastnosti inputValuesChanges\$, použijeme operátor debounceTime. 
Ten nám povolí poslat dotaz na server až po uplynutí určité doby od poslední změny. 
Subscribe zavolá věřejnou metodu služby (getTranslation), která nám vrátí přeložený text. 
Nakonec aby nám dotazování serveru fungovalo, je třeba metodu setupInputChangeSubscription zavolat v konstruktoru nebo hooku OnInit.

Se službou TranslationService a veřejnou metodou pro vykonání dotazu nám pomůže třída HttpClient. Ta je dostupná přímo v základních modulech Angularu.
Službu HttpClient získáme v konstruktoru, kde ji pomocí klíčového slova private přiřadíme do vlastností třídy. 
Následně na HTTP klientovi zavoláme metodu post vůči API, které nám vrátí přeložený text. 
Pokud nám v úspěšné odpovědi ze serveru přijde nějaká složitější struktura, z které chceme vrátit jen nějakou část, pak nám s konverzí odpovědi pomůže RxJS operátor map(). 
Metoda getTranslation vrací observable, proto musíme v translator komponentě hodnoty odebírat pomocí metody subscribe. 

V momentě, kdy dostaneme odpověď ze serveru, chceme zobrazit přeložený text uživateli. 
K tomu nám poslouží TranslationOutputComponent, které na vstupu předáme výstupní text spolu s dalšími vstupními vlastnostmi. 
V rámci šablony pak podmíněně vykreslíme přeložený text, chybu nebo načítání. 

Při zarovnání vstupního a výstupního pole v UI si musíme dát pozor na to, že šířku musíme nastavit již v prvním potomku div elementu, na kterém nastavíme flexbox. 
Důvod je zřejmý -- Angular v DOM vytvoří speciálně element pro každou komponentu.

\begin{citemize}
	\item předávání vlastností nahoru a dolů
	\item fetchování dat
	\item body k vypíchnutí: velice odlišné reakce na změny, stylování komponent nebo elementů, update textarey (hodnoty), jiné řešení modularity (update stylů textarey)
	\item problémy:
	\item výhody frameworku: předávání vlastností má nej Svelte
\end{citemize}

\begin{flushleft}
  \textbf{Tvorba formulářů}
\end{flushleft}



\begin{flushleft}
  \textbf{Country guesser}
\end{flushleft}

\begin{flushleft}
  \textbf{Layout a routování}
\end{flushleft}

\subsubsection{React}

\begin{flushleft}
  \textbf{Instalace projektu}
\end{flushleft}

\begin{flushleft}
  \textbf{Správa stavů}
\end{flushleft}

\begin{flushleft}
  \textbf{Interakce v uživatelském prostředí}
\end{flushleft}

\begin{flushleft}
  \textbf{Předávání vlastností, získávání dat z API}
\end{flushleft}


\subsubsection{Svelte}

\begin{flushleft}
  \textbf{Instalace projektu}
\end{flushleft}

\begin{flushleft}
  \textbf{Správa stavů}
\end{flushleft}

\begin{flushleft}
  \textbf{Interakce v uživatelském prostředí}
\end{flushleft}

\begin{flushleft}
  \textbf{Předávání vlastností, získávání dat z API}
\end{flushleft}


\subsection{Testování aplikací a výsledky}

\begin{citemize}
	\item výsledky a průběh z 3.1
\end{citemize}

