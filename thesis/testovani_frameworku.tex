\section{Testování frameworků}

\begin{citemize}
	\item proč a co je obsahem kapitoly?
\end{citemize}

\subsection{Analýza a návrh testových úloh}

\begin{citemize}
	\item co a proč porovnávám,
	\item v návrhu - jak, jaké testové úlohy?
	\item (dokumentace - možná nahoře, syntax, výkonnostní testy, velikosti bundlů, účel aplikace, rychlost, srozumitelnost, ...)
\end{citemize}

\subsection{Demonstrační aplikace}

V této kapitole srovnáme implementaci stejných funkcionalit ve třech vybraných frameworcích.

\subsubsection{Angular}

\begin{flushleft}
  \textbf{Instalace projektu}
\end{flushleft}

\begin{citemize}
	\item Node.js + NPM
  \item npm init @angular@latest NAZEV\_APLIKACE
  \item \iadresa{https://www.npmjs.com/package/@angular/create}
  \item \iadresa{https://tailwindcss.com/docs/guides/angular}
\end{citemize}

\begin{flushleft}
  \textbf{Správa stavů}
\end{flushleft}

\begin{citemize}
	\item šablony + logika komponenty
	\item správa stavů (reaktivita)
	\item body k vypíchnutí: boilerplate frameworku
\end{citemize}

\begin{flushleft}
  \textbf{UI interakce (Dropdown)}
\end{flushleft}

\begin{citemize}
	\item body k vypíchnutí: dynamické stylování, logika v template
	\item problémy: zavírání posledně otevřeného dropdownu před otevřením dalšího D.
	\item výhody frameworku: podle bodů nahoře..., tvorba typů ve Svelte
\end{citemize}

\begin{flushleft}
  \textbf{Předávání vlastností, získávání dat z API}
\end{flushleft}

\begin{citemize}
	\item předávání vlastností nahoru a dolů
	\item fetchování dat
	\item body k vypíchnutí: velice odlišné reakce na změny, stylování komponent nebo elementů, update textarey (hodnoty), jiné řešení modularity (update stylů textarey)
	\item problémy:
	\item výhody frameworku: předávání vlastností má nej Svelte
\end{citemize}

\subsubsection{React}

\begin{flushleft}
  \textbf{Instalace projektu}
\end{flushleft}

\begin{flushleft}
  \textbf{Správa stavů}
\end{flushleft}

\begin{flushleft}
  \textbf{UI interakce (Dropdown)}
\end{flushleft}

\begin{flushleft}
  \textbf{Předávání vlastností, získávání dat z API}
\end{flushleft}


\subsubsection{Svelte}

\begin{flushleft}
  \textbf{Instalace projektu}
\end{flushleft}

\begin{flushleft}
  \textbf{Správa stavů}
\end{flushleft}

\begin{flushleft}
  \textbf{UI interakce (Dropdown)}
\end{flushleft}

\begin{flushleft}
  \textbf{Předávání vlastností, získávání dat z API}
\end{flushleft}

\subsection{Testování aplikací a výsledky}

\begin{citemize}
	\item výsledky a průběh z 3.1
\end{citemize}
