\subsection{Angular}

Angular byl původně interní JavaScriptový framework společnosti Google. 
Dle oficiální dokumentace \cite{angulario} je Angular kompletní platforma, určená k vývoji webových aplikací. 
Framework je postaven na principu komponent, jež tvoří základní stavební jednotku aplikace. 
Součástí frameworku jsou knihovny, které jsou velmi dobře integrovatelné a usnadňují práci s různými částmi aplikace. 
Dále také nástroje, které vývojářům usnadňují vývoj, testování či aktualizaci kódu.\cite{angulario,learningangular}

První verze Angularu byla vydána v roce 2012 pod názvem AngularJS. 
V roce 2016, po kolaboraci se společností Microsoft, Google vydal novou verzi, o které mluvíme jako o Angular 2. 
Verze 2 byla kompletně přepsána, framework byl přepsán z JavaScriptu do jazyku TypeScript. 
Součástí frameworku je i knihovna RxJS pro práci s asynchronními událostmi. 
Knihovnu reaktivních rozšíření, RxJS, mohou využívat také programátoři při vývoji Angular aplikací. 
Angular se samozřejmě neustále vyvíjí a v současné době můžeme pracovat s verzí 17.\cite{angulardev,learningangular}

Díky robusnosti a velikosti frameworku je Angular vhodnější pro větší aplikace, které vyžadují mnoho funkcí a komplexních struktur. 
V poslední době framework spíše ztrácí na popularitě, stále jej však používá mnoho společností včetně Google.\cite{learningangular}

\subsubsection{Komponenty}

Hlavní bloky kódu v Angularu tvoří komponenty. Komponenta je OOP třída definovaná v rámci TypeScript souboru ve formátu nazev-tridy.component.ts. 
Komponenta obsahuje nastavení v dekorátoru @Component, v němž definujeme selektor a ostatní části (soubory) komponenty. 
Dále v metadatech může být nastavení použitého API, šablona nebo kaskádové styly. 
Mezi další části komponenty patří soubor s šablonou, kaskádovými styly či testovací scénáře. 
Tyto soubory jsou nepovinné, obvykle jsou však žádoucí a vedou k lepší organizaci kódu. 

Původní komponenty byly tvořeny pomocí modulů (NgModules), které obsahovaly deklarace komponent, direktiv a služeb. 
Od verze 14 je možné využít standalone API, které umožňuje vytvářet komponenty bez nutnosti tvoření a registrace modulů v jiných souborech, a díky tomu je kód kratší. 
V dekorátoru @Component pak je třeba importovat všechny potřebné závislosti -- direktivy, služby či jiné komponenty.\cite{angulardev,learningangular}

V šabloně vykreslíme hodnoty pomocí dvojitých závorek, které obsahují název proměnné. K podmíněnému zobrazení komponent/elementu slouží bloky @if, @else if, @else. 
Pro iteraci přes pole hodnot slouží blok @for.\cite{angulardev}

\begin{prog}
import \{ Component \} from '@angular/core';

@Component(\{
  selector: 'my-component',
  standalone: true,
  template: `
    <div>\{\{ content \}\}</div>
  `,
  styles: ['div \{ background-color: red; \}']
\})
export class MyComponent \{
  content = 'nějaký-obsah';
\}
\end{prog}

\subsubsection{Správa stavů}

K vytvoření stavů v Angularu využijeme vlastnosti třídy (class fields). Vlastnosti mohou být veřejné (public), protected (chráněné) nebo soukromé (private). 
V případě, že viditelnost nespecifikujeme, je vlastnost veřejná. Soukromé vlastnosti jsou viditelné pouze v rámci třídy, chráněné v rámci třídy a šablony. 
Veřejné vlastnosti jsou viditelné odkudkoli.\cite{angulardev,learningangular}

Pro aktualizaci stavů využijeme přiřazení nové hodnoty do vlastnosti, k níž přistoupíme pomocí klíčového slova this.\cite{angulardev}

\begin{prog}
import \{ Component \} from '@angular/core';

@Component(\{
  selector: 'my-component',
  standalone: true,
  template: `
    <button (click)="increment()">
      Klikli jste na tlačítko \{\{ count \}\}x.
    </button>
  `,
\})
export class MyComponent \{
  protected count = 0;

  protected increment(): void \{
    this.count++;
  \}
\}
\end{prog}

Počínaje verzí 16 můžeme používat také nestabilní verzi signálů (Angular Signals), přičemž se jedná o systém, který mnohem efektivněji sleduje využití stavů v rámci aplikace. 
Signály pak umožňují efektivnější a optimalizavanější aktualizace DOM. 

\begin{prog}
import \{ Component, signal \} from '@angular/core';

@Component(\{
  selector: 'my-component',
  standalone: true,
  template: `
    <button (click)="increment()">
      Klikli jste na tlačítko \{\{ count() \}\}x.
    </button>
  `,
\})
export class MyComponent \{
  protected count = signal(0);

  protected increment(): void \{
    this.count.update((value) => value + 1);
  \}
\}
\end{prog}

Od verze 17 jsou signály stabilní součástí Angularu, na druhou stranu předávání a modely signálů zatím stabilní nejsou. 
Proto v této práci budeme využívat klasické hodnoty stavů.\cite{angulardev}

\subsubsection{Předávání vlastností}

Předávat vlastnosti či jiné hodnoty je možné pomocí vstupního dekorátoru @Input a výstupního dekorátoru @Output. 
V rámci šablony danou hodnotu předáme do vnořené komponenty skrze název vstupu v hranatých závorkách.   
Ve vnořené komponentě využijeme dekorátor @Input, sloužící k získání hodnot z rodičovské komponenty. K předaným hodnotám není možné přistupovat v rámci konstruktoru. 

K předání hodnoty z vnořené komponenty nejprve předáme vnořené komponentě název výstupu v kulatých závorkách a obslužnou metodu, která se vykoná po předání vlastnosti. 
Dále definujeme @Output ve vnořené komponentě, na němž zavoláme metodu emit(), která přes argument metody umožňuje předání (vypublikování) hodnot z vnořené do rodičovské komponenty.\cite{angulardev,learningangular}

\begin{prog}
import \{ CommonModule \} from '@angular/common';
import \{ Component, Input, Output, EventEmitter \} from '@angular/core';

@Component(\{
  selector: 'parent-component',
  standalone: true,
  imports: [ChildComponent],
  template: `
    <child-component 
      [color]="someProps.color" 
      (colorClicked)="handleColorClicked(\$event)" 
    />
    <p>\{\{ colorClickedText \}\}</p>
  `,
\})
export class ParentComponent \{
  protected someProps = \{ color: 'cervena' \};
  protected colorClickedText = 'Na název barvy jste zatím neklikli.';

  protected handleColorClicked(clickCount: number): void \{
    this.colorClickedText = `Na název barvy 
      v ChildComponent jste klikli: \$\{clickCount\}x`;
  \}
\}

@Component(\{
  selector: 'child-component',
  standalone: true,
  imports: [CommonModule],
  template: `<p [ngClass]="color" (click)="handleClick()">\{\{ color \}\}</p>`,
\})
export class ChildComponent \{
  private clickCount = 0;

  @Input() color = 'Žádná barva nebyla specifikována.';
  @Output() colorClicked = new EventEmitter<number>();

  protected handleClick(): void \{
    this.clickCount++;
    this.colorClicked.emit(this.clickCount);
  \}
\}
\end{prog}

\subsubsection{Služby, direktivy, roury}

Angular disponuje pestrou škálou možností, jak sdílet bloky kódu či logiku mezi různými částmi aplikace. Služby, direktivy, nebo také roury tvoříme pomocí JavaScriptových tříd. 
Služby (services) umožňují znovupoužití určité části kódu např. v rámci více komponent. 
Služby obvykle používáme ke komunikaci s HTTP endpointy, sdílíme v nich stavy více komponent, a také je využíváme k transformacím.\cite{angulardev,learningangular}

\begin{prog}
import \{ Injectable \} from '@angular/core';
  
@Injectable(\{ providedIn: 'root' \})
export class DataService \{
  private data = 'Initial data';

  getCurrentData(): string \{
    return this.data;
  \}
      
  setNewData(newData: string): void \{
    this.data = newData;
  \}
\}
\end{prog}

Direktivy (directives), ať už zabudované či vlastní, slouží k obohacení HTML elementů o různé funkce (dynamické atributy, CSS třídy) nebo manipulaci s DOM elementy. 
Roury (pipes) umožňují trasformaci hodnot v šabloně.\cite{angulardev,angulario}

\begin{prog}
import \{ Component, Pipe, PipeTransform \} from '@angular/core';

@Pipe(\{ name: 'czechDate', standalone: true \})
export class CzechDateFormatPipe implements PipeTransform \{
  transform(value: Date): string \{
    return value.toLocaleDateString('cs-CZ');
  \}
\}

@Component(\{
  selector: 'my-component',
  standalone: true,
  imports: [CzechDateFormatPipe],
  template: `<p>\{\{ today | czechDate\}\}</p>`,
\})
export class MyComponent \{
  protected today = ;
\}
\end{prog}

\subsubsection{Životní cyklus}

Životním cyklem komponenty rozumíme sekvenci kroků, které se vykonají mezi vytvořením a zničením komponenty. 
Životní cyklus může být rozdělen do čtyř částí: vytvoření, aktualizace, vykreslení a zničení. 

První metoda, která se spouští při vytvoření komponenty, je konstruktor. 
Dále můžeme využít metody ngOnInit, ngOnChanges, ngDoCheck, ngAfterViewInit, ngAfterContentInit, ngAfterViewChecked, ngAfterContentChecked, jež se spouští v různých fázích aktualizace (detekce). 
Nedávno byly přidány také metody po vykreslení komponenty -- afterNextRender a afterRender. V neposlední řadě je možné využít metodu ngOnDestroy, která se spouští při zničení komponenty.\cite{angulardev,learningangular} 
% super tabulka: https://angular.dev/guide/components/lifecycle

\subsubsection{State management}

K základní práci se stavy slouží vlastnosti třídy, které inicializujeme JavaScriptovou hodnotou. 
Do budoucna můžeme počítat s lepší podporou signálů, které aktualizují DOM efektivněji a rychleji. 
Pokročilejší způsob sdílení stavů v rámci aplikace spočívá ve využití služeb, v nichž uložíme stavy a následně je sdílíme mezi komponentami.\cite{angulardev}

Pro reaktivní či asynchronní operace, nebo obecně složitější funkcionality, využijeme balíček RxJS, který je již součástí Angularu. 
RxJS poskytuje datový typ Observable, který reprezentuje data, jež se mohou měnit v čase.\cite{angulario,rxjslibrary}

V případě, že potřebujeme sdílet stavy globálně (mezi různými částmi aplikace), můžeme využít knihovnu NgRx, která je inspirována knihovnou Redux.\cite{angularstatemanagement,ngrxlib}

\subsubsection{Routování}

Angular poskytuje vestavěný systém routování -- konkrétně balíček @angular/router, který na straně klienta umožňuje přepínat mezi různými částmi aplikace. 
Klasické webové stránky při změně URL pokaždé žádají o nové dokumenty. Routování na straně klienta může provést aktualizaci stránky bez dalších duplikátních dotazů. 
Při vyžádání dané cesty pak vykreslíme požadovaný obsah a požádáme pouze o data potřebné pro vykreslení. 
Získáme tak rychlejší uživatelskou zkušenost, jelikož prohlížeč nevyžaduje nové dokumenty a nemusí vyhodnocovat kaskádové styly či JavaScript.

Abychom zaregistrovali cesty aplikace pomocí knihovny @angular/router, poskytneme samotný router pomocí funkce provideRouter do aplikačního nastavení. 
Routeru následně předáme pole cest aplikace. Jednotlivé obsahy stránek, které se mají zobrazit, vykreslíme pomocí elementu router-outlet.\cite{angulardev,learningangular}

\subsubsection{Ekosystém}

Angular je komplexní framework, v němž jsou obsaženy základní balíčky všeho druhu potřebné pro vývoj webových aplikací. 
Framework se stále vyvíjí a jeho ekosystém se neustále rozrůstá, a to i přes velikost projektu. 
Angular nepostrádá ani v množství balíčků třetích stran, které vývojáři využívají pro usnadnění práce s různými částmi aplikace. 
Nechybí ani balíček @angular/cli, jenž usnadňuje vývoj aplikace, testování, aktualizaci kódu a jeho nasazení.\cite{angulardev,learningangular}