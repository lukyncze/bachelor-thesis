\subsection{Svelte}

Svelte je relativně novým open-source JavaScript frameworkem, za jejímž stvořením stojí vývojář Richard Harris. 
Framework se odlišuje tím, že kompiluje komponenty do čistého JavaScriptu. To vše ještě před tím, než uživatel navštíví webovou aplikaci v prohlížeči. 
Tato metoda poskytuje výhodu hlavně co se týče rychlosti oproti klasickým deklarativním frameworkům jako jsou např. React, Vue nebo Angular. 
Stejně jako tyto frameworky je Svelte určen k vývoji rychlého a kompaktního uživatelského rozhraní pro webové aplikace.

První verze byla představena ke konci roku 2016. Verze 3, jež byla vydána v dubnu 2019, přinesla vylepšení týkající se zjednodušení tvorby komponent. 
Mimo jiné tato verze hlavně představila vylepšení ve smyslu reaktivity. Po této verzi framework nabral na popularitě díky jeho jednoduchosti.
Verze 4 pak v roce 2023 představila pouze minimální změny, jež spočívají v údržbě a přípravách pro verzi nastávající.

Přestože Svelte nedisponuje rozsáhlým ekosystémem jako jiné JavaScriptové frameworky, získal si přízeň mnoha velkých společností. 
Mezi ně patří například firmy jako The New York Times, Avast, Rakuten a Razorpay.\cite{sveltemdn,svelte,sveltedevinterface}

% Svelte CZ https://www.sveltejs.cz/
% Bakalarka https://www.theseus.fi/bitstream/handle/10024/500643/Oksanen_Miikka.pdf
% Diplomka
% https://www.doria.fi/handle/10024/177433
% https://www.doria.fi/bitstream/handle/10024/177433/levlin_mattias.pdf?sequence=2&isAllowed=y
% Kniha
% https://www.syncfusion.com/succinctly-free-ebooks/svelte-succinctly
% Joy of code
% https://joyofcode.xyz/svelte-for-beginners

\subsubsection{Komponenty}

Podobně jako v Reactu, komponenty jsou základní stavební bloky Svelte. Komponentu tvoří HTML, CSS a JavaScript, kde vše patří do jednoho souboru s příponou .svelte. 
Všechny tři části komponenty jsou nepovinné. Logika komponenty musí být zapsána mezi párové script tagy. Následuje jeden nebo více značek pro definovaní šablony komponenty. 
V neposlední řadě kaskádové styly se zapisují mezi style tagy.

V rámci šablony Svelte umožňuje využívat logické bloky pro podmíněné vykreslování nebe také iterace přes pole hodnot (list). 
Zabudovaná je i podpora manipulace s asynchronním JavaScriptem - promises.\cite{svelte}

\subsubsection{Reaktivita}

Srdcem Svelte jsou reaktivní stavy komponenty, které jednoduše definujeme jako proměnné v JavaScriptu. Jejich hodnotu aktualizuje JavaScript funkce pomocí přidělování nových hodnot. 
Kupříkladu stav o datovém typu pole tudíž nelze aktualizovat pouze pomocí metody push či splice. Je nutné využít jiné intuitivní řešení pomocí přidělení nové hodnoty.
O všechno ostatní se pak ale doslova postará sám Svelte v pozadí. Svelte aktualizuje DOM při každé změně stavu komponenty. 

Mezi specifické funkce Svelte patří reaktivní deklarace, které se starají o aktualizaci stavů na základě stavů jiných. 
Další zabudovanou funkcí jsou tzv. reactive statements, jež umožní definovat akce, které se mají vykonat reaktivně -- jako reakce na nějaký výrok.\cite{sveltehandbook,svelte}

\subsubsection{Vlastnosti}
\subsubsection{Eventy}
\subsubsection{Životní cyklus}
\subsubsection{State management}
\subsubsection{Routování}
\subsubsection{Ekosystém}