\subsection{Svelte}

% Kniha: https://www.syncfusion.com/succinctly-free-ebooks/svelte-succinctly
% https://vercel.com/docs/beginner-sveltekit
Svelte je relativně novým open-source JavaScript frameworkem, za jejímž stvořením stojí vývojář Richard Harris. 
Framework kompiluje komponenty přímo do čistého nativního a vysoce optimalizovaného JavaScriptu bez potřeby runtime. 
To vše ještě před tím, než uživatel navštíví webovou aplikaci v prohlížeči. 
Tato metoda poskytuje výhodu hlavně co se týče rychlosti oproti klasickým deklarativním frameworkům jako jsou např. React, Vue nebo Angular. 
Stejně jako tyto frameworky je Svelte určen k vývoji rychlého a kompaktního uživatelského rozhraní pro webové aplikace.

První verze byla představena ke konci roku 2016. Verze 3, jež byla vydána v dubnu 2019, přinesla vylepšení týkající se zjednodušení tvorby komponent. 
Mimo jiné tato verze hlavně představila vylepšení ve smyslu reaktivity. Po této verzi framework nabral na popularitě díky jeho jednoduchosti.
Verze 4 pak v roce 2023 představila pouze minimální změny, jež spočívají v údržbě a přípravách pro verzi nastávající.

Přestože Svelte nedisponuje rozsáhlým ekosystémem jako jiné JavaScriptové frameworky, získal si přízeň mnoha velkých společností. 
Mezi ně patří například firmy jako The New York Times, Avast, Rakuten a Razorpay.\cite{sveltemdn,svelte,sveltedevinterface}

\subsubsection{Komponenty}

Podobně jako v Reactu, komponenty jsou základní stavební bloky Svelte. Komponentu tvoří HTML, CSS a JavaScript, kde vše patří do jednoho souboru s příponou .svelte. 
Všechny tři části komponenty jsou nepovinné. Logika komponenty musí být zapsána mezi párové script tagy. Následuje jedna nebo více značek pro definovaní šablony komponenty. 
V neposlední řadě kaskádové styly se zapisují mezi style tagy.

V rámci šablony Svelte umožňuje využívat logické bloky pro podmíněné vykreslování nebe také iterace přes pole hodnot (list). 
Zabudovaná je i podpora manipulace s asynchronním JavaScriptem - promises.\cite{svelte}

\begin{prog}
<script>
  let content = 'nějaký-kontent';
</script>

<div>\{content\}</div>
  
<style>
  div \{
    background-color: red;
  \}
</style>
\end{prog}

\subsubsection{Reaktivita}

Srdcem Svelte jsou reaktivní stavy komponenty, které jednoduše definujeme jako proměnné v JavaScriptu. Jejich hodnotu aktualizuje JavaScript funkce pomocí přidělování nových hodnot. 
Kupříkladu stav o datovém typu pole tudíž nelze aktualizovat pouze pomocí metody push či splice. Je nutné využít jiné intuitivní řešení pomocí přidělení nové hodnoty.
O všechno ostatní se pak postará sám Svelte v pozadí. Svelte aktualizuje DOM při každé změně stavu komponenty. 

Mezi specifické funkce Svelte patří reaktivní deklarace, které se starají o aktualizaci stavů na základě stavů jiných. 
Další zabudovanou funkcí jsou tzv. reactive statements, jež umožní definovat akce, které se mají vykonat reaktivně -- jako reakce na nějaký výrok.\cite{sveltehandbook,svelte}

\begin{prog}
<script>
  let count = 0;

  function increment() \{
    count++;
  \}
</script>

<button on:click=\{increment\}>
  Klikli jste na tlačítko \{count\}x.
</button>
\end{prog}

\subsubsection{Předávání vlastností}

Pro komunikaci mezi komponentami slouží mechanismus předávání vlastností. 
V rodičovské komponentě je nutné komponentě říci, jakou hodnotu chceme předat a do jaké proměnné ji chceme uložit v child komponentě. 
Pak v child komponentě vytvoříme stejnojmennou vlastnost s klíčovým slovem export.

Pokud chceme předávat vlastnosti parent komponentě, je třeba vytvořit vlastnost již na parent komponentě. 
Následně ji předat child komponentě a v rodičovské komponentě přidat před předání vlastnosti do komponenty klíčové slovo bind.\cite{svelte}

\begin{prog}
// Parent.svelte
<script>
  import ChildComponent from './Child.svelte';

  const someProps = \{color: 'cervena'\};
</script>

<ChildComponent color='cervena' />
<ChildComponent color=\{someProps.color\} />
<ChildComponent \{...someProps\} />

// Child.svelte
<script>
  export let color;
</script>

<div class=\{color\}></div>
\end{prog}

\subsubsection{Eventy}

Svelte má velice jednoduché API pro práci s DOM eventy. Stačí použít direktivu on na HTML elementu, která vyžaduje název eventu a callback funkci.

\begin{prog}
<script>
  let count = 0;
</script>

<button on:click=\{() => count++\}>
  Klikli jste na tlačítko \{count\}x.
</button>
\end{prog}

Vývojáři také přišli s možností odesílání a přijímání eventů pro komponenty. 
V child komponentě je třeba mít nějaký DOM event handler, na který chceme reagovat v parent komponentě. 
Poté je nutné využít zabudovanou metodu createEventDispatcher, které předáme potřebné parametry -- náš libovolný název pro event komponenty a hodnotu. 
V rodičovské komponentě pak reagujeme na event pomocí klíčového slova on a našeho libovolného názvu pro event. Naši hodnotu poté získáme v callback funkci.\cite{sveltehandbook,svelte}

\begin{prog}
// Parent.svelte
<script>
  import Child from './Child.svelte';

  function handleMessage(event) \{
    alert(event.detail.text);
  \}
</script>

<Child on:message=\{handleMessage\} />

// Child.svelte
<script>
  import \{ createEventDispatcher \} from 'svelte';

  const dispatch = createEventDispatcher();

  function sayHello() \{
    dispatch('message', \{
      text: 'Hello world!'
    \});
  \}
</script>

<button on:click=\{sayHello\}>Klikněte pro "Hello world!"</button>
\end{prog}

Zdroj zdrojového kódu: \cite{svelte}

\subsubsection{Životní cyklus}

Komponenty ve Svelte disponují životním cyklem, který začíná v momentě vytvoření komponenty a končí jejím zničením. 
Funkce onMount tvoří callback, který je zavolán po přidání komponenty do DOMu. Pokud chceme vykonat určité akce při zničení komponenty, můžeme toho dosáhnout dvěma způsoby. 
Prvním způsobem je vracení callback funkce v rámci onMount funkce. Druhou možnost představuje využití funkce onDestroy, která v argumentu přijímá callback funkci. 

Pro práci převážně s imperativními akcemi slouží zabudované funkce beforeUpdate a afterUpdate. 
V případě beforeUpdate funkce jde o callback, který se volá před aktualizací komponenty, tj. před prvním voláním onMount nebo po každé změně stavu. 
Oproti tomu, afterUpdate je callbackem, jenž Svelte vykoná po prvním zavolání onMount nebo po každé aktualizaci komponenty.\cite{sveltehandbook,svelte}

\subsubsection{State management}

Svelte poskytuje pestrou škálu API pro správu stavů aplikace v závislosti na rozsahu a složitosti ukládaných dat. 
Základním přístupem pro správu stavů je ukládání a manipulace se stavy v rámci stromu komponent. 
To zahranuje tvorbu reaktivních stavů a jejich distribuci ve stromě pomocí předávání vlastností, bindování či eventů. 

Další možnost state managementu představuje využití Context API, které umožňuje jednorázové uložení jakékoli hodnoty. 
Nasledně je možné získat tuto hodnotu i v rámci neincidentních komponent. Ukládání a získávání contextu umožňují funkce setContext a getContext.

Pro sofistikovanější práci se stavy slouží tzv. stores. V podstatě se jedná o globání úložiště stavů, které umožňuje uchovávat a získávat data. 
Store je jednoduše objekt s metodou subscribe, která umožní konzumentu dostat aktualizovaná data. 
Jednodušší variantu pro získání aktuálních dat představuje použití znaku \$ před názvem proměnné. Svelte nám poskytuje hned několik podob storu. 
To jsou jednak writable a readable stores, kde jediný rozdíl spočívá v možnosti aktualizace dat. 
Pro stavy, které jsou odvozeny z jiných stores, existuje tzv. derived store. V neposlední řadě nám Svelte povoluje vytvořit i vlastní store. 
% Store je možné používat i v klasických JS filech, např pomocí subscribe... 

Již zabudované globální úložiště můžeme jednoduše vytvořit pomocí metod writable, readable a derived. Writable požaduje jako argument počáteční hodnotu. 
Readable navíc jako druhý argument může přijímat funkci start, jež implementuje callbacky volající se při prvním a posledním subscribe.\cite{sveltehandbook,svelte,sveltestatemanagement}

\subsubsection{Routování}

Svelte nemá přímou podporu routování v aplikacích. Oficiální dokumentace uvádí jako oficiální knihovnu pro routování SvelteKit. 
Ve skutečnosti se jedná o framework nad Svelte, který poskytuje i další možnosti rozšíření webové aplikace. 
Dokumentace však doporučuje i jiné knihovny pro routování na základě odlišných přístupů. 
Konkrétně knihovny page.js, svelte-routing, svelte-navigator, svelte-spa-router nebo routify.\cite{svelte,svelteforbeginners}

Routování ve SvelteKit je implementováno pomocí file systému. Komponenta s názvem +page.svelte definuje stránku aplikace. 
Framework umožňuje pro opakující se uživatelská prostředí využít tzv. layouts. Jde o soubor, který aplikuje určité elementy (duplicitní kód) pro aktuální adresář komponent. 
Pro vykreslení obsahu na základě samotných komponent se využívá element slot. SvelteKit také umožnuje vytváření dynamických parametrů přímo v souborovém systému. 
Díky takovým cestám je možné tvořit např. individuální příspěvky na blogu. Pomocí +server.js můžeme definovat API routy (endpointy) aplikace. 
Chybové stránky vytváříme pomocí +error.svelte souborů.\cite{svelte,sveltekit}
% Martin by to hodil do šablonování :D

\subsubsection{Ekosystém}

I přesto, že Svelte používá stále více vývojářů, framework nedisponuje přiliš rozsáhlým ekosystémem. Hlavní rozšíření spočívá v použití rozšiřujícího frameworku SvelteKit a jazyka TypeScript. 
Vztah Svelte a SvelteKit můžeme definovat jako sourozenecký, kdy SvelteKit poskytuje adaptivní prostředí pro vývoj aplikace jakéhokoli rozsahu.

Dle \cite{sveltedailydev} neexistuje mnoho specifických knihoven přímo pro Svelte. 
Na druhou stranu je možné využít rozsáhlého ekosystému JavaScriptu, jelikož Svelte poskytuje přímou kontrolu nad DOM. 
V porovnání se specifickými knihovnami tento přístup však obvykle vyžaduje práci navíc. 
Problematické bývá využití knihoven, jež využívají API prohližeče.\cite{svelteheyreliable,sveltedailydev,sveltejslibs}
% Případně doplnit, jakým způsobem se řeší stylování v komponentách :)