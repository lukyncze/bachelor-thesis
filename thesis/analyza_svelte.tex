\subsection{Svelte}

Svelte je relativně novým open-source JavaScript frameworkem, za jejímž stvořením stojí vývojář Richard Harris. 
Framework se odlišuje tím, že kompiluje komponenty do čistého JavaScriptu. To vše ještě před tím, než uživatel navštíví webovou aplikaci v prohlížeči. 
Tato metoda poskytuje výhodu hlavně co se týče rychlosti oproti klasickým deklarativním frameworkům jako jsou např. React, Vue nebo Angular. 
Stejně jako tyto frameworky je Svelte určen k vývoji rychlého a kompaktního uživatelského rozhraní pro webové aplikace.

První verze byla představena ke konci roku 2016. Verze 3, jež byla vydána v dubnu 2019, přinesla vylepšení týkající se zjednodušení tvorby komponent. 
Mimo jiné tato verze hlavně představila vylepšení ve smyslu reaktivity. Po této verzi framework nabral na popularitě díky jeho jednoduchosti.
Verze 4 pak v roce 2023 představila pouze minimální změny, jež spočívají v údržbě a přípravách pro verzi nastávající.

Přestože Svelte nedisponuje rozsáhlým ekosystémem jako jiné JavaScriptové frameworky, získal si přízeň mnoha velkých společností. 
Mezi ně patří například firmy jako The New York Times, Avast, Rakuten a Razorpay.\cite{sveltemdn,svelte,sveltedevinterface}

% Bakalarka https://www.theseus.fi/bitstream/handle/10024/500643/Oksanen_Miikka.pdf
% Diplomka
% https://www.doria.fi/handle/10024/177433
% https://www.doria.fi/bitstream/handle/10024/177433/levlin_mattias.pdf?sequence=2&isAllowed=y
% Kniha
% https://www.syncfusion.com/succinctly-free-ebooks/svelte-succinctly

\subsubsection{Komponenty}

Podobně jako v Reactu, komponenty jsou základní stavební bloky Svelte. Komponentu tvoří HTML, CSS a JavaScript, kde vše patří do jednoho souboru s příponou .svelte. 
Všechny tři části komponenty jsou nepovinné. Logika komponenty musí být zapsána mezi párové script tagy. Následuje jeden nebo více značek pro definovaní šablony komponenty. 
V neposlední řadě kaskádové styly se zapisují mezi style tagy.

V rámci šablony Svelte umožňuje využívat logické bloky pro podmíněné vykreslování nebe také iterace přes pole hodnot (list). 
Zabudovaná je i podpora manipulace s asynchronním JavaScriptem - promises.\cite{svelte}

\subsubsection{Reaktivita}

Srdcem Svelte jsou reaktivní stavy komponenty, které jednoduše definujeme jako proměnné v JavaScriptu. Jejich hodnotu aktualizuje JavaScript funkce pomocí přidělování nových hodnot. 
Kupříkladu stav o datovém typu pole tudíž nelze aktualizovat pouze pomocí metody push či splice. Je nutné využít jiné intuitivní řešení pomocí přidělení nové hodnoty.
O všechno ostatní se pak ale doslova postará sám Svelte v pozadí. Svelte aktualizuje DOM při každé změně stavu komponenty. 

Mezi specifické funkce Svelte patří reaktivní deklarace, které se starají o aktualizaci stavů na základě stavů jiných. 
Další zabudovanou funkcí jsou tzv. reactive statements, jež umožní definovat akce, které se mají vykonat reaktivně -- jako reakce na nějaký výrok.\cite{sveltehandbook,svelte}

\subsubsection{Předávání vlastností}

Pro komunikaci mezi komponentami slouží mechanismus předávání vlastností. 
V rodičovské komponentě je nutné komponentě říci, jakou hodnotu chceme předat a do jaké proměnné ji chceme uložit v child komponentě. 
Pak v child komponentě vytvoříme stejnojmennou vlastnost s klíčovým slovem export.

Pokud chceme předávat vlastnosti parent komponentě, je třeba vytvořit vlastnost již na parent komponentě. 
Následně ji předat child komponentě a v rodičovské komponentě přidat před předání vlastnosti do komponenty klíčové slovo bind.\cite{svelte}

\subsubsection{Eventy}

Svelte má velice jednoduché API pro práci s DOM eventy. Stačí použít direktivu on na HTML elementu, která vyžaduje název eventu a callback funkci.

Vývojáři také přišli s možností odesílání a přijímání eventů pro komponenty. 
V child komponentě je třeba mí nějaký DOM event handler, na který chceme reagovat v parent komponentě. 
Poté je nutné využít zabudovanou metodu createEventDispatcher, které předáme potřebné parametry -- náš libovolný název pro event komponenty a hodnotu. 
V rodičovské komponentě pak reagujeme na event pomocí klíčového slova on a našeho libovolného názvu pro event. Naši hodnotu poté získáme v callback funkci.\cite{sveltehandbook,svelte}

\subsubsection{Životní cyklus}

Komponenty ve Svelte disponují životním cyklem, který začíná v momentě vytvoření komponenty a končí jejím zničením. 
Funkce onMount tvoří callback, který je zavolán po přidání komponenty do DOMu. Pokud chceme vykonat určité akce při zničení komponenty, můžeme toho dosáhnout dvěma způsoby. 
Prvním způsobem je vracení callback funkce v rámci onMount funkce. Druhou možnost představuje využití funkce onDestroy, která v argumentu přijímá callback funkci. 

Pro práci převážně s imperativními akcemi slouží zabudované funkce beforeUpdate a afterUpdate. 
V případě beforeUpdate funkce jde o callback, který se volá před aktualizací komponenty, tj. před prvním voláním onMount nebo po každé změně stavu. 
Oproti tomu, afterUpdate je callbackem, jenž Svelte vykoná po prvním zavolání onMount nebo po každé aktualizaci komponenty.\cite{sveltehandbook,svelte}

\subsubsection{State management}

Svelte poskytuje pestrou škálu API pro správu stavů aplikace v závislosti na rozsahu a složitosti ukládaných dat. 
Základním přístupem pro správu stavů je ukládání a manipulace se stavy v rámci stromu komponent. 
To zahranuje tvorbu reaktivních stavů a jejich distribuci ve stromě pomocí předávání vlastností, bindování či eventů. 

Pro sofistikovanější práci se statem aplikace je možné využít Context API, jenž umožňuje pracovat se stavy v rámci neincidentních komponent. 
Ukládání a získání contextu umožňují funkce setContext a getContext.

Třetí zabudované API pro manipulaci se stavy představují tzv. stores. V podstatě se jedná o globání úložiště stavů, které umožňuje uchovávat a získávat data. 
Store je jednoduše objekt s metodou subscribe, která umožní komzumentu dostat aktualizovaná data. Svelte nám poskytuje hned několik podob storu. 
To jsou jednak writable a readable stores, kde jediný rozdíl spočívá v možnosti aktualizace dat. Pro stavy, které jsou odvozeny z jiných stores, existuje tzv. derived store. 
V neposlední řadě nám Svelte povoluje vytvořit i vlastní store. 

Již zabudované globální úložiště můžeme jednoduše vytvořit pomocí metod writable, readable a derived. Writable požaduje jako argument počáteční hodnotu. 
Readable navíc jako druhý argument potřebuje funkci start, jež implementuje callbacky volající se při prvním a poslední subscribe.\cite{sveltehandbook,svelte,sveltejoyofcode}

\subsubsection{Routování}
\subsubsection{Ekosystém}