\section{Webové aplikace}

Webová aplikace je typ počítačového software, který je uživateli sdílen prostředníctvím internetu. 
Uživatel se k aplikaci dostane pomocí webového prohlížeče a nemusí si ji instalovat na svůj počítač. 
Aplikace jsou jakýmsi prostředníkem mezi uživatelem a serverem, kde se nachází data a logika aplikace.\cite{codeacademywebapp}

\begin{flushleft}
  \textbf{Frontend}
\end{flushleft}

% TODO

\begin{flushleft}
  \textbf{Backend}
\end{flushleft}

% TODO

\begin{flushleft}
  \textbf{Single Page Application}
\end{flushleft}

Pod pojmem Single Page Application (SPA) rozumíme webové aplikace, které se skládají z jediné stránky a také jednotlivých částí aplikace. 
Obsah stránek bývá dynamicky aktualizován pomocí JavaScriptu. Tento přístup umožňuje aktualizaci stránek bez nutnosti obnovování celé stránky. 
V praxi to znamená, že při jakékoli akci uživatele se aktualizuje pouze obsah stránky, nikoli celá stránka. 
Mezi výhody tohoto přístupu patří rychlejší odezva aplikace, méně dotazů na server a také lepší uživatelský zážitek.\cite{spausingangularjs}

\begin{flushleft}
  \textbf{Framework}
\end{flushleft}

% TODO
% https://www.codecademy.com/resources/blog/what-is-a-framework/