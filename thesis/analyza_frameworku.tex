\section{Analýza frameworků}

Text druhé kapitoly.

\begin{citemize}
	\item odůvodnit výběr frameworků:
	\begin{citemize}
		\item \iadresa{https://survey.stackoverflow.co/2023/}
	\end{citemize}

	\item analýza frameworků:
	\begin{citemize}
		\item co to je
		\item stručná historie
		\item kdo jej vytvořil
		\item kdo jej používá
		\item jak se používá
		\item popularita ve společnosti
	\end{citemize}

	\item srování frameworků dle:
	\begin{cenumerate}
		\item Component-Based architektura (HTML, CSS, JS/TS)
		\item Šablony
		\item Předávání vlastností (props)
		\item Správa stavů
		\item State management
		\item Routování
		\item DOM
		\item Server-Side Rendering
		\item Kompilace zdrojových souborů
		\item Reaktivita
		\item Ekosystém
	\end{cenumerate}
\end{citemize}


\subsection{React}

Pod pojmem React rozumíme open-source JavaScript framework, který vyvinula a~dále vyvíjí společnost Meta (dříve Facebook). 
Podle \cite{reactbanks} jde spíše o knihovnu funkcí, než-li o komplexní nástroj pro tvorbu webových aplikací (framework). 
Tato technologie se používá pro vývoj interaktivních uživatelských rozhraní a~webových aplikací.\cite{reacthubspot}

První kořeny Reactu sahají až do roku 2010, kdy tehdejší společnost Facebook přidala novou technologii XHP do PHP. 
Jde o možnost znovu použít určitý blok kódu, stejného principu posléze využívá i React. Následně Jordan Walke vytvořil FaxJS, jenž byl prvním prototypem Reactu.
O rok později byl přejmenován na React a~začal jej využívat Facebook. 
V roce 2013 byl na konferenci JS ConfUS představen jeho autorem široké veřejnosti a~stal se open-source.

Od roku 2014 vývojáři představují nespočet vylepšení samotné knihovny, stejně jako spoustu rozšíření pro zlepšení vývojových procesů. 
Kolem roku 2015 postupně React nabývá na popularitě i celkové stabilitě. Následně je představen také React Native, což je framework pro vývoj nativních aplikací.
V dnešní době je React využíván společnostmi každého rozsahu po celém světě. 
Z těch největších jde například o Metu, Uber, Twitter a~Airbnb.\cite{reactbanks,reactrisingstack}

\subsubsection{Komponenty}

Hlavním stavebním kamenem Reactu jsou komponenty, jež představují nezávislé, vnořitelné a opakovaně použitelné bloky kódu. 
Komponentu v Reactu tvoří JavaScript funkce a HTML šablona. Validně seskládané komponenty poté tvoří webovou aplikaci.
V Reactu se můžeme setkat s funkčními a třídními komponentami. Vytváření třídních komponent oficiální dokumentace nedoporučuje.
Pro komunikaci mezi komponentami se používá předávání vlastností (props), přes které je možné předávat jakékoli validní hodnoty z JavaScriptu.
Výstup komponent tvoří elementy ve formě JSX. Tyto elementy obsahují informace o vzhledu a funkcionalitě dané komponenty.\cite{reactbanks,react}

\subsubsection{JSX}

Název JSX kombinuje zkratku jazyka JavaScript -- JS a počáteční písmeno ze zkratky XML. 
Konkrétně jde o syntaktické rozšíření, které vývojářům umožňuje tvořit React elementy pomocí hypertextového značkovacího jazyku přímo v JavaScriptu. 
V rámci JSX pak je možné dynamicky vykreslovat obsah na základě logiky definované pomocí JavaScriptových hodnot.
Při kompilaci se JSX překládá do JavaScriptu pomocí nástroje Babel.\cite{reactbanks,react}


\subsection{Svelte}

\subsection{Vue}

\subsection{Porovnání}

\begin{citemize}
	\item co zjistím na první pohled, platforma,
	\item jako bych si četl reklamu ...,
	\item prvotní srovnání.
\end{citemize}
