\section{Analýza frameworků}

V této části práce se zabýváme analýzou frameworků Angular, React, Svelte a~Vue. Nejdříve analyzujeme každý framework samostatně. 
Zaměříme se především na klíčové koncepty frameworků, stavební bloky, správu stavů, předávání vlastností. 
Dále rozebereme životní cykly, state management, routování a~ekosystém jednolivých technologií. 
Kapitolu zakončíme porovnáním analyzovaných frameworků.

Výběr porovnávaných technologiíí byl proveden na základě analýzy výsledků z Developer Survey 2023 od Stack Overflow. 
Rozhodující byla sekce \emph{Web frameworks and technologies}, ve které byly hodnoceny webové technologie dle jejich žádanosti na trhu a~obdivem mezi vývojářskou komunitou. 
Tato strategie zajistila, že vybrané frameworky odpovídají současným trendům a~potřebám webového vývoje, ale také vycházejí z uznání a~podpory komunity.\cite{stackoverflow, developersurvey}

\subsection{Angular}

\subsubsection{Komponenty}

\subsubsection{Správa stavů}

\subsubsection{Předávání vlastností}

\subsubsection{Služby a direktivy}
% případně Pipes

\subsubsection{Životní cyklus}

\subsubsection{State management}

\subsubsection{Routování}

\subsubsection{Ekosystém}
\subsection{React}

Pod pojmem React rozumíme open-source JavaScript framework, který vytvořila a~dále rozvíjí společnost Meta (dříve Facebook). 
Podle \cite{reactbanks} jde spíše o~knihovnu funkcí, než-li o~komplexní nástroj pro tvorbu webových aplikací (framework). 
Tato technologie se používá pro vývoj interaktivních uživatelských rozhraní a~webových aplikací.\cite{reacthubspot}

\begin{figure}[htb]
	\centering
		\includegraphics[width=.3\textwidth]{images/react-logo.png}
	\caption[React logo]{React logo \cite{react}}
	\label{fig:reactlogo}
\end{figure}

První kořeny Reactu sahají až do roku 2010, kdy tehdejší společnost Facebook přidala novou technologii XHP do PHP. 
Jde o~možnost znovu použít určitý blok kódu, stejného principu posléze využívá i~React. Následně Jordan Walke vytvořil FaxJS, jenž byl prvním prototypem Reactu.
O~rok později byl přejmenován na React a~začal jej využívat Facebook. 
V~roce 2013 byl na konferenci JS ConfUS představen široké veřejnosti a~stal se open-source.

Od roku 2014 vývojáři představují nespočet vylepšení samotné knihovny, stejně jako spoustu rozšíření pro zlepšení vývojových procesů. 
Kolem roku 2015 postupně React nabývá na popularitě i~celkové stabilitě. Následně je představen také React Native, což je framework pro vývoj nativních aplikací.
React používá široká škála společností, od malých startupů až po velké nadnárodní korporace. 
Z~těch největších jde například o~Metu, Uber, Twitter a~Airbnb.\cite{reactbanks,reactrisingstack}

\subsubsection{Komponenty}

Hlavním stavebním kamenem Reactu jsou komponenty, jež představují nezávislé, vnořitelné a~opakovaně použitelné bloky kódu. 
Komponentu v~Reactu tvoří JavaScript funkce a~HTML šablona. Validně seskládané komponenty poté tvoří webovou aplikaci.
V~Reactu se můžeme setkat s~funkčními a~třídními komponentami. Vytváření třídních komponent je na ústupu a~oficiální dokumentace je rovněž nedoporučuje. 
Výstup komponent tvoří elementy ve formě JSX. Tyto elementy obsahují informace o~vzhledu a~funkcionalitě dané komponenty.

Pro komunikaci mezi komponentami se používá předávání vlastností (props), přes které je možné předávat hodnoty jakýchkoli datových typů. 
Vlastnost vnořené komponentě předáme stejně jako atribut HTML elementu. 
Pro předání hodnoty do rodičovské komponenty slouží tzv. callback funkce, které se volají ve vnořené komponentě, ale modifikují vlastnosti rodičovské komponenty.\cite{reactbanks,react}

\begin{prog}
import React from 'react';

function ParentComponent() \{
  const someProps = \{color: 'cervena'\};

  return (
    <div>
      <ChildComponent color=\{'cervena'\} />
      <ChildComponent color=\{someProps.color\} />
      <ChildComponent \{...someProps\} />
    </div>
  );
\}

function ChildComponent(\{color\}) \{
  return (
    <div className=\{color\}></div>
  );
\}
\end{prog}

\subsubsection{JSX}

Název JSX kombinuje zkratku jazyka JavaScript -- JS a~počáteční písmeno ze zkratky XML. 
Konkrétně jde o~syntaktické rozšíření, které vývojářům umožňuje tvořit React elementy pomocí hypertextového značkovacího jazyku přímo v~JavaScriptu. 
V~rámci JSX pak je možné dynamicky vykreslovat obsah na základě logiky definované pomocí JS hodnot.
Při kompilaci se JSX překládá do JavaScriptu pomocí nástroje Babel.\cite{reactbanks,react}

\begin{prog}
import React from 'react';
import ChildComponent from './ChildComponent';

function MyComponent() \{
  const loaded = true;

  return (
    <div>
      \{loaded ? 
        <ChildComponent color=\{'cervena'\} width=\{100\} heigth=\{100\} /> 
        : 'Načítání ...'\}
    </div>
  );
\}
\end{prog}

\subsubsection{Správa stavů}

Stav lze definovat jako lokální vnitřní vlastnost či proměnnou dané komponenty, jež představuje základní mechanismus pro uchovávání a~aktualizaci dat. 
Pro aktualizaci komponenty je tedy nutné stav změnit. React pak na tuto skutečnost zareaguje a~vyvolá tzv. re-render neboli překreslení komponenty s~novými daty.

Za účelem ukládání stavu se využívá hook (funkce) useState. Ten poskytuje stavovou proměnnou, přes kterou se dostaneme k~aktuálnímu stavu. 
Dále useState poskytuje state setter funkci, díky které můžeme stav aktualizovat. Jediný argument useState definuje počáteční hodnotu daného stavu.\cite{reactitnetwork,react}

\begin{prog}
import React, \{ useState \} from 'react';

function App() \{
  const [count, setCount] = useState(0);

  return (
    <button onClick=\{() => setCount(count + 1)\}>
      Klikli jste na tlačítko \{count\}x.
    </button>
  );
\}
\end{prog}

\subsubsection{Hooky}

Specifickou funkcionalitou pro React jsou tzv. hooky, které byly do Reactu přidány až ve verzi 16.8.0.\cite{reactgithub} 
Hook je definován jako funkce, která obohacuje komponenty pomocí předdefinovaných funkcionalit. Jedním z~nejpoužívanejších hooků je useState. 
Vývojáři mohou používat již zabudované hooky, nebo si vytvářet své vlastní s~pomocí předdefinovaných hooků. 
Mezi zabudované hooky patří např. useEffect, useMemo, useCallback, useRef, useContext.\cite{react}

\begin{prog}
import \{ useEffect \} from 'react';

useEffect(() => \{
  // obvykle kód určený pro nastavení (setup)

  return () => \{
    // kód pro úklid prostředků
  \};
\}, [
  // seznam závislostí, na jejichž změnu má efekt reagovat
]);
\end{prog}

\subsubsection{Životní cyklus}

Životní cyklus komponenty je sekvence událostí, jež nastanou mezi vytvořením a~zničením komponenty. 
Ve třídních komponentách existovaly speciální metody, tzv. lifecycle metody, starající se o~provedení určité části kódu při daném okamžiku v~životě komponenty. 
Nyní React disponuje pár hooky, které umožňují provádět side-effects podobně jako lifecycle metody.

O~momentu, kdy je komponenta přidána na stránku, mluvíme jako o~namontování (mount). Při změně stavu či obdržení nových parametrů hovoříme o~aktualizaci (update) komponenty. 
Po odstranění komponenty z~DOM proběhne odmontování (unmount).\cite{reactlifecycle, react}

\subsubsection{State management}

Základní práce se stavy spočívá v~lokálních stavech komponent a~následným předáváním stavu do potomků či rodičů. 
V~případě, že potřebujeme sdílet stav mezi komponentami, měli bychom zvážit odlišné řešení. React sám o~sobě disponuje pouze základním řešením, kterému říká Context API. 
Context umožňuje sdílet data celému podstromu dané komponenty. 
To se může hodit například při vytváření barevných módů aplikace, sdílení informace o~přihlášeném uživateli, anebo routování.\cite{react}

Správa stavů v~komplexních aplikacích se stává výzvou. Problémy začínají při potřebě sdílení identických dat mezi větším množstvím konzumentů. 
Existuje však mnoho knihoven třetích stran, které vývojáři využívají pro usnadnění manipulace se stavy. 
Společné cíle state management knihoven spočívají v~ukládání a~získávání globálního stavu, jednodušší správě stavů a~rozšiřitelnosti aplikace.
Mezi tyto knihovny patří kupříkladu Redux, MobX, Recoil nebo Jotai.\cite{statemanagementreact,reactstatemanagement}

\subsubsection{Routování}

React nemá žádný nativní standard pro routování. Podle \cite{reactbanks} je React Router jedním z~nejvíce populárních řešení pro React. 
Knihovna React Router umožňuje nastavení jednotlivých cest aplikace. Zajišťuje tedy routování na straně klienta.

Instanci routeru vytvoříme například pomocí funkce createBrowserRouter, která přijímá pole definovaných cest aplikace. 
Další možností je vytvoření cest pomocí funkce createRoutesFromElements. Router následně předáme do komponenty RouterProvider. 
K vykreslení požadované komponenty, která je spojena s~danou cestou, slouží komponenta Outlet.\cite{reactbanks,reactrouter}

\subsubsection{Ekosystém}

Tento framework sám o~sobě není úplně komplexním nástrojem. I~přesto se stále vyvíjí a~jeho ekosystém se neustále rozrůstá. 
Na druhou stranu React disponuje velmi diverzifikovaným ekosystémem knihoven, který nabízí bohatý výběr nástrojů pro různé aspekty vývoje. 
Knihovny jsou převážně zaměřené na stylování, tvorbu tabulek, formulářů, grafů či grafických animací, správu stavů, routování, dotazování na API. 
Nechybí ani dokumentační knihovny, vývojářské rozšíření pro prohlížeče, striktní typování, překlady, testovací balíčky. 
V~neposlední řadě pro React existují nadstavby ve formě frameworků, které poskytují komplexnější nástroje pro produkční aplikace.\cite{awesomereact,builderreacteco,react}
\subsection{Svelte}

Svelte je relativně novým open-source JavaScript frameworkem, za jejímž stvořením stojí vývojář Richard Harris. 
Framework se odlišuje tím, že kompiluje komponenty do čistého JavaScriptu. To vše ještě před tím, než uživatel navštíví webovou aplikaci v prohlížeči. 
Tato metoda poskytuje výhodu hlavně co se týče rychlosti oproti klasickým deklarativním frameworkům jako jsou např. React, Vue nebo Angular. 
Stejně jako tyto frameworky je Svelte určen k vývoji rychlého a kompaktního uživatelského rozhraní pro webové aplikace.

První verze byla představena ke konci roku 2016. Verze 3, jež byla vydána v dubnu 2019, přinesla vylepšení týkající se zjednodušení tvorby komponent. 
Mimo jiné tato verze hlavně představila vylepšení ve smyslu reaktivity. Po této verzi framework nabral na popularitě díky jeho jednoduchosti.
Verze 4 pak v roce 2023 představila pouze minimální změny, jež spočívají v údržbě a přípravách pro verzi nastávající.

Přestože Svelte nedisponuje rozsáhlým ekosystémem jako jiné JavaScriptové frameworky, získal si přízeň mnoha velkých společností. 
Mezi ně patří například firmy jako The New York Times, Avast, Rakuten a Razorpay.\cite{sveltemdn,svelte,sveltedevinterface}

% Svelte CZ https://www.sveltejs.cz/
% Bakalarka https://www.theseus.fi/bitstream/handle/10024/500643/Oksanen_Miikka.pdf
% Diplomka
% https://www.doria.fi/handle/10024/177433
% https://www.doria.fi/bitstream/handle/10024/177433/levlin_mattias.pdf?sequence=2&isAllowed=y
% Kniha
% https://www.syncfusion.com/succinctly-free-ebooks/svelte-succinctly
% Joy of code
% https://joyofcode.xyz/svelte-for-beginners

\subsubsection{Komponenty}

Podobně jako v Reactu, komponenty jsou základní stavební bloky Svelte. Komponentu tvoří HTML, CSS a JavaScript, kde vše patří do jednoho souboru s příponou .svelte. 
Všechny tři části komponenty jsou nepovinné. Logika komponenty musí být zapsána mezi párové script tagy. Následuje jeden nebo více značek pro definovaní šablony komponenty. 
V neposlední řadě kaskádové styly se zapisují mezi style tagy.

V rámci šablony Svelte umožňuje využívat logické bloky pro podmíněné vykreslování nebe také iterace přes pole hodnot (list). 
Zabudovaná je i podpora manipulace s asynchronním JavaScriptem - promises.\cite{svelte}

\subsubsection{Reaktivita}

Srdcem Svelte jsou reaktivní stavy komponenty, které jednoduše definujeme jako proměnné v JavaScriptu. Jejich hodnotu aktualizuje JavaScript funkce pomocí přidělování nových hodnot. 
Kupříkladu stav o datovém typu pole tudíž nelze aktualizovat pouze pomocí metody push či splice. Je nutné využít jiné intuitivní řešení pomocí přidělení nové hodnoty.
O všechno ostatní se pak ale doslova postará sám Svelte v pozadí. Svelte aktualizuje DOM při každé změně stavu komponenty. 

Mezi specifické funkce Svelte patří reaktivní deklarace, které se starají o aktualizaci stavů na základě stavů jiných. 
Další zabudovanou funkcí jsou tzv. reactive statements, jež umožní definovat akce, které se mají vykonat reaktivně -- jako reakce na nějaký výrok.\cite{sveltehandbook,svelte}

\subsubsection{Vlastnosti}
\subsubsection{Eventy}
\subsubsection{Životní cyklus}
\subsubsection{State management}
\subsubsection{Routování}
\subsubsection{Ekosystém}
\subsection{Vue}

% Kniha od O'Reilly
% Fullstack Vue.js book
% https://vuejs.org/
% https://www.w3schools.com/vue/
% https://developer.mozilla.org/en-US/docs/Learn/Tools_and_testing/Client-side_JavaScript_frameworks/Vue_getting_started
% https://www.tutorialspoint.com/vuejs/vuejs_overview.htm
% https://www.itnetwork.cz/javascript/vuejs/uvod-do-vuejs-a-prvni-aplikace
% https://worldline.github.io/vuejs-training/
% https://www.rascasone.com/cs/blog/co-je-framework-vuejs

% https://flexiple.com/vue/deep-dive
% https://madushaprasad21.medium.com/vue-js-history-1a6b8567198f

Vue dostalo svůj název díky anglickému slovu view. Jedná se o deklarativní JavaScriptový open-source framework. 
Je určen efektivní tvorbě jak jednoduchých, tak i komplexních uživatelských rozhraní na webu. 
Autor, Evan You, se při tvorbě Vue inspiroval frameworkem AngularJS, který měl velmi strmou učící se křivku. Vue tedy mělo být lehké, přizpůsobivé a snadné k naučení.

Framework byl vytvořen roku 2013, uvolněn do světa až o rok později. Od té doby byly vydány pouze 3 majoritní verze, avšak ty přinesly mnoho změn. 
Vue umožnuje svobodnou volbu ve formě API, které definuje styl komponent -- Options a Composition API. 
Options API můžeme přirovnat k objektovému přístupu, v porovnání s Composition API, jež využívá funkcionální přístup. 
Podle \cite{vue} Composition API umožňuje lepší flexibilitu a silnější návrhové vzory pro organizaci a znovupoužitelnost kódu. 

\subsubsection{Komponenty}

Single-File Component (SFC)

\subsubsection{Reaktivita}
\subsubsection{Předávání vlastností}
\subsubsection{Eventy???}
\subsubsection{Životní cyklus}
\subsubsection{State management}
% Vuex - https://vuex.vuejs.org/
\subsubsection{Routování}
% Vue Router - https://router.vuejs.org/
\subsubsection{Ekosystém}


\subsection{Porovnání analyzovaných frameworků}

Analyzované technologie využívají podobné základní koncepty, což vývojáři ocení hlavně v momentě, kdy se chtějí naučit jiný framework. 
Každý z frameworků však disponuje odlišnou syntaxí, přístupy, možnostmi a~API. Díky tomu každá technologie vyniká v jiných oblastech.

Framework Angular je velice robustním frameworkem s mnoha funkcionalitami, které jsou zabudovány uvnitř balíků frameworku. 
Díky tomu vývojáři disponují téměř všemi základními nástroji pro vývoj webových aplikací. 
Mezi další výhody můžeme zařadit také přechod z projektu na projekt jiný, jelikož ve většině případů projekty využívají stejné nástroje a~konvence. 
Nevýhodou Angularu je jednoznačně složitější křivka učení, čemuž nepřispívá ani to, že v Angular projektech často využíváme knihovnu RxJS. 
Do nevýhod bychom také mohli zařadit velikost výsledné aplikace a~delší syntax.

Oproti tomu React je populární díky své jednoduchosti a~flexibilitě. 
Nespornou výhodu představuje to, že technologie se dá použít při vývoji jak webových, tak i mobilních aplikací. 
Vývojáři mají možnost vybrat si z mnoha knihoven a~nástrojů, které využijí k vývoji. 
Toto může být bráno i negativně, protože vývojáři musí mít přehled o balíčcích a~nástrojích, které mohou použít. 
V důsledku pak může být vývoj aplikace složitější a~vývojář rovněž nemusí využít vhodné postupy. 
Na druhou stranu pro React je dostupná pestrá škála návodů a~tutoriálů, které mohou být přínosné při osvojování nových znalostí.

Svelte je frameworkem nejmladším, zároveň jej však můžeme považovat za nejvíce inovativní. 
Technologie je vhodná pro začátečníky, protože má pouze minimum boilerplate kódu. 
Programátoři rozhodně ocení kompilaci zdrojových kódů do nativního JavaScriptu již při sestavení aplikace. 
Jako výhodu rovněž můžeme uvést pokročilejší optimalizace, které Svelte nabízí. 
Nevýhodou pak může být menší komunita kolem frameworku, což v důsledku může znamenat menší množství dostupných knihoven a~nástrojů. 
Svelte však umožňuje využití JS knihoven, které přímo ovlivňují DOM, což do jisté míry kompenzuje předchozí nevýhody.

V neposlední řadě Vue umožňuje vývojářům využívat jak objektový, tak i funkcionální přístup při tvorbě komponent. 
Framework, podobně jako Svelte, vyniká v oblasti optimalizace a~výkonu. Rozsáhlá komunita Vue se aktivně podílí na vývoji nemalého množství knihoven a~nástrojů. 
Kvůli tomu, že velká část komunity pochází z Číny, mohou být jak technické dokumentace, tak i online zdroje primárně v čínštině. 
To pak znesnadňuje hledání informací vývojářům v angličtině. Celkově ale můžeme říci, že Vue je frameworkem velice vyváženým. 
Díky inspiracím z jiných frameworků nabízí mnoho propracovaných možností pro programování webových aplikací.