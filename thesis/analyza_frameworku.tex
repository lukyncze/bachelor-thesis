\section{Analýza frameworků}

Text druhé kapitoly.

\begin{citemize}
	\item odůvodnit výběr frameworků:
	\begin{citemize}
		\item \iadresa{https://survey.stackoverflow.co/2023/}
	\end{citemize}

	\item analýza frameworků:
	\begin{citemize}
		\item co to je
		\item stručná historie
		\item kdo jej vytvořil
		\item kdo jej používá
		\item jak se používá
		\item popularita ve společnosti
	\end{citemize}

	\item srování frameworků dle:
	\begin{cenumerate}
		\item Component-Based architektura (HTML, CSS, JS/TS)
		\item Šablony
		\item Předávání vlastností (props)
		\item Správa stavů
		\item Life-cycle
		\item State management
		\item Reaktivita ???
		\item DOM
		\item Routování
		\item Server-Side Rendering
		\item Kompilace zdrojových souborů
		\item Ekosystém
	\end{cenumerate}
\end{citemize}


\subsection{React}

Pod pojmem React rozumíme open-source JavaScript framework, který vyvinula a~dále vyvíjí společnost Meta (dříve Facebook). 
Podle \cite{reactbanks} jde spíše o knihovnu funkcí, než-li o komplexní nástroj pro tvorbu webových aplikací (framework). 
Tato technologie se používá pro vývoj interaktivních uživatelských rozhraní a~webových aplikací.\cite{reacthubspot}

První kořeny Reactu sahají až do roku 2010, kdy tehdejší společnost Facebook přidala novou technologii XHP do PHP. 
Jde o možnost znovu použít určitý blok kódu, stejného principu posléze využívá i React. Následně Jordan Walke vytvořil FaxJS, jenž byl prvním prototypem Reactu.
O rok později byl přejmenován na React a~začal jej využívat Facebook. 
V roce 2013 byl na konferenci JS ConfUS představen široké veřejnosti a~stal se open-source.

Od roku 2014 vývojáři představují nespočet vylepšení samotné knihovny, stejně jako spoustu rozšíření pro zlepšení vývojových procesů. 
Kolem roku 2015 postupně React nabývá na popularitě i celkové stabilitě. Následně je představen také React Native, což je framework pro vývoj nativních aplikací.
V dnešní době je React využíván společnostmi každého rozsahu po celém světě. 
Z těch největších jde například o Metu, Uber, Twitter a~Airbnb.\cite{reactbanks,reactrisingstack}

\subsubsection{Komponenty}

Hlavním stavebním kamenem Reactu jsou komponenty, jež představují nezávislé, vnořitelné a~opakovaně použitelné bloky kódu. 
Komponentu v Reactu tvoří JavaScript funkce a~HTML šablona. Validně seskládané komponenty poté tvoří webovou aplikaci.
V Reactu se můžeme setkat s funkčními a~třídními komponentami. Vytváření třídních komponent oficiální dokumentace nedoporučuje.
Pro komunikaci mezi komponentami se používá předávání vlastností (props), přes které je možné předávat hodnoty jakýchkoli datových typů.
Výstup komponent tvoří elementy ve formě JSX. Tyto elementy obsahují informace o vzhledu a~funkcionalitě dané komponenty.\cite{reactbanks,react}

\subsubsection{JSX}

Název JSX kombinuje zkratku jazyka JavaScript -- JS a~počáteční písmeno ze zkratky XML. 
Konkrétně jde o syntaktické rozšíření, které vývojářům umožňuje tvořit React elementy pomocí hypertextového značkovacího jazyku přímo v JavaScriptu. 
V rámci JSX pak je možné dynamicky vykreslovat obsah na základě logiky definované pomocí JavaScriptových hodnot.
Při kompilaci se JSX překládá do JavaScriptu pomocí nástroje Babel.\cite{reactbanks,react}

\subsubsection{Správa stavů}

Stav lze definovat jako lokální vnitřní vlastnost či proměnnou dané komponenty, jež představuje základní mechanismus pro uchovávání a~aktualizaci dat. 
Pro aktualizaci komponenty je tedy nutné stav změnit. React pak na tuto skutečnost zareaguje a~vyvolá tzv. re-render neboli překreslení komponenty s novými daty.

Za účelem ukládání stavu se využívá hook (funkce) useState. Ten poskytuje stavovou proměnnou, přes kterou se dostaneme k aktuálnímu stavu. 
Dále useState poskytuje state setter funkci, díky které můžeme stav aktualizovat. Jediný argument useState definuje počátační hodnotu daného stavu.\cite{reactitnetwork,react}

\subsubsection{Hooky}

Specifickou funkcionalitou pro React jsou tzv. hooky, které byly do Reactu přidány až ve verzi 16.8.0.\cite{reactgithub} 
Hook je definován jako funkce, která obohacuje komponenty pomocí předdefinovaných funkcionalit. Jedním z nejpoužívanejších hooků je useState. 
Vývojáři mohou používat již zabudované hooky, nebo si vytvářet své vlastní s pomocí předdefinovaných hooků. 
Mezi zabudované hooky patří např. useEffect, useMemo, useCallback, useRef, useContext.\cite{react}

\subsubsection{Životní cyklus}

Životní cyklus komponenty je sekvence událostí, jež nastanou mezi vytvořením a zničením komponenty. 
Ve třídních komponentách existovaly speciální metody, tzv. lifecycle metody, starající se o provedení určité části kódu při daném okamžiku v životě komponenty. 
Nyní React disponuje pár hooky, které umožňují provádět side-effects podobně jako lifecycle metody.

O momentu, kdy je komponenta přidána na obrazovku, mluvíme jako o namontování (mount) komponenty. Při změně stavu či obdržení nových parametrů hovoříme o aktualizaci (update) komponenty. 
A v neposlední řadě okamžik, kdy je komponenta odstraněna z obrazovky, nazýváme odmontování (unmount) komponenty.\cite{reactlifecycle, react}

\subsubsection{State management}

Základní práce se stavy spočívá v lokálních stavech komponent a následným předáváním stavu do potomků či rodičů. 
V případě, že potřebujeme sdílet stav mezi komponentami, měli bychom zvážit odlišné řešení. React sám o sobě disponuje pouze základním řešením, kterému říká Context API. 
Context umožňuje sdílet data celému podstromu dané komponenty. 
To se může hodit například při vytváření barevných módů aplikace, sdílení informace o přihlášeném uživateli, anebo routování.\cite{react}

Správa stavů v komplexních aplikacích se stává výzvou. Problémy začínají při potřebě sdílení identických dat mezi větším množstvím konzumentů. 
Existuje však mnoho knihoven třetích stran, které vývojáři využívají pro usnadnění manipulace se stavy. 
Společné cíle state management knihoven spočívají v ukládání a získávání globálního stavu, jednodušší správě stavů a rozšiřitelnosti aplikace.
Mezi tyto knihovny patří kupříkladu Redux, MobX, Recoil nebo Jotai.\cite{statemanagementreact,reactstatemanagement}

\subsubsection{Routování}

\iadresa{https://reactrouter.com/en/main}

\subsubsection{Ekosystém}


% https://www.simplilearn.com/tutorials/reactjs-tutorial/what-is-reactjs

\subsection{Svelte}

\subsection{Vue}

\subsection{Porovnání}

\begin{citemize}
	\item co zjistím na první pohled, platforma,
	\item jako bych si četl reklamu ...,
	\item prvotní srovnání.
\end{citemize}
