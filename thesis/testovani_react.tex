\subsubsection{React}

\begin{flushleft}
  \textbf{Instalace projektu}
\end{flushleft}

\begin{flushleft}
  \textbf{Správa stavů, předávání vlastností}
\end{flushleft}

Při implementaci jednoduchého čítače začneme tím, že vytvoříme Counter komponentu. Ta bude mít stav count a setter setCount pro tento stav.

Dále vytvoříme komponentu Button kvůli principu DRY a celkově znovupoužitelnosti kódu. 
Typ ButtonProps obsahuje vlastnosti, které můžeme tlačítku předat -- className, onClick a children. 
Díky tomu, že typ rozšiřuje ButtonHTMLAttributes<HTMLButtonElement>, můžeme předat do komponenty i další běžné atributy HTML tlačítek (např. type, value, disabled).

V rámci argumentu Button komponenty použijeme ES6 destructuring assignment pro získání vlastností. 
Z objektu vlastností získáme className a children, ostatní vlastnosti ponecháme zabalené v proměnné props pomocí spread operátoru. 
Nyní můžeme vytvořit JSX pro samotné tlačítko. Vlastnost className přidáme do tříd tlačítka. 
Pomocí children můžeme do tlačítka vložit libovolný obsah, který bude mezi párovými značkami <Button>. 
Všechny ostatní vlastnosti pomocí spread operátoru předáme přímo tlačítku.

V Counter komponentě v rámci JSX vrátíme hodnotu stavu count a vykreslíme Button komponenty, jimž předáme potřebné vlastnosti. 
Pro aktualizaci stavu využijeme vlastnost onClick, které předáme anonymní funkci (arrow function) a v ní zavoláme setCount.

\begin{flushleft}
  \textbf{Interakce v uživatelském prostředí}
\end{flushleft}

Pro vytvoření jakékoliv UI komponenty můžeme začít tvořit jak JSX, definici komponenty, nebo znovupoužitelný hook. 
My začneme naprogramováním vlastního hooku, který se odděleně postará o veškerou logiku seznamu.

Hook useDropdown bude mít 2 parametry -- výchozí hodnotu vybrané možnosti (defaultValue) a obslužnou funkci ke změně vybrané v možnosti v rodičovské komponentě (onChange). 
V rámci hooku nadefinujeme stavy selectedOption, isOpen a vygenerujeme unikátní identifikátor. 
Dále vytvoříme funkci handleOptionClick, která zajistí změnu vybrané možnosti, zavření seznamu a vypublikuje změnu hodnoty do rodičovské komponenty. 
Z hooku vracíme potřebné stavy a funkce ve formě objektu nebo pole -- pole musíme označit jako const.

Pokračujeme tvorbou JSX komponenty Dropdown, kde vložíme tlačítko a seznam možností. Otevření možností zajistíme přidáním onClick (což je vlastně MouseEventHandler). 
V anonymní funkci pak změníme stav pomocí isOpen na opačnou hodnotu. Abychom předešli event bubblingu, v rámci anynomní obslužné funkce zavoláme event.stopPropagation().

Seznam možností zobrazíme podmíněně na základě stavu isOpen. Pro vykreslení možností seznamu (options) použijeme JavaScriptovou funkci map uvnitř JavaScriptové hodnoty v JSX. 
V Reactu je důležité vždy při použití funkce map nastavit unikátní klíč (key) pro každou položku v seznamu. Tento klíč slouží k identifikaci jednotlivých prvků a optimalizaci procesu renderování. 
Pro vybrání konkrétní možnosti použijeme onClick, kterému předáme anonymní funkci. V anonymní funkci zavoláme funkci handleOptionClick hooku useDropdown s aktuální položkou ze seznamu.

Abychom uzavřeli jakýkoli aktuálně otevřený rozbalovací seznam na stránce po kliknutí mimo tento seznam, předáme kořenovému elementu dříve vytvořený unikátní identifikátor. 
Do useDropdown přidáme useEffect a díky němu budeme naslouchat na události pointerdown v DOM. Obslužná funkce pak zajistí zavření aktuálně otevřeného dropdownu.

Dropdown samozřejmě může mít i jiné vstupy, které povedou k lepší znovupoužitelnosti. 
Dynamické třídy ve formě JavaScriptu na element přidáme pomocí šablonových literálů (template literals) a JavaScriptové hodnoty.

\begin{flushleft}
  \textbf{Reaktivita, asynchronní operace}
\end{flushleft}

\begin{flushleft}
  \textbf{Tvorba formulářů, validace}
\end{flushleft}

\begin{flushleft}
  \textbf{Modularita, použití knihoven}
\end{flushleft}

\begin{flushleft}
  \textbf{Layout aplikace, routování}
\end{flushleft}