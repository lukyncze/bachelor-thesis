\subsubsection{React}

\begin{flushleft}
  \textbf{Instalace projektu}
\end{flushleft}

\begin{flushleft}
  \textbf{Správa stavů, předávání vlastností}
\end{flushleft}

Při implementaci jednoduchého čítače začneme tím, že vytvoříme Counter komponentu. Ta bude mít stav count a setter setCount pro tento stav.

Dále vytvoříme komponentu Button kvůli principu DRY a celkově znovupoužitelnosti kódu. 
Typ ButtonProps obsahuje vlastnosti, které můžeme tlačítku předat -- className, onClick a children. 
Díky tomu, že typ rozšiřuje ButtonHTMLAttributes<HTMLButtonElement>, můžeme předat do komponenty i další běžné atributy HTML tlačítek (např. type, value, disabled).

V rámci argumentu Button komponenty použijeme ES6 destructuring assignment pro získání vlastností. 
Z objektu vlastností získáme className a children, ostatní vlastnosti ponecháme zabalené v proměnné props pomocí spread operátoru. 
Nyní můžeme vytvořit JSX pro samotné tlačítko. Vlastnost className přidáme do tříd tlačítka. 
Pomocí children můžeme do tlačítka vložit libovolný obsah, který bude mezi párovými značkami <Button>. 
Všechny ostatní vlastnosti pomocí spread operátoru předáme přímo tlačítku.

V Counter komponentě v rámci JSX vrátíme hodnotu stavu count a vykreslíme Button komponenty, jimž předáme potřebné vlastnosti. 
Pro aktualizaci stavu využijeme vlastnost onClick, které předáme anonymní funkci (arrow function) a v ní zavoláme setCount.

\begin{flushleft}
  \textbf{Interakce v uživatelském prostředí}
\end{flushleft}

Pro vytvoření jakékoliv UI komponenty můžeme začít tvořit jak JSX, definici komponenty, nebo znovupoužitelný hook. 
V tomto případě, při vývoji komponenty rozevíracího seznamu, začneme naprogramováním vlastního hooku, který se odděleně postará o veškerou logiku seznamu.

Hook useDropdown bude mít 2 parametry -- výchozí hodnotu vybrané možnosti (defaultValue) a obslužnou funkci ke změně vybrané v možnosti v rodičovské komponentě (onChange). 
V rámci hooku nadefinujeme stavy selectedOption, isOpen a vygenerujeme unikátní identifikátor. 
Dále vytvoříme funkci handleOptionClick, která zajistí změnu vybrané možnosti, zavření seznamu a vypublikuje změnu hodnoty do rodičovské komponenty. 
Z hooku vracíme potřebné stavy a funkce ve formě objektu nebo pole -- pole musíme označit jako const.

Pokračujeme tvorbou JSX komponenty Dropdown, kde vložíme tlačítko a seznam možností. Otevření možností zajistíme přidáním onClick (což je vlastně MouseEventHandler). 
V anonymní funkci pak změníme stav pomocí isOpen na opačnou hodnotu. Abychom předešli event bubblingu, v rámci anynomní obslužné funkce zavoláme event.stopPropagation().

Seznam možností zobrazíme podmíněně na základě stavu isOpen. Pro vykreslení možností seznamu (options) použijeme JavaScriptovou funkci map uvnitř JavaScriptové hodnoty v JSX. 
V Reactu je důležité vždy při použití funkce map nastavit unikátní klíč (key) pro každou položku v seznamu. Tento klíč slouží k identifikaci jednotlivých prvků a optimalizaci procesu renderování. 
Pro vybrání konkrétní možnosti použijeme onClick, kterému předáme anonymní funkci. V anonymní funkci zavoláme funkci handleOptionClick hooku useDropdown s aktuální položkou ze seznamu.

Abychom uzavřeli jakýkoli aktuálně otevřený rozbalovací seznam na stránce, po kliknutí mimo tento seznam, předáme kořenovému elementu dříve vytvořený unikátní identifikátor. 
Do useDropdown přidáme useEffect a díky němu budeme naslouchat na události pointerdown v DOM. Obslužná funkce pak zajistí zavření aktuálně otevřeného dropdownu.

Dropdown samozřejmě může mít i jiné vstupy, které povedou k lepší znovupoužitelnosti. 
Dynamické CSS třídy ve formě JavaScriptu na element přidáme pomocí šablonových literálů (template literals) a JavaScriptové hodnoty.

\begin{flushleft}
  \textbf{Reaktivita, asynchronní operace}
\end{flushleft}

Následující komponenta bude demonstrovat využití reaktivity a asynchronních operací. Vytvoříme komponentu, která přeloží zadaný text do cílového jazyka. 
Začneme vytvořením komponenty Translator. Komponenta při změně stavů (zadaného textu uživatelem a výstupního jazyka) zavolá API, které vrátí přeložený text.
V rámci komponenty vytvoříme vnořené komponenty pro zadání vstupního textu, výběr jazyka a zobrazení výsledku.

Komponenta LanguageDropdown uživateli umožní vybrat jazyk, do kterého chce text přeložit. 
Díky vlastnosti onChange (callback funkci) aktualizujeme výstupní jazyk v rodičovské komponentě. 

Pokračujeme implementací komponenty TranslationInput, která bude sloužit k zadání vstupního textu přes textové pole. Aktuální hodnotu formulářového prvku nastavíme pomocí atributu value.
Po změně hodnoty textového pole, kterou získáme v události přes atribut onChange, aktualizujeme hodnotu vstupního textu v Translator komponentě. 
Abychom reaktivně aktualizovali výšku pole na základě obsahu, použijeme vlastní hook. 
Hook bude potřebovat referenci elementu, a tak vytvoříme ref, který přidáme na element textového pole.

Hook useAutosizeTextArea bude příjímat referenci na element. Dále také hodnotu textového pole, aby po jakékoli změně této hodnoty přepočítala výška pole. 
V rámci hooku vytvoříme useEffect, který se znovu zavolá při každé změně textAreaRef, nebo hodnoty textu. Následně v rámci těla hooku aktualizujeme výšku textového pole.

V Translator komponentě potřebujeme ukládat vstupní hodnotu a výstupní jazyk z vnořených komponent. Dále při každé změně těchto hodnot zavoláme API, k čemuž využijeme useEffect.
V rámci hooku definujeme asynchronní funkci handleTranslation, která pomocí fetch API odešle korektní HTTP POST požadavek na server. 
Pokud bychom definovali funkci mimo useEffect, museli bychom ji přidat do pole závislostí hooku.
Při úspěšné odpovědi aktualizujeme stav s přeloženým textem, v opačném případě nastavíme chybový stav.

Aby dotazování fungovalo, vytvoříme referenci delayTimerRef. V rámci těla useEffect hooku nejprve zrušíme předchozí časovač. 
Funkci handleTranslation zavoláme v callbacku funkce setTimeout, která umožní předejít dotazování serveru ihned po změně nějaké vstupní hodnoty. 
Výsledek funkce setTimeout uložíme do delayTimerRef.current. Nesmíme také zapomenout na zrušení časovače při zničení komponenty.

V okamžiku, kdy obdržíme odpověď ze serveru, zobrazíme přeložený text uživateli pomocí komponenty TranslationOutput. 
Předáme jí samotný výstupní text a další vstupní vlastnosti, na základě kterých pak podmíněně vykreslíme přeložený text, chybu nebo načítání.

\begin{flushleft}
  \textbf{Tvorba formulářů, validace}
\end{flushleft}

React sám o sobě poskytuje jen základní API pro správu formulářů. Disponuje však mnoha knihovnami, které tuto funkcionalitu rozšiřují. 
Mezi takové knihovny patří např. Formik, Redux Form nebo React Hook Form. V této sekci se zaměříme na tvorbu formulářů pomocí React Hook Form. 
Vytvoříme komponentu pro jednoduchou investiční kalkulaci. 
V rámci této komponenty naprogramujeme formulář pro zadání vstupních dat a komponentu výsledku kalkulace, která se zobrazí po potvrzení formuláře.

Začneme s reaktivním fomulářem, který bude přijímat počáteční hodnoty (defaultValues) a callback funkci handleFormSubmit pro předání hodnot formuláře do rodičovské komponenty. 
Strukturu formuláře popíšeme v typu InvestFormData. Pomocí hooku useForm z knihovny React Hook Form vytvoříme instanci formuláře, které předáme defaultValues a nastavíme reaktivní validaci. 
Následně z hooku dostameme funkce register, handleSubmit a formState, které poslouží ke správě formuláře.

Následně do JSX přidáme form s onSubmit atributem, kterému předáme funkci handleSubmit z React Hook Form. Do handleSubmit pak vložíme vstup handleFormSubmit v rámci nějž získáme aktuální hodnoty formuláře. 
Ve formuláři vytvoříme formulářové prvky, které propojíme s reaktivním formulářem pomocí funkce register. První argument představuje název formulářového prvku, druhý argument je validační objekt. 
V rámci range inputu potřebujeme HTML atributy min a max, díky kterým omezíme rozsah vstupních hodnot. Abychom mohli měli přístup k aktuální hodnotě range inputu, využijeme vlastnost value a onChange. 
Chyby formuláře získáme z formState a vykreslíme je pod formulářovými prvky. V neposlední řadě přidáme tlačítko s typem submit, které zajistí odeslání formuláře a zavolání callback funkce handleFormSubmit.

V rodičovské komponentě získáme aktuální hodnoty formuláře díky obslužné funkci handleFormSubmit. Pomocí funkce futureValuesCalculator získáme hodnoty, které následně vykreslíme v komponentě FutureValuesInfo. 
Tato komponenta obsahuje dvě vnořené komponenty FutureValueInfo, pro zobrazení jednotlivých výsledků. Hodnotu v JSX transformujeme pomocí JavaScriptové funkce.

\begin{flushleft}
  \textbf{Modularita, použití knihoven}
\end{flushleft}

V následující sekci vytvoříme webovou hru, ve které cílem uživatele je uhádnout název státu na základě poskytnutých nápovědí. Práci si ulehčíme pomocí externích knihoven a služeb.
Postupně se bude odkrývat 8 nápovědí, které by měly pomoci s uhádnutím daného státu. Klíčovým prvkem je textové pole, přes které uživatel zadává názvy hádaných zemí a tlačítko pro potvrzení. 
Součástí také bude seznam zemí, které uživatel hádal a modální okna sloužící k vyhodnocení hry.

Začneme s implementací rodičovské komponenty, která získá země z veřejného API. Naprogramujeme hook useAllCountries, který bude vracet data (countries), chybu a stav načítání. 
V rámci useEffect zavoláme asynchronní funkci fetchCountriesData. Uvnitř funkce fetchCountriesData zavoláme getAllCountries. 
Jde o převzatý requestHandler s využítím knihovny axios, jenž umožní otypování příchozí odpovědi. Po ošetření chyb aktualizujeme patřičné stavy, které následně z hooku vrátíme. 
Implementaci načítacích a chybových stavů nebo rušení asynchronních dotazů nám do značné míry může ulehčit knihovna react-query.

Následně v rámci rodičovské komponenty podmíněně vykreslíme dané komponenty. V případě chyby komponentu ErrorAlert. Pokud ze serveru úspěšně dostaneme země, tak vykreslíme komponentu CountryGuesser. 
Pokud nevykreslíme ani jednu z předchozích komponent, zobrazíme LoadingSkeleton.

Komponenta CountryGuesser bude vyhodnocovat průběh hry a zobrazovat jednotlivé herní prvky. Začneme definicí stavů a inicializujeme náhodnou zemi (randomCountry), kterou bude uživatel hádat. 
Dále použijeme hook useCountryFlagPolyfill, který při namontování komponenty zajistí podporu zobrazení ikon vlajek v prohlížečích, které to přímo nepodporují. 
Prohlížeč pak však musí podporovat emojis a webové fonty. Pokračujeme implementací obslužných metod handleEvaluateGuessAndUpdateState a handleSetInitialState, které budou sloužit k aktualizaci stavu hry. 
V rámci JSX pak vykreslíme jednotlivé herní prvky a modální okna při výhře či prohře.

Hook useCountryFlagPolyfill zavolá funkci polyfillCountryFlagEmojis, která do HTML hlavičky přidá webový font Twemoji Country Flags. Aby se font využil, přidáme jej do CSS stylů.

Úkolem komponenty HintBoxes bude postupné vykreslení nápověd. Na základě vstupu randomCountry vytvoříme pole nápověd. V JSX pak iterujeme přes pole nápověd a vykreslíme jednotlivé nápovědy. 
Jednotlivé komponenty HintBox pak dynamicky vykreslí název a SVG ikonu nápovědy, textovou nápovědu, případně obrázek vlajky státu.

Klíčová komponenta CountryGuessInput pak uživateli umožní zadat svůj tip. 
Začneme s JSX, kde vytvoříme formulářový prvek pro zadání názvu země, potvrzovací tlačítko a podmenu textového pole, které zobrazí nejpodobnější země na základě zadaného textu (filtrované země). 
Přidáme obslužné metody pro akce a události nad formulářem, které následně doimplementujeme.

Ve funkční komponentě CountryGuessInputComponent na základě vstupu countries získáme pole všech zemí bez těch, které uživatel již hádal (countriesWithoutAlreadyGuessed). 
Poté definujeme a inicializujeme ostatní stavy komponenty. 
Při kliknutí na tlačíko se zavolá funkce handleGuessButtonClick, která volá obslužnou funkci handleEvaluateGuessAndUpdateState v rodičovské komponentě a také funkci handleChangeSelectedGuess. 
Funkce handleChangeSelectedGuess aktualizuje aktuální tip, filtrované země a uzavře podmenu. Funkce handleInputChange převede tip uživatele do daného formátu, poté aktualizuje aktuální tip a filtrované země. 
Ovládání formulářového prvku pomocí klávesnice umožní funkce handleKeyDown.

Pomocná funkce updateGuessAndFilteredCountries získá aktuálně filtrované země na základě uživatelova tipu. Následně aktualizuje stavy currentGuess, isValidGuess a filteredCountries. 
Funkce clampSelectedGuessIndex zajistí, aby index uživatelem vybrané země byl v požadovaném rozmezí (0 až počet filtrovaných zemí). 
Pro aktualizaci vlastnosti selectedGuessIndex slouží funkce changeSelectedGuessIndex, která index aktualizuje o hodnotu předanou v argumentu.
V neposlední řadě funkce convertToFormattedGuess převede tip uživatele tak, aby začínal velkým písmenem a zbytek řetezce byl složen z malých písmen.

Ke zobrazení všech již hádaných zemí uživatelem vytvoříme komponentu GuessedCountriesList. Ze vstupních vlastností countries, guessedCountries a randomCountry získáme proměnnou enrichedGuessedCountries. 
Jde o uživatelem hádané země s vlajkou a vzdáleností od randomCountry. K převodu využijeme JavaScriptové funkce z jiného souboru. 
K vypočtení vzdálenosti použijeme knihovnu calculate-distance-between-coordinates, která obsahuje funkci getDistanceBetweenTwoPoints. Jednotlivé prvky pole enrichedGuessedCountries pak vykreslíme v rámci JSX.

Nakonec vytvoříme modální okna, která se zobrazí při výhře či prohře. Stavy isWinModalOpen a isLoseModalOpen aktualizujeme v rámci funkce handleEvaluateGuessAndUpdateState v CountryGuesser. 
Na základě těchto stavů pak podmíněně vykreslíme daná modální okna. Oběma modálům předáme randomCountry a obslužnou funkci handleClose. Výhernímu modálu také počet potřebných pokusů. 
V jednotlivých komponentách (WinModal, LoseModal) vykreslíme komponentu BaseModal, která bude sloužit jako šablona pro obě okna. Do této komponenty vždy předáme titulek, obsah a obslužnou metodu handleClose. 
BaseModal následně v JSX vykreslí základní strukturu modálního okna, s dynamickými možnostmi pro titulek, obsah a obslužnou metodu handleClose.

\begin{flushleft}
  \textbf{Layout aplikace, routování}
\end{flushleft}

Layout aplikace bude rozdělen do tří částí: hlavičky, patičky a samotného obsahu, v nemž se vykreslí jednotlivé komponenty. Uživatel bude mít možnost přepínání mezi jednotlivými stránkami pomocí navigačního menu.

Pro routování využijeme knihovnu react-router-dom. Začneme vytvořením souboru s cestami (routes). Následně v kořeni aplikace vytvoříme router pomocí předem definovaných cest aplikace. 
Router vytvoříme díky dvěma pomocným funkcím k tomu určených: createBrowserRouter a createRoutesFromElements. Pokračujeme přiřazením routeru do kořenové komponenty aplikace, konkrétně do poskytovatele routeru.

Hlavní komponenta AppLayout pak v JSX vykreslí hlavičku, patičku a dynamický obsah dle aktuální URL adresy, jenž vykreslí komponenta Outlet.

V hlavičce aplikace se budou nacházet odkazy na jednotlivé stránky. My se inspirujeme architekturou a vzhledem navigačního menu Flowbite. 
V rámci komponenty Header vypíšeme všechny cesty aplikace pomocí komponenty NavLink z knihovny react-router-dom. Ta nám umožní v atributu className získat vlastnost isActive, která indikuje, zda je odkaz aktivní. 
Pro korektní nastavení aria-current atributu použijeme hook useLocation, který nám umožní získat aktuální URL.

Mobilní navigaci a barevný režim implementujeme díky stavům isMobileNavOpen a isDarkMode. Informaci o tom, zda má uživatel zapnutý tmavý režim budeme ukládat do LocalStorage v prohlížeči. 
K tomu použijeme useEffect hook, který při změně stavu isDarkMode přidá patřičný data-mode a provede aktualizaci LocalStorage.