\subsubsection{React}

\begin{flushleft}
  \textbf{Instalace projektu}
\end{flushleft}

\begin{flushleft}
  \textbf{Správa stavů, předávání vlastností}
\end{flushleft}

Při implementaci jednoduchého čítače začneme tím, že vytvoříme Counter komponentu. Ta bude mít stav count a setter setCount pro tento stav.

Dále vytvoříme komponentu Button kvůli principu DRY a celkově znovupoužitelnosti kódu. 
Typ ButtonProps obsahuje vlastnosti, které můžeme tlačítku předat -- className, onClick a children. 
Díky tomu, že typ rozšiřuje ButtonHTMLAttributes<HTMLButtonElement>, můžeme předat do komponenty i další běžné atributy HTML tlačítek (např. type, value, disabled).

V rámci argumentu Button komponenty použijeme ES6 destructuring assignment pro získání vlastností. 
Z objektu vlastností získáme className a children, ostatní vlastnosti ponecháme zabalené v proměnné props pomocí spread operátoru. 
Nyní můžeme vytvořit JSX pro samotné tlačítko. Vlastnost className přidáme do tříd tlačítka. 
Pomocí children můžeme do tlačítka vložit libovolný obsah, který bude mezi párovými značkami <Button>. 
Všechny ostatní vlastnosti pomocí spread operátoru předáme přímo tlačítku.

V Counter komponentě v rámci JSX vrátíme hodnotu stavu count a vykreslíme Button komponenty, jimž předáme potřebné vlastnosti. 
Pro aktualizaci stavu využijeme vlastnost onClick, které předáme anonymní funkci (arrow function) a v ní zavoláme setCount.

\begin{flushleft}
  \textbf{Interakce v uživatelském prostředí}
\end{flushleft}

Pro vytvoření jakékoliv UI komponenty můžeme začít tvořit jak JSX, definici komponenty, nebo znovupoužitelný hook. 
My začneme naprogramováním vlastního hooku, který se odděleně postará o veškerou logiku seznamu.

Hook useDropdown bude mít 2 parametry -- výchozí hodnotu vybrané možnosti (defaultValue) a obslužnou funkci ke změně vybrané v možnosti v rodičovské komponentě (onChange). 
V rámci hooku nadefinujeme stavy selectedOption, isOpen a vygenerujeme unikátní identifikátor. 
Dále vytvoříme funkci handleOptionClick, která zajistí změnu vybrané možnosti, zavření seznamu a vypublikuje změnu hodnoty do rodičovské komponenty. 
Z hooku vracíme potřebné stavy a funkce ve formě objektu nebo pole -- pole musíme označit jako const.

Pokračujeme tvorbou JSX komponenty Dropdown, kde vložíme tlačítko a seznam možností. Otevření možností zajistíme přidáním onClick (což je vlastně MouseEventHandler). 
V anonymní funkci pak změníme stav pomocí isOpen na opačnou hodnotu. Abychom předešli event bubblingu, v rámci anynomní obslužné funkce zavoláme event.stopPropagation().

Seznam možností zobrazíme podmíněně na základě stavu isOpen. Pro vykreslení možností seznamu (options) použijeme JavaScriptovou funkci map uvnitř JavaScriptové hodnoty v JSX. 
V Reactu je důležité vždy při použití funkce map nastavit unikátní klíč (key) pro každou položku v seznamu. Tento klíč slouží k identifikaci jednotlivých prvků a optimalizaci procesu renderování. 
Pro vybrání konkrétní možnosti použijeme onClick, kterému předáme anonymní funkci. V anonymní funkci zavoláme funkci handleOptionClick hooku useDropdown s aktuální položkou ze seznamu.

Abychom uzavřeli jakýkoli aktuálně otevřený rozbalovací seznam na stránce po kliknutí mimo tento seznam, předáme kořenovému elementu dříve vytvořený unikátní identifikátor. 
Do useDropdown přidáme useEffect a díky němu budeme naslouchat na události pointerdown v DOM. Obslužná funkce pak zajistí zavření aktuálně otevřeného dropdownu.

Dropdown samozřejmě může mít i jiné vstupy, které povedou k lepší znovupoužitelnosti. 
Dynamické třídy ve formě JavaScriptu na element přidáme pomocí šablonových literálů (template literals) a JavaScriptové hodnoty.

\begin{flushleft}
  \textbf{Reaktivita, asynchronní operace}
\end{flushleft}

Následující komponenta bude demonstrovat využití reaktivity a asynchronních operací. Vytvoříme komponentu, která přeloží zadaný text do cílového jazyka. 
Začneme vytvořením komponenty Translator. Komponenta při změně stavů (zadaného textu uživatelem a výstupního jazyka) zavolá API, které vrátí přeložený text.
V rámci komponenty vytvoříme vnořené komponenty pro zadání vstupního textu, výběr jazyka a zobrazení výsledku.

Komponenta LanguageDropdown uživateli umožní vybrat jazyk, do kterého chce text přeložit. 
Díky vlastnosti onChange (callback funkce) aktualizujeme výstupní jazyk v rodičovské komponentě. 

Pokračujeme implementací komponenty TranslationInput, která bude sloužit k zadání vstupního textu přes textové pole. Aktuální hodnotu formulářového prvku nastavíme pomocí atributu value.
Po změně hodnoty textového pole, kterou získáme v události přes atribut onChange, aktualizujeme hodnotu vstupního textu v Translator komponentě. 
Abychom reaktivně aktualizovali výšku pole na základě obsahu, použijeme vlastní hook. 
Hook bude potřebovat referenci elementu, a tak vytvoříme ref, který přidáme na element textového pole.

Hook useAutosizeTextArea bude příjímat referenci na element. Dále také hodnotu textového pole, aby po jakékoli změně této hodnoty přepočítala výška pole. 
V rámci hooku vytvoříme useEffect, který se znovu zavolá při každé změně textAreaRef, nebo hodnoty textu. Následně v rámci těla hooku aktualizujeme výšku textového pole.

V Translator komponentě potřebujeme ukládat vstupní hodnotu a výstupní jazyk z vnořených komponent. Dále při každé změně těchto hodnot zavoláme API, k čemuž využijeme useEffect.
V rámci hooku definujeme asynchronní funkci handleTranslation, která pomocí fetch API odešle HTTP POST požadavek na server. 
Pokud bychom definovali funkci mimo useEffect, museli bychom ji přidat do pole závislostí hooku.
Při úspěšné odpovědi aktualizujeme stav s přeloženým textem, v opačném případě nastavíme chybový stav.

Aby dotazování fungovalo, vytvoříme referenci delayTimerRef. V rámci těla useEffect hooku nejprve zrušíme předchozí časovač. 
Funkci handleTranslation zavoláme v callbacku funkce setTimeout, která umožní předejít dotazování serveru ihned po změně nějaké vstupní hodnoty. 
Výsledek funkce setTimeout uložíme do delayTimerRef.current. Nesmíme také zapomenout na zrušení časovače při zničení komponenty.

V okamžiku, kdy obdržíme odpověď ze serveru, zobrazíme přeložený text uživateli pomocí komponenty TranslationOutput. 
Předáme jí samotný výstupní text a další vstupní vlastnosti, na základě kterých pak podmíněně vykreslíme přeložený text, chybu nebo načítání.

\begin{flushleft}
  \textbf{Tvorba formulářů, validace}
\end{flushleft}

\begin{flushleft}
  \textbf{Modularita, použití knihoven}
\end{flushleft}

\begin{flushleft}
  \textbf{Layout aplikace, routování}
\end{flushleft}