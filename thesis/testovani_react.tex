\subsubsection{React}

\begin{flushleft}
  \textbf{Instalace projektu}
\end{flushleft}

\begin{flushleft}
  \textbf{Správa stavů, předávání vlastností}
\end{flushleft}

Při implementaci jednoduchého čítače začneme tím, že vytvoříme Counter komponentu. Ta bude mít stav count a setter setCount pro tento stav.

Dále vytvoříme komponentu Button kvůli principu DRY a celkově znovupoužitelnosti kódu. 
Typ ButtonProps obsahuje vlastnosti, které můžeme tlačítku předat -- className, onClick a children. 
Díky tomu, že typ rozšiřuje ButtonHTMLAttributes<HTMLButtonElement>, můžeme předat do komponenty i další běžné atributy HTML tlačítek (např. type, value, disabled).

V rámci argumentu Button komponenty použijeme ES6 destructuring assignment pro získání vlastností. 
Z objektu vlastností získáme className a children, ostatní vlastnosti ponecháme zabalené v proměnné props pomocí spread operátoru. 
Nyní můžeme vytvořit JSX pro samotné tlačítko. Vlastnost className přidáme do tříd tlačítka. 
Pomocí children můžeme do tlačítka vložit libovolný obsah, který bude mezi párovými značkami <Button>. 
Všechny ostatní vlastnosti pomocí spread operátoru předáme přímo tlačítku.

V Counter komponentě v rámci JSX vrátíme hodnotu stavu count a vykreslíme Button komponenty, jimž předáme potřebné vlastnosti. 
Pro aktualizaci stavu využijeme vlastnost onClick, které předáme anonymní funkci (arrow function) a v ní zavoláme setCount.

\begin{flushleft}
  \textbf{Interakce v uživatelském prostředí}
\end{flushleft}

\begin{flushleft}
  \textbf{Reaktivita, asynchronní operace}
\end{flushleft}

\begin{flushleft}
  \textbf{Tvorba formulářů, validace}
\end{flushleft}

\begin{flushleft}
  \textbf{Modularita, použití knihoven}
\end{flushleft}

\begin{flushleft}
  \textbf{Layout aplikace, routování}
\end{flushleft}