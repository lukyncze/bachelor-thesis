\section{Práce se zdroji}

Veškeré informace, které odněkud převezmete, je třeba ozdrojovat. Pozor, nejde jen o~doslovné citace nebo přejaté obrázky či tabulky, ale i~o~statistické informace, definice, vysvětlení pojmů, metody, postupy a další tvrzení či návody. Pokud přejímáme pouze samotnou informaci (nikoliv doslovnou citaci), hovoříme o parafrázi.

\subsection{Seznam použité literatury}

V seznamu literatury by mělo být vše, co jste použili, a logicky na každou položku seznamu by někde v textu měl být odkaz (čímž sdělíme, kde jsme zdroj použili).

Položky v seznamu literatury tvořte na webu \iadresa{https://citacepro.com}\cite{generatorcitaci}. Tento web má základní a komerční variantu, při přihlašování přes univerzitní CRO používáte tu komerční. Položky v seznamu literatury na straně \pageref{chap:literatura} byly tvořeny právě na tomto webu.



\subsection{Citace}

Citace musí být vždy doslovná. Je špatně, kdy v citaci něco pozměníte. Pokud chcete část citace vynechat, je to sice možné, ale musí to být patrné (například použijeme výpustky, tedy tři tečky, případně při zkrácení dlouhého výpisu připojíme na konec informaci, že výpis byl zkrácen). Citace nesmí být osekána natolik, že by byl původní význam pozměněn. Odkaz na zdroj musí být blízko, bylo by chybou ho umístit třeba až za několik odstavců.

Ukázka citace:
\bigskip


\uv{\itshape
Za určitých okolností však text můžete kopírovat bez souhlasu autora. Vychází se obecně z předpokladu, že každá osoba má právo reprodukovat text bezplatně a~hlavně i~bez souhlasu autora výhradně však při splnění následujících podmínek:

\noindent
$\!\!$-- text je kopírován pro své osobní použití a na vlastních kopírovacích zařízeních.}\cite{autorskepravo}
\bigskip

Citaci a odkaz na zdroj můžeme zakomponovat do věty, třeba takto:
\bigskip

V knize \cite{autorskepravo} se můžeme dočíst:
\uv{\itshape
Za určitých okolností však text můžete kopírovat bez souhlasu autora. Vychází se obecně z předpokladu, že každá osoba má právo reprodukovat text bezplatně a~hlavně i~bez souhlasu autora výhradně však při splnění následujících podmínek:

\noindent
$\!$-- text je kopírován pro své osobní použití a na vlastních kopírovacích zařízeních.}
\bigskip


Nebo lze použít vhodné \uv{zužující prostředí}, které nahradí nutnost uvedení ohraničujících uvozovek:


\begin{quote}\itshape
Za určitých okolností však text můžete kopírovat bez souhlasu autora. Vychází se obecně z předpokladu, že každá osoba má právo reprodukovat text bezplatně a hlavně i bez souhlasu autora výhradně však při splnění následujících podmínek:

{}-- text je kopírován pro své osobní použití a na vlastních kopírovacích zařízeních.\cite{autorskepravo}
\end{quote}

Citovat můžeme například definice nebo cokoliv, kde by nemělo smysl vymýšlet vlastní formulace (čímž není myšlena nemoc zvaná lenora). S citacemi to nepřeháníme, závěrečná práce by měla být především dílem autorovým.

Speciálním typem citace je také přejatý obrázek či tabulka.


\subsection{Parafráze}

Parafráze využívá položku v seznamu literatury pouze jako zdroj informací, nikoliv jako zdroj samotného textu či obrázku. Bývá podstatně kratší (nebo sice delší, ale doplněný vlastními myšlenkami a formulacemi) a také lze parafrázovat z více zdrojů najednou (protože je velmi praktické tutéž informaci ověřit v několika zdrojích). Ostatně, ze zdroje nemusíme nutně potřebovat kompletní souhrn informací, ale jen něco.

Pokud odněkud zkopírujeme pár vět a v nich sem tam pozměníme slovo či jinak přeformulujeme, nejde o parafrázi, ale o chybnou citaci. To je samozřejmě špatně a může to vést i k zamítnutí práce.

Ukázka parafráze:
\bigskip

Pokud si zkopírujeme dokument na vlastním zařízení (třeba na multifunkční tiskárně nebo v případě elektronického dokumentu na počítači nebo vyfotíme mobilem) a použijeme pro své vlastní účely, nepotřebujeme souhlas autora.\cite{autorskepravo}
\bigskip

Nebo opět zakomponujeme:
\bigskip

Podle \cite{autorskepravo} nepotřebujeme souhlas autora, pokud si zkopírujeme dokument na vlastním zařízení (třeba na multifunkční tiskárně nebo v případě elektronického dokumentu na počítači nebo vyfotíme mobilem) a použijeme pro své vlastní účely.
\bigskip


Parafráze jsou v závěrečné práci velmi běžná forma. Používáme je při vysvětlování pojmů (pokud není nutné citovat), uvádění postupů, statistických údajů či jiných číselných údajů,\dots

Pokud z téhož zdroje vycházíme v několika za sebou následujících odstavcích, můžeme uvést zdroje jen na jednom místě. Ale pouze v případě, že ty odstavce nejsou rozděleny nějakým nadpisem.




