\subsubsection{Svelte}

\begin{flushleft}
  \textbf{Instalace projektu}
\end{flushleft}

\begin{flushleft}
  \textbf{Správa stavů, předávání vlastností}
\end{flushleft}

Prvním krokem k vytvoření jednoduchého čítače bude definice komponenty Counter s reaktivním stavem count.

Dále vytvoříme komponentu Button z důvodu dodržování principu DRY a efektivnějšímu znovupoužití kódu v budoucnu.
Komponenta bude přijímat vlastnosti className a onClick. ClassName rozšíří CSS třídy tlačítka a onClick bude obsahovat obslužnou funkci, která se zavolá při kliknutí na tlačítko.
Nyní do šablony přidáme tlačítko a předáme mu vlastnosti className a onClick.
Svelte umožňuje zachytit všechny nedefinované vlastnosti do proměnné \$\$restProps. Proměnnou \$\$restProps tedy pomocí spread operátoru předáme tlačítku a tím jej obohatíme o další vlastnosti. 
Obsah tlačítka, který definujeme mezi párovými značkami Button, vykreslíme pomocí komponenty slot.

V Counter komponentě pak v rámci šablony vykreslíme stav count a Button komponenty, kterým předáme příslušné vlastnosti. 
Pro aktualizaci stavu count použijeme obslužné funkce, v nichž přímo modifikujeme count.

\begin{flushleft}
  \textbf{Interakce v uživatelském prostředí}
\end{flushleft}

\begin{flushleft}
  \textbf{Reaktivita, asynchronní operace}
\end{flushleft}

\begin{flushleft}
  \textbf{Tvorba formulářů, validace}
\end{flushleft}

\begin{flushleft}
  \textbf{Modularita, použití knihoven}
\end{flushleft}

\begin{flushleft}
  \textbf{Layout aplikace, routování}
\end{flushleft}